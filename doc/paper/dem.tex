\section{The Empirical Dust Attenuation Model} \label{sec:dem}
In this section we describe the Empirical Dust Attenuation (\eda) model, a
flexible prescription for applying dust attenuation curves to galaxy
populations. With the \eda, we can apply a wide variety of dust attenuation and
include correlation between dust attenuation and physical galaxy properties. 
Later, we demonstrate that we can accurately reproduce SDSS observations with
the \eda~and use it to test galaxy formation models and shed light on dust in
galaxies. 

We define the dust attenuation curve, $A(\lambda)$, as 
\begin{equation} \label{eq:full_atten}
    F_o (\lambda) = F_i (\lambda) 10^{-0.4 A(\lambda)}
\end{equation}
where $F_o$ is the observed flux and $F_i$ is the intrinsic flux. We normalize
the attenuation at the $V$ band, 
\begin{equation} 
    A(\lambda) = A_V \frac{k(\lambda)}{k_V}
\end{equation}
so that $A_V$ determines the amplitude of the attenuation, while $k(\lambda)$
determines the wavelength dependence. 

To determine $A(\lambda)$ for each galaxy, we first assign $A_V$ using the slab
model from~\citep[\eg][]{somerville1999, somerville2012} where $A_V$ is
calculated based on the inclination of the galaxy, $i$, and its optical depth, $\tau_V$: 
\begin{equation} \label{eq:slab}
    A_V = -2.5 \log \left[ \frac{1 - e^{-\tau_V\,\sec i}}{\tau_V\,\sec i} \right].
\end{equation}
For each galaxy, we uniformly sample $i$ from 0 to $\pi/2$. Then in $\tau_V$ we
include the correlation between $A_V$ and galaxy properties ($M_*$ and $\ssfr$),
found in both observations and simulations~\citep[\eg][]{narayanan2018, salim2020}. 
We parameterize $\tau_V$ with a simple and linear $M_*$ and $\ssfr$ dependence:
\begin{equation} \label{eq:tauv}
    \tau_V(M_*, \sfr) = \mtaum \log \left(\frac{M_*}{10^{10} M_\odot}\right) +
    \mtaus \log \left(\frac{\ssfr}{10^{-10}yr^{-1}}\right) + c_\tau.
\end{equation}
$\mtaum$, $\mtaus$, and $c_\tau$ represent the $M_*$ dependence, the $\ssfr$
dependence, and overall amplitude of $\tau_V$. Since $\tau_V$ is
optical depth, we impose a $\tau_V \ge 0$ limit.

The slab model is a naive approximation. In reality, $A_V$ for a galaxy depends on complexities 
of its star-to-dust geometry, variations in the extinction curves, and other
properties beyond just inclination and $\tau_V$. The purpose of the \eda,
however, is not to accurately model dust attenuation for individual galaxies,
but rather to accurately represent the distribution of dust attenuation for galaxy
populations. In this regard, the slab model is consistent with the
correlation between $A_V$ and $i$ found in the literature: edge-on galaxies
have higher $A_V$ than face-on galaxies~\citep[\eg][]{conroy2010, wild2011,
battisti2017, salim2020}. More
importantly, the $A_V$ distribution, $p(A_V)$, produced using the slab model with
uniformly sampled inclinations closely matches the $p(A_V)$ of our SDSS sample
(Figure~\ref{fig:av_dist}). Also, replacing the slab model with a more flexible
prescription for sampling $A_V$ does not significant impact our analysis 
(Appendix~\ref{sec:nonslab}). Therefore, we conclude that the slab model is a
sufficient, flexible empirical prescription for sampling $A_V$. 

For the wavelength dependence of the attenuation curve, $k(\lambda)$, we
use the \cite{noll2009} parameterization: 
\begin{equation} \label{eq:noll}
    k(\lambda) = \left(k_{\rm Cal}(\lambda) + D(\lambda)\right) \left(
    \frac{\lambda}{\lambda_V} \right)^\delta.
\end{equation}
Here $k_{\rm Cal}(\lambda)$ is the \cite{calzetti2001} curve: 
\[
    k_{\rm Cal}(\lambda) = 
    \begin{cases} 
        2.659 (-1.857 + 1.040/\lambda) + R_V, & 6300 A \le \lambda \le
        22000 A \\ 
        2.659 (-2.156 + 1.509/\lambda - 0.198/\lambda^2 + 0.011/\lambda^3) +
        R_V & 1200 A \le \lambda \le 6300 A
    \end{cases}
\]
where $\lambda_V = 5500 A$ is the $V$ band wavelength and $\delta$ is the slope
offset of the attenuation curve from $k_{\rm Cal}$. Since $\delta$ correlates 
with galaxy properties~\citep[\eg][]{leja2017, salim2018},
we parameterize $\delta$ with a similar $M_*$ and $\ssfr$ dependence as
$\tau_V$:  
\begin{align} \label{eq:delta}
    \delta(M_*, \sfr) &= \mdeltam \log \left(\frac{M_*}{10^{10}
    M_\odot}\right) + \mdeltas \log \left(\frac{\ssfr}{10^{-10}yr^{-1}}\right)
    + c_\delta.
\end{align}
% Although a number of works have found correlation between the attenuation
% curve slope and inclination~\citep{wild2011, chevallard2013, battisti2017b},
% \cite{salim2020}, most recently, found that the driver of this trend is the
% relationship between $A_V$ and slope. We therefore do not include an
% inclination dependence in $\delta$. 
$D(\lambda)$ in Eq.~\ref{eq:noll} is the UV dust bump, which we parameter using
the standard Lorentzian-like Drude profile:
\begin{equation}
    D(\lambda) = \frac{E_b(\lambda~\Delta \lambda)^2}{(\lambda^2 -
    \lambda_0^2)^2 + (\lambda~\Delta \lambda)^2}
\end{equation}
where $\lambda_0$, $\Delta \lambda$, and $E_b$ are the central wavelength,
FWHM, and strength of the bump, respectively. We assume fixed $\lambda_0 = 2175
A$ and $\Delta \lambda = 350A$. \cite{kriek2013} and \cite{tress2018} find
that $E_b$ correlates with the $\delta$ for star-forming galaxies $z\sim2$.
\cite{narayanan2018} confirmed this dependence in simulations. For the
\eda~model, we assume a fixed relation between $E_B$ and $\delta$ from \cite{kriek2013}: 
$E_b = -1.9~\delta + 0.85$. Allowing the slope and amplitude of the $E_B$ and
$\delta$ relation to vary does {\em not} impact our results; however, we do not
derive any meaningful constraints on them either. In
Table~\ref{tab:free_param}, we list and describe all of the free parameters in
the \eda. 

%Next, to attenuate the galaxy SEDs, we apply $A(\lambda)$ we separately to the
%star light and nebular emssion: 
%\begin{equation} \label{eq:full_atten}
%    F_o (\lambda) = F^{\rm star}_i (\lambda) 10^{-0.4 A(\lambda)} + F^{\rm
%    neb}_i (\lambda) 10^{-0.4 A_{\rm neb}(\lambda)}.
%\end{equation}
%We parameterize
%\begin{equation}
%    A_{\rm neb}(\lambda) = f_{\rm neb}  A(\lambda) 
%\end{equation} 
%and allow $f_{\rm neb}$ to vary freely. 

$\ssfr$ of galaxies are used to calculate $\tau_V$ and $\delta$ in
Eqs.~\ref{eq:tauv} and~\ref{eq:delta}. However, due to mass and temporal resolutions,
some galaxies in the simulations have $\sfr=0$ --- \ie~an unmeasurably low
SFR~\citep{hahn2019c}. They account for 17, 19, 9\% of galaxies
in SIMBA, TNG, and EAGLE, respectively. Since Eqs.~\ref{eq:tauv}
and~\ref{eq:delta} depend on $\log\ssfr$, they cannot be used in the equations
to derive $\tau_V$ and $\delta$ for these galaxies. To account for this issue,
we assign $\sfr_{\rm min}$, the minimum non-zero $\sfr$ in the simulations, to
$\sfr=0$ galaxies when calculating $\tau_V$ and $\delta$. For SIMBA, TNG, and
EAGLE, $\sfr_{\rm min}=0.000816$, $0.000268$, and $0.000707 M_\odot/yr$. Although 
this assumes that $\sfr=0$ galaxies have similar dust properties as the galaxies 
with $\sfr = \sfr_{\rm min}$, since the simulations have very low $\sfr_{\rm min}$ 
we expect galaxies with $\sfr = \sfr_{\rm min}$ to have little recent
star-formation and low gas mass, similar to $\sfr=0$ galaxies. 

%Since $\sfr=0$ galaxies do not account for a large fraction of our simulated galaxies, we directly sample their observables ($G, R, NUV$, and $FUV$) from the distribution of observables for SDSS quiescent galaxies. This way, we ensure that the attenuation of $\sfr=0$ galaxies does not impact the rest of the \eda~parameters. In Appendix~\ref{sec:res}, we discuss the resolution effects in more detail and demonstrate that our results are \emph{not} impacted by other prescriptions for attenuating $\sfr=0$ galaxies.

In practice, to apply the \eda~to a simulated galaxy population, we begin by
assigning a uniformly sample inclination, $i$, to each galaxy. Then $\tau_V$,
and $\delta$ are calculated for the galaxy based on its $i$, $M_*$, $\ssfr$ and
the \eda~parameters. From $\tau_V$ and $\delta$, we determine $A_V$ and
$k(\lambda)$, which together gives $A(\lambda)$.  Afterwards, we attenuate the
galaxy SEDs using Eq.~\ref{eq:full_atten} and use the attenuated SEDs to
calculate the observables: $G, R, NUV$, and $FUV$
absolute magnitudes. In Figure~\ref{fig:dem_av}, we present attenuation curves,
$A(\lambda)$, generated by the \eda~for galaxies with different $\sfr$ and $M_*$: 
star-forming ($\sfr=10^{0.5}M_\odot/yr$) with low mass ($10^{10}M_\odot$;
blue), with high mass ($10^{11}M_\odot$; green) and quiescent
($\sfr=10^{-2}M_\odot/yr$) with low mass ($10^{10}M_\odot$; orange), with high
mass ($10^{11}M_\odot$; red). All galaxies are edge-on (\ie~$i=0$) and we use
\eda~parameters: $\{\mtaum, \mtaus, c_\tau, \mdeltam, \mdeltas, c_\delta\} =
\{2., -2., 2., -0.1, -0.1, -0.2\}$, arbitrarily chosen within the prior range
listed in Table~\ref{tab:free_param}. For comparison, we include the
\cite{calzetti2001} attenuation curve. Even for only edge-on galaxies, the
\eda~produces attenuation curves with a wide range of amplitude and
slope to galaxies based on their physical properties. 

\begin{figure}
\begin{center}
    \includegraphics[width=0.6\textwidth]{figs/dems.pdf}
    \caption{\label{fig:dem_av}
    Attenuation curves, $A(\lambda)$, generated by the Empircial Dust Attenuation (\eda)
    model for galaxies with different $\sfr$ and $M_*$. We include $A(\lambda)$ for 
    star-forming ($\sfr=10^{0.5}M_\odot/yr$) low mass galaxy ($10^{10}M_\odot$;
    blue), high mass galaxy ($10^{11}M_\odot$; green) and quiescent
    ($\sfr=10^{-2}M_\odot/yr$) low mass galaxy ($10^{10}M_\odot$; orange), 
    high mass galaxy ($10^{11}M_\odot$; red). All galaxies are edge-on
    (\ie~$i=0$) and we use \eda~parameter values near the center of our prior
    range:$\{\mtaum, \mtaus, c_\tau, \mdeltam, \mdeltas, c_\delta\} = \{2.,
    -2., 2., -0.1, -0.1, -0.2\}$ (Table~\ref{tab:free_param}). For comparison,
    we include the \cite{calzetti2001} attenuation curve. In the \eda, the
    amplitude and slope of $A(\lambda)$ depend on $M_*$, and $\ssfr$ (Eqs.~\ref{eq:tauv}
    and~\ref{eq:delta}). {\em The \eda~provides a flexible model for
    applying dust attenuation to galaxies based on their physical properties}.
    } 
\end{center}
\end{figure}


%%%%%%%%%%%%%%%%%%%%%%%%%%%%%%%%%%%%%%%%%%
% table of free parameters
%%%%%%%%%%%%%%%%%%%%%%%%%%%%%%%%%%%%%%%%%%
\begin{table}
    \caption{Free parameters of the Empirical Dust Attenuation Model}
    \begin{center}
        \begin{tabular}{ccc} \toprule
            Parameter & Definition & prior\\[3pt] \hline\hline
            %\multicolumn{3}{c}{DEM with slab model}\\ \hline
            $\mtaum$ & $M_*$ dependence of the optical depth, $\tau_V$ & flat $[-5., 5.]$\\
            $\mtaus$ & $\ssfr$ dependence of $\tau_V$  & flat $[-5., 5.]$\\
            $c_{\tau}$ & amplitude of $\tau_V$ & flat $[0., 6.]$\\
            %\hline
            %\multicolumn{3}{c}{DEM with $\mathcal{N}_T$ model}\\ \hline
            %$m_{\mu,1}$ & Slope of the $\log M_*$ dependence of optical depth,
            %$\tau_V$ & flat $[-5., 5.]$\\
            %$m_{\mu,2}$ & Slope of the $\log {\rm SFR}$ dependence of optical
            %depth, $\tau_V$ & flat $[-5., 5.]$\\
            %$c_{\mu}$ & amplitude of the optical depth, $\tau_V$ & flat $[0., 6.]$\\ 
            %$m_{\sigma,1}$ & Slope of the $\log M_*$ dependence of optical depth, $\tau_V$ & flat $[-5., 5.]$\\
            %$m_{\sigma,2}$ & Slope of the $\log {\rm SFR}$ dependence of optical depth, $\tau_V$ & flat $[-5., 5.]$\\
            %$c_{\sigma}$ & amplitude of the optical depth, $\tau_V$ & flat $[0.1, 3.]$\\ 
            %\hline
            $\mdeltam$ & $M_*$ dependence of $\delta$, the attenuation curve slope offset & flat $[-4., 4.]$\\
            $\mdeltas$ & $\ssfr$ dependence of $\delta$ & flat $[-4., 4.]$\\
            $c_{\delta}$ & amplitude of $\delta$ & flat $[-4., 4.]$\\
            %$f_{\rm neb}$ & nebular attenuation fraction & flat $[1., 4.]$\\
            \hline
        \end{tabular} \label{tab:free_param}
    \end{center}
\end{table}
%%%%%%%%%%%%%%%%%%%%%%%%%%%%%%%%%%%%%%%%%%

