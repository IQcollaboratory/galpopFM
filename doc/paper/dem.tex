\section{The Empirical Dust Attenuation Framework} \label{sec:dem}
\tks{how do you feel now about alternative names? 
Rachel proposed EDAM; in our discussion of names I think the best I came with was DEAP 
(Dust Empirical Attenuation Prescription); looking at this section title we could also go for FADE which I really like in the context 
(Framework to Attenuate by Dust Empirically).}
\chedit{
    In this section, we describe the Empirical Dust Attenuation (\eda)
    framework and present one particular \eda~prescription. The \eda~is a
    flexible framework for applying dust attenuation curves to simulated galaxy
    populations. For each simulated galaxy, the \eda~assigns a dust attenuation
    curve assigned that is parameterized as a function of the galaxy's
    properties (\eg~$M_*$, ${\rm SSFR}$), the \eda~parameters, and randomly
    sampled inclincation. With the \eda, we can apply a wide variety of dust
    attenuation that include correlation between dust attenuation and physical
    galaxy properties. 
}
% Later, we demonstrate that we can accurately reproduce SDSS observations with the \eda~and use it to test galaxy formation models and shed light on dust in galaxies. 

We begin by defining the dust attenuation curve, $A(\lambda)$, as 
\begin{equation} \label{eq:full_atten}
    F_o (\lambda) = F_i (\lambda) 10^{-0.4 A(\lambda)}
\end{equation}
where $F_o$ is the observed flux and $F_i$ is the intrinsic flux. We normalize
the attenuation to the $V$ band attenuation, 
\begin{equation} 
    A(\lambda) = A_V \frac{k(\lambda)}{k_V}
\end{equation}
so that $A_V$ determines the amplitude of the attenuation, while $k(\lambda)$
determines the wavelength dependence. 

\chedit{
    The \eda~framework assigns $A(\lambda)$ to every galaxy in the simulations
    using some flexible prescription. For the \eda~prescription in this work,
    we assign $A_V$ for each galaxy using the slab model, where $A_V$ is a
    function of galaxy inclination, $i$, and its optical depth,
    $\tau_V$~\citep[\eg][]{somerville1999, somerville2012}: 
}
\begin{equation} \label{eq:slab}
    A_V = -2.5 \log \left[ \frac{1 - e^{-\tau_V\,\sec i}}{\tau_V\,\sec i} \right].
\end{equation}
\chedit{
    To include correlation between $A_V$ and the galaxy's properties ($M_*$ and
    $\ssfr$), we parameterize $\tau_V$ using a linear $M_*$ and $\ssfr$ dependence:
}
\begin{equation} \label{eq:tauv}
    \tau_V(M_*, \sfr) = \mtaum \log \left(\frac{M_*}{10^{10} M_\odot}\right) +
    \mtaus \log \left(\frac{\ssfr}{10^{-10}yr^{-1}}\right) + c_\tau.
\end{equation}
$\mtaum$, $\mtaus$, and $c_\tau$ represent the $M_*$ dependence, the $\ssfr$
dependence, and amplitude of $\tau_V$. Since $\tau_V$ is optical depth, we
impose a $\tau_V \ge 0$ limit.
\chedit{
    For each galaxy, we uniformly sample $\cos i$ from 0 to 1. By sampling
    $\cos i$, our \eda~prescription includes significant variance in
    $A(\lambda)$. Galaxies with the same galaxy properties do not have the same
    dust attenuation.
}

\chedit{
    We use the slab model primarily as a flexible prescription for $A_V$ that
    depend on a randomly sampled $i$, with \emph{loose} physical motivations.
    The slab model is a naive approximation. $A_V$, in reality, depends on
    properties such as the detailed star-to-dust geometry or variations in the
    extinction curves. However, the purpose ~\eda~is to assign an accurate
    distribution of dust attenuation curves for the galaxy population --- 
    \emph{not} to accurately model dust attenuation for individual galaxies.
    In Appendix~\ref{sec:slab}, we present how our slab model based \eda~with
    randomly sampled $i$, produces a $A_V$ distribution, $p(A_V)$ that 
    matches $p(A_V)$ of the \cite{salim2018} GSWLC2 catalog. 
}

\chedit{
    All galaxies in the simulations are assigned $A_V$ from the slab model.
    For star-forming galaxies, which typically have disc-like morphologies, the
    slab model produces $A_V$ that is correlated with $i$ in a way consistent
    with the literature: edge-on galaxies have higher $A_V$ than face-on
    galaxies~\citep[\eg][]{conroy2010, wild2011, battisti2017, salim2020}. 
    Quiescent galaxies, however, typically have ellipitcal morphologies. In
    this case, the slab model is an \emph{empirical} prescription for statistically 
    sampling $A_V$. In Appendix~\ref{sec:slab}, we demonstrate that slab model 
    can match $p(A_V)$ of quiescent galaxies ($\ssfr < 10^{-11}yr^{-1}$) 
    in GSWLC2. We, therefore, conclude that the slab model is a sufficiently 
    flexible prescription for sampling $A_V$ for all galaxies. 
}


For the wavelength dependence of the attenuation curve, $k(\lambda)$, we
use the \cite{noll2009} parameterization: 
\begin{equation} \label{eq:noll}
    k(\lambda) = \left(k_{\rm Cal}(\lambda) + D(\lambda)\right) \left(
    \frac{\lambda}{\lambda_V} \right)^\delta.
\end{equation}
Here $k_{\rm Cal}(\lambda)$ is the \cite{calzetti2001} curve: 
\[
    k_{\rm Cal}(\lambda) = 
    \begin{cases} 
        2.659 (-1.857 + 1.040/\lambda) + R_V, & 6300 A \le \lambda \le
        22000 A \\ 
        2.659 (-2.156 + 1.509/\lambda - 0.198/\lambda^2 + 0.011/\lambda^3) +
        R_V & 1200 A \le \lambda \le 6300 A
    \end{cases}
\]
where $\lambda_V = 5500 A$ is the $V$ band wavelength and $\delta$ is the slope
offset of the attenuation curve from $k_{\rm Cal}$. Since $\delta$ correlates 
with galaxy properties~\citep[\eg][]{leja2017, salim2018},
we parameterize $\delta$ with a similar $M_*$ and $\ssfr$ dependence as
$\tau_V$:  
\begin{align} \label{eq:delta}
    \delta(M_*, \sfr) &= \mdeltam \log \left(\frac{M_*}{10^{10}
    M_\odot}\right) + \mdeltas \log \left(\frac{\ssfr}{10^{-10}yr^{-1}}\right)
    + c_\delta.
\end{align}
% Although a number of works have found correlation between the attenuation
% curve slope and inclination~\citep{wild2011, chevallard2013, battisti2017b},
% \cite{salim2020}, most recently, found that the driver of this trend is the
% relationship between $A_V$ and slope. We therefore do not include an
% inclination dependence in $\delta$. 
$D(\lambda)$ in Eq.~\ref{eq:noll} is the UV dust bump, which we parameter using
the standard Lorentzian-like Drude profile:
\begin{equation}
    D(\lambda) = \frac{E_b(\lambda~\Delta \lambda)^2}{(\lambda^2 -
    \lambda_0^2)^2 + (\lambda~\Delta \lambda)^2}
\end{equation}
where $\lambda_0$, $\Delta \lambda$, and $E_b$ are the central wavelength,
full width at half maximum, and strength of the bump, respectively. We assume
fixed $\lambda_0 = 2175
A$ and $\Delta \lambda = 350A$. \cite{kriek2013} and \cite{tress2018} find
that $E_b$ correlates with the $\delta$ for star-forming galaxies $z\sim2$.
\cite{narayanan2018} confirmed this dependence in simulations. 
\chedit{
    For our \eda~prescription, we include the UV dust bump since we use UV
    colors as one of our observables.
}
However, we assume a fixed relation between $E_B$ and $\delta$ from
\cite{kriek2013}: $E_b = -1.9~\delta + 0.85$. Allowing the slope and amplitude
of the $E_B$ and $\delta$ relation to vary does {\em not} impact our results;
however, we do not derive any meaningful constraints on them either. In
Table~\ref{tab:free_param}, we list and describe all of the free parameters in
the \eda. 

% motivation for the M* and SSFR dependence 
\todo{@tks: motivation for $M_*$ and $\ssfr$ dependence}
\tksedit{


\begin{figure}
\begin{center}
    \includegraphics[width=\textwidth]{figs/gswlc_Av_mstar_ssfr_dependence.pdf}
    \caption{\label{fig:dep}
    Dependence of the V-band attenuation on stellar mass (left, for $-11 <$ log(sSFR [yr$^{-1}$]) $< -10.5$ (purple),   
    $-10.5 <$ log(sSFR [yr$^{-1}$]) $< -10$ (red), and  $-10 <$ log(sSFR [yr$^{-1}$]) (orange)),
    and sSFR (right, for  $9.5 <$ log($M_* [M_{\odot}]) < 10.5$ (blue) and $10.5 <$ log($M_* [M_{\odot}]) < 11.5$ (green)) for the GALEX-SDSS-WISE Legacy Catalog (GSWLC)\footnote{\url{https://salims.pages.iu.edu/gswlc/}} \citep{salim2016}. 
    At lower stellar masses the $A_V$ and $M_*$ are more strongly correlated than $A_V$ and sSFR, while the opposite is true at higher stellar masses.
    The correlation between $A_V$ and sSFR is most noticeable at intermediate sSFR (${\sim}-11 <$ log(sSFR [yr$^{-1}$]) $< {\sim}-10$), 
    where there is no apparent correlation with stellar mass. At higher sSFR (e.g. for galaxies on the SFS) the correlation with stellar mass is stronger.
    \tks{not sure we want to include this figure because it is confusing that it uses different data. We could also move it to another appendix if we want?}
    }
\end{center}
\end{figure}
}
%In $\tau_V$ we include the correlation between $A_V$ and the galaxy's properties , found in both observations and simulations~\citep[\eg][]{narayanan2018, salim2020}. 


$\ssfr$ of galaxies are used to calculate $\tau_V$ and $\delta$ in
Eqs.~\ref{eq:tauv} and~\ref{eq:delta}. However, due to mass and temporal resolutions,
some galaxies in the simulations have $\sfr=0$ --- \ie~an unmeasurably low
SFR~\citep{hahn2019c}. They account for 17, 19, 9\% of galaxies
in SIMBA, TNG, and EAGLE, respectively. Since Eqs.~\ref{eq:tauv}
and~\ref{eq:delta} depend on $\log\ssfr$, they cannot be used in the equations
to derive $\tau_V$ and $\delta$ for these galaxies. To account for this issue,
we assign $\sfr_{\rm min}$, the minimum non-zero $\sfr$ in the simulations, to
$\sfr=0$ galaxies when calculating $\tau_V$ and $\delta$. For SIMBA, TNG, and
EAGLE, $\sfr_{\rm min}=0.000816$, $0.000268$, and $0.000707 M_\odot/yr$. Although 
this assumes that $\sfr=0$ galaxies have similar dust properties as the galaxies 
with $\sfr = \sfr_{\rm min}$, since the simulations have very low $\sfr_{\rm min}$ 
we expect galaxies with $\sfr = \sfr_{\rm min}$ to have little recent
star-formation and low gas mass, similar to $\sfr=0$ galaxies. 

%Since $\sfr=0$ galaxies do not account for a large fraction of our simulated galaxies, we directly sample their observables ($G, R, NUV$, and $FUV$) from the distribution of observables for SDSS quiescent galaxies. This way, we ensure that the attenuation of $\sfr=0$ galaxies does not impact the rest of the \eda~parameters. In Appendix~\ref{sec:res}, we discuss the resolution effects in more detail and demonstrate that our results are \emph{not} impacted by other prescriptions for attenuating $\sfr=0$ galaxies.

\chedit{
    In practice, to apply the \eda~to a simulated galaxy population, we first
    assign a randomly sampled inclincation, $i$, to each galaxy ($\cos i$
    uniformly sampled from 0 to 1).  $\tau_V$ and $\delta$ are calculated for
    the galaxy based on its $M_*$,
    $\ssfr$ and the \eda~parameters. We then determine $A_V$ from $i$ and
    $\tau_V$ and $k(\lambda)$ from $\delta$, which combined gives us
    $A(\lambda)$ for each galaxy.
} 
Afterwards, we attenuate the galaxy SEDs using Eq.~\ref{eq:full_atten} and use
the attenuated SEDs to calculate the observables: $g, r, NUV$, and $FUV$
absolute magnitudes. In Figure~\ref{fig:dem_av}, we present attenuation
curves,
$A(\lambda)$, generated by the \eda~for galaxies with different $\sfr$ and $M_*$: 
star-forming ($\sfr=10^{0.5}M_\odot/yr$) with low mass ($10^{10}M_\odot$;
blue), with high mass ($10^{11}M_\odot$; green) and quiescent
($\sfr=10^{-2}M_\odot/yr$) with low mass ($10^{10}M_\odot$; orange), with high
mass ($10^{11}M_\odot$; red). All galaxies are edge-on (\ie~$i=0$) and we use
\eda~parameters: $\{\mtaum, \mtaus, c_\tau, \mdeltam, \mdeltas, c_\delta\} =
\{2., -2., 2., -0.1, -0.1, -0.2\}$, arbitrarily chosen within the prior range
listed in Table~\ref{tab:free_param}. For comparison, we include the
\cite{calzetti2001} attenuation curve. Even for only edge-on galaxies, the
\eda~produces attenuation curves with a wide range of amplitude and
slope to galaxies based on their physical properties. 

\begin{figure}
\begin{center}
    \includegraphics[width=0.6\textwidth]{figs/dems.pdf}
    \caption{\label{fig:dem_av}
    \chedit{
        Attenuation curves, $A(\lambda)$, assigned by our Empirical Dust
        Attenuation (\eda) prescription to edge-on galaxies with different $\sfr$ and
        $M_*$ for an arbitrary set of \eda~parameters. We include the
        \eda~$A(\lambda)$ for star-forming galaxies ($\sfr=10^{0.5}M_\odot/yr$)
        with $M_* = 10^{10}M_\odot$ (blue) and $10^{11}M_\odot$ (green) and
        quiescent galaxies ($\sfr=10^{-2}M_\odot/yr$) with $M_* =
        10^{10}M_\odot$ (orange) and $10^{11}M_\odot$ (red). We set $i=0$ for
        all the galaxies in the figure for simplicity but in practice the
        \eda~uniformly samples $\cos i$ from 0 to 1 for each galaxy.
        For comparison, we include the \cite{calzetti2001} attenuation curve.
        {\em The \eda~provides a flexible prescription for assigning dust
        attenuation to galaxies based on their inclination, physical properties
        ($M_*$ and $\ssfr$), and the \eda~parameters.}
    }
    } 
\end{center}
\end{figure}


%%%%%%%%%%%%%%%%%%%%%%%%%%%%%%%%%%%%%%%%%%
% table of free parameters
%%%%%%%%%%%%%%%%%%%%%%%%%%%%%%%%%%%%%%%%%%
\begin{table}
    \caption{Free parameters of the Empirical Dust Attenuation Model}
    \begin{center}
        \begin{tabular}{ccc} \toprule
            Parameter & Definition & prior\\[3pt] \hline\hline
            %\multicolumn{3}{c}{DEM with slab model}\\ \hline
            $\mtaum$ & $M_*$ dependence of the optical depth, $\tau_V$ & flat $[-5., 5.]$\\
            $\mtaus$ & $\ssfr$ dependence of $\tau_V$  & flat $[-5., 5.]$\\
            $c_{\tau}$ & amplitude of $\tau_V$ & flat $[0., 6.]$\\
            %\hline
            %\multicolumn{3}{c}{DEM with $\mathcal{N}_T$ model}\\ \hline
            %$m_{\mu,1}$ & Slope of the $\log M_*$ dependence of optical depth,
            %$\tau_V$ & flat $[-5., 5.]$\\
            %$m_{\mu,2}$ & Slope of the $\log {\rm SFR}$ dependence of optical
            %depth, $\tau_V$ & flat $[-5., 5.]$\\
            %$c_{\mu}$ & amplitude of the optical depth, $\tau_V$ & flat $[0., 6.]$\\ 
            %$m_{\sigma,1}$ & Slope of the $\log M_*$ dependence of optical depth, $\tau_V$ & flat $[-5., 5.]$\\
            %$m_{\sigma,2}$ & Slope of the $\log {\rm SFR}$ dependence of optical depth, $\tau_V$ & flat $[-5., 5.]$\\
            %$c_{\sigma}$ & amplitude of the optical depth, $\tau_V$ & flat $[0.1, 3.]$\\ 
            %\hline
            $\mdeltam$ & $M_*$ dependence of $\delta$, the attenuation curve slope offset & flat $[-4., 4.]$\\
            $\mdeltas$ & $\ssfr$ dependence of $\delta$ & flat $[-4., 4.]$\\
            $c_{\delta}$ & amplitude of $\delta$ & flat $[-4., 4.]$\\
            %$f_{\rm neb}$ & nebular attenuation fraction & flat $[1., 4.]$\\
            \hline
        \end{tabular} \label{tab:free_param}
    \end{center}
\end{table}
%%%%%%%%%%%%%%%%%%%%%%%%%%%%%%%%%%%%%%%%%%

