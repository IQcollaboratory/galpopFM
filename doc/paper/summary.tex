\section{Summary}
In this work, we present the \eda, a framework for statistically applying dust
attenuation to simulated galaxy populations. It uses a parameterization of 
the attenuation curves motivated from observations~\citep{noll2009} and a
flexible method for sampling the attenuation curve parameters that includes
correlations with galaxy properties ($M_*$ and SSFR). We apply the \eda~to 
three state-of-the-art hydrodynamical simulations (SIMBA, TNG, and EAGLE) and
forward model the optical and UV color-magnitude relations. We then compare
the forward modeled simulations to a $M_r < -20$ complete SDSS galaxy sample  
using simulation-based inference. Based on this comparison, we find the
following results: 

\begin{itemize}
    \item Dust attenuation is essential for simulations to reproduce observations.
        Without dust attenuation, all of the hydrodynamical simulations struggle
        to reproduce the observed UV and optical color-magnitude relation. 
    \item With the \eda, the TNG and EAGLE simulations are able to produce UV and
        optical color-magnitude relations in good agreement with SDSS observations. 
        SIMBA, however, overpredicts a substantial starburst galaxy population.
        In order to reproduce observations, these starburst galaxies would require
        both high attenuation and reddening, which goes against the observed 
        attenuation-slope relation. 
    \item The attenuation curves predicted by the bestfit \eda~for TNG and
        EAGLE are in excellent agreement with the observed attenuation-slope
        relation. They also closely reproduce the observed attenuation curves
        of star-forming galaxies. The success of the \eda~in reproducing these
        observations, which were not included in the comparison, highlights the 
        advantages of a forward modeling approach. 
    \item Lastly, the \eda~sheds light on dust attenuation in quiescent
        galaxies, which remains poorly understood due to observations
        challenges. From the \eda, we find that quiescent galaixes have
        significant UV and optical attenuation with shallow attenuation curves.
        Moreover, since the \eda~predict dust attenuation for all galaxies, it
        reveals the connection between galaxy properties and
        dust attenuation. More massive galaxies have higher overall dust
        attenuation while galaxies with higher SSFR have steeper attenuation
        curves. 
\end{itemize}

Our results clearly demonstrate that the \eda~and a forward modeling approach
can provide key insights into dust attenuation. For those uninterested in dust,
the \eda~also provides a framework for marginalizing over dust when comparing
simulations to observations. In the case of SIMBA, we found that dust
attenuation is insufficient to accurately reproduce observations due to its 
starburst population. For TNG and EAGLE, however, dust attenuation is highly
degenerate with the differences they have in thier galaxy physics
prescriptions. Even though TNG and EAGLE predict galaxy populations with
significantly different physical properties, there is enough uncertainty in our
understanding of dust that by adjusting attenuation alone both TNG and EAGLE
can reproduce the same SDSS observations. This also suggests that any
comparisons across simulations must marginalize over dust attenuation or run 
the risk of overinterpretation --- differences between simulations may not
produce any change in the observables once dust attenuation is taken into account. 
Therefore, our current understanding of dust, or lack of understanding, limit
our ability to distinguish between the various hydrodynamical models and is a
major bottleneck for investigating galaxy formation using simulations.

The forward modeling approach we present offers many avenues for improving on
our understanding of dust. In this paper, we used a relatively restrictive 
$M_r$ complete SDSS galaxy sample. Comparison to a larger observed galaxy
sample will place tighter constraints on \eda~parameters and enable better
differentiation between the simulations. One way to expand the observed galaxy
sample would be to remove the $M_r$ completeness limit by including the
selection function to our forward model. Upcoming surveys, such as the DESI
Bright Galaxy Survey and the PFS Galaxy Evolution Survey, will also soon
provide much larger galaxy samples. Furthermore, IR observations, which measure
dust emission and trace dust attenuation, also have the potential to tightly 
constrain the \eda~parameters and therefore break degeneracies between dust
and the galaxy physics in simulations. In future works, we will apply the
\eda~and a forward mdoeling approach to more statistically powerful samples and
include IR observables in order to tightly constrain and reveal new insights 
into dust attenuation. 
