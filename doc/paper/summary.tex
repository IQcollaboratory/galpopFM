\section{Summary}
In this work, we present the \eda, a framework for applying dust attenuation to
simulated galaxy populations. It uses a parameterization of 
the attenuation curves motivated from observations~\citep{noll2009} and
assigns attenuation curves to simulated galaxies based on their physical properties ($M_*$ and SSFR).
We apply the \eda~to 
three state-of-the-art hydrodynamical simulations (SIMBA, TNG, and EAGLE) and
forward model the optical and UV color-magnitude relations. We then compare
the forward modeled simulations to a $M_r < -20$ complete SDSS galaxy sample  
using likelihood-free inference. Based on this comparison, we find the
following results: 

\begin{itemize}
    \item Dust attenuation is essential for our hydrodynamical simulations to
        reproduce observations.
        \chedit{
            Without dust, SIMBA, TNG, and EAGLE all struggle to reproduce the
            observed UV and optical color-magnitude relation. 
            They predict galaxies that are intrinsically much bluer and more
            luminous in the optical and with a broader distriubtion of $\fnuv$
            colors than SDSS. 
        }
    \item With the \eda, each of the simulations are able to produce UV and
        optical color-magnitude relations in good agreement with SDSS
        observations. 
        \chedit{
            However, we find that star formation quenching is too efficient for
            the most massive quiescent galaxies in SIMBA and TNG. 
            When forward modeled, these galaxies appear as luminous UV red
            galaxies outside the observed UV color-magnitude relation and they
            cannot be reconciled by dust. 
        }
    \item The attenuation curves of star-forming galaxies predicted by the
        \eda~forthe simulations are in good agreement with the observed
        attenuation-slope relation. 
        They also closely reproduce the observed attenuation curves
        of star-forming galaxies. 
        \chedit{
            The simulations, however, predict star-forming galaxies that are
            intrinsically too luminous compared to observations, so the
            \eda~does not find star-forming galaxies with $A_V < 0.3$, found in
            the literature. 
        }
    \item \chedit{
            Lastly, with the \eda~we predict the attuenation curves of
            quiescent galaxies, which observations struggle to directly
            measure.
            We find that quiescent galaxies have $A(\lambda)$ curves
            with lower amplitudes and shallower slopes than star-forming
            galaxies. 
            Over the entire population, we find that more massive galaxies have
            higher overall dust attenuation while galaxies with higher SSFR
            have steeper attenuation curves. 
        }
\end{itemize}

Our results clearly demonstrate that the \eda~and a forward modeling approach
provides key insights into dust attenuation. For those uninterested in dust,
the \eda~also provides a computationally feasible framework for marginalizing
over dust when comparing simulations to observations. 
%In the case of SIMBA, we found that the \eda~dust attenuation is insufficient to accurately reproduce observations due to its excess starburst population. 
\chedit{
    However, we find that dust attenuation is highly degenerate with
    differences in their galaxy physics prescriptions.
    Even though the simulations predict galaxy populations with significantly
    different physical properties, there is enough uncertainty in our
    understanding of dust that by adjusting attenuation alone they can all
    reproduce the same SDSS observations.
}
This also suggests that any comparisons across simulations must marginalize
over dust attenuation or run the risk of overinterpretation. 
Therefore, our current understanding of dust, or lack of, limit our ability to
distinguish between the various hydrodynamical models and is a major bottleneck
for investigating galaxy formation using simulations.

The forward modeling approach we present offers many avenues for improving on
our understanding of dust. In this paper, we used a restrictive 
$M_r$ complete SDSS galaxy sample. Comparison to a larger observed galaxy
sample will place tighter constraints on \eda~parameters and enable better
differentiation between the simulations. One way to expand the observed galaxy
sample would be to remove the $M_r$ completeness limit by including the
selection function to our forward model. Upcoming surveys, such as the DESI
Bright Galaxy Survey and the PFS Galaxy Evolution Survey, will also soon
provide much larger observational galaxy samples. Furthermore, IR observations,
which measure dust emission and trace dust attenuation, also have the potential
to tightly constrain the \eda~parameters and therefore break degeneracies
between dust and the galaxy physics in simulations. 
In the next paper of the series, we will use the forward modeling approach
with the \eda~to investigate star formation quenching in galaxy formation
models. 
In other future works, we will apply the \eda~and a forward modeling approach
to more statistically powerful samples and include IR observables in order to
tightly constrain and reveal new insights into dust attenuation. 
