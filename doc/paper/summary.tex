\section{Summary}
In this work, we present the DEM, an empirical framework for including dust
attenuation in simulated galaxy populations. It uses a parameterization of 
the attenuation curves motivated from observations~\citep[][]{noll2009} and a
flexible method for sampling the attenuation curve parameters that includes
correlations with galaxy properties ($M_*$ and $\sfr$). We apply the DEM to 
three state-of-the-art hydrodynamical simulations (SIMBA, TNG, and EAGLE) and
forward model the optical and UV color-magnitude relations. Afterwards, we
compare these forward modeled simulations to observed central galaxies in SDSS
using simulation-based inference. Based ont his comparison, we find the
following results: 

\begin{itemize}
\item Dust attenuation is essential for simulations to reproduce observations.
With the DEM, we are able to reproduce SDSS observations for all three hydrodynamical 
simulations. However, SIMBA requires an extreme dust attenuation that reverses 
the established relationship between color and $\sfr$ in order to account for
SIMBA overpredicting starburst galaxies at $<10^{10}M_\odot$.
\item The DEM attenuation curves are in good agreement with major observational 
constraints. They closely reproduce the observed attenuation--slope relation,
better than radiative transfer models. We also find significant variation in 
the attenuation curves as in observations. For star-forming galaxies, the DEM
atteunation curves are in good agreement with the literature.  
\item Focusing on the DEM for TNG and EAGLE, we find significant $M_*$ and
$\sfr$ dependences in $A_V$. More massive galaxies have higher dust
attenuation; galaxies with lower $\sfr$ have  higher dust attenuation. 
\item Lastly, the DEM is able to constrain the attenuation curves of quiescent
galaxies, which are poorly constrained by observations. We find that quiescent
galaxies have shallower attenuation curves with higher $A_V$ and larger
varation than star-forming galaxies. 
\end{itemize}

Our results clearly demonstrate that the DEM can be used to provide insight
into dust attenuation. For those uninterested in dust, the DEM also provides 
a straightforward framework for marginalizing over dust. Although accounting
for dust is necessy to reproduce observations, our limited understanding of
dust attenatuion allows simulations that predict galaxy populations with
significantly different physical properties to reproduce the same observable. Since varying attenuation alone
can entirely reproduce observations, dust is highly degenerate with differences
in subgrid physics across simulations. After marginalizing over dust, observations do not have the
constraining power to differentiate between the various hydrodyanmical models.
Hence, detailed comparisons across simulations and to observations likely 
overinterpret the differences and similarities found in simulations. Therefore, 
we demonstrate that the current limitations in our understanding of dust is a 
major bottleneck for investigating galaxy formation using simulations.

Even with the limited statistical power of our $M_r$ complete SDSS central
galaxy sample, we derived tight constarints on dust attenuation. Upcoming
surveys such as the Bright Galaxy Survey of the Dark Energy Spectroscopic
Instrument~\citep[DESI;][\ch{Ruiz\etal2020}]{desicollaboration2016}, the Galaxy
Evolution Survey of the Prime Focus
Spectrograph~\citep[PFS;][]{takada2014,tamura2016}, and the Wide-Area VISTA
Extragalactic Survey~\citep[WAVES;][]{driver2016, driver2019} will provide
much more statistically powerful observations and at higher redshifts. With 
these observations, the DEM will be able to more tightly constrain and reveal 
new insights into dust attenuation. 



