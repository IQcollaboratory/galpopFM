\begin{figure}
\begin{center}
    \includegraphics[width=0.9\textwidth]{figs/abc_av_mssfr.pdf}
    \caption{\label{fig:avmsfr}
    \chedit{ 
        $M_*$ and $\ssfr$ dependence of dust attenuation at $1500 \AA$
        ($A_{1500}$; top) and at $5500\AA$ ($A_{V}$ bottom) predicted by the
        \eda~for TNG (left) and EAGLE (right). The colormap in each hexbin 
        represents the median attenuation for all simulated galaxies in the
        bin (right color bar). We only include bins with more than 10 galaxies that
        satisfy our $M_r < -20$ completeness limit.
        For reference, we include in each panel the $M_*-\ssfr$ relation of
        all galaxies from the simulations (black dashed).
        Overall, TNG and EAGLE galaxies with higher $M_*$ have higher dust
        attenuation --- consistent with the literature.
        Furthermore, since previous works have primarily focused on star-forming
        galaxies, the \eda~provides new insight into the $\ssfr$ dependence of
        dust attenuation: simulated galaxies with higher $\ssfr$ have steeper
        attenuation curves. 
    }
    }
\end{center}
\end{figure}

\subsection{The Galaxy -- Dust Connection}  
\chedit{
    In the previous section, we presented the attenuation curves predicted by
    the \eda~for quiescent galaxies. 
    By comparing them to the attenuation curves of star-forming galaxies, we
    found significant $\ssfr$ dependence in dust attenuation: quiescent galaxies
    have attenuation curves with shallower slopes and lower amplitude than
    star-forming galaxies. 
    With the $M_*$ and $\ssfr$ dependent parameterization of our
    \eda~prescription (Eqs~\ref{eq:tauv} and~\ref{eq:delta}), we can also shed
    light on the connection between the physical properties of the simulated
    galaxies and dust attenuations through the \eda~parameter constraints. 
    In Table~\ref{tab:posterior}, we list the median values and the 68\%
    confidence interval of the inferred \eda~parameter posteriors for the 
    three simulations. 
    In addition to the $\ssfr$ dependence from the last section, we also find
    significant $M_*$ dependence in $\tau_V$: $V$-band dust attenuation is
    higher for more massive galaxies.  
    There is, however, little $M_*$ dependence in the slope of the dust
    attenuation.
}

%%%%%%%%%%%%%%%%%%%%%%%%%%%%%%%%%%%%%%%%%%
% table of free parameters
%%%%%%%%%%%%%%%%%%%%%%%%%%%%%%%%%%%%%%%%%%
\begin{table}
    \caption{Inferred the Empirical Dust Attenuation Model Parameters}
    \begin{tabular}{lcccccc} \toprule
        & $m_{\tau,M_*}$ & $m_{\tau,\ssfr}$ & $c_\tau$ & $m_{\delta,M_*}$ & $m_{\delta,\ssfr}$ & $c_\delta$ \\[3pt] \hline\hline
        SIMBA   & $1.27\substack{+0.46\\-0.46}$ &
        $1.28\substack{+0.24\\-0.23}$ & $1.58\substack{+0.12\\-0.12}$ &
        $0.07 \substack{+0.12\\-0.11}$ & $0.13 \substack{+0.10\\-0.10}$ &
        $-0.18\substack{+0.04\\-0.04}$ \\
        TNG     & $0.57\substack{+0.44\\-0.53}$ &
        $0.62\substack{+0.21\\-0.20}$ & $1.34\substack{+0.19\\-0.21}$ &
        $-0.18\substack{+0.20\\-0.19}$ & $-0.19\substack{+0.15\\-0.16}$ &
        $-0.07\substack{+0.08\\-0.08}$ \\
        EAGLE   & $0.59\substack{+0.33\\-0.33}$ &
        $0.18\substack{+0.20\\-0.17}$ & $0.81\substack{+0.14\\-0.15}$ &
        $-0.13\substack{+0.17\\-0.18}$ & $-0.22\substack{+0.14\\-0.14}$ &
        $-0.34\substack{+0.08\\-0.08}$\\
        \hline
    \end{tabular} \label{tab:posterior}
\end{table}

We take a closer look at the $M_*$ and $\ssfr$ dependence of the attenuation
curve in Figure~\ref{fig:avmsfr}. 
We present dust attenuation at $1500\AA$ ($A_{1500}$; top) and $5500\AA$
($A_V$; bottom) as a function of $\log M_*$ and $\log \ssfr$ predicted by the
\eda~for SIMBA (left), TNG (center) and EAGLE (right). 
For each hexbin, the colormap represents the median attenuation for all
simulated galaxies in the bin. 
We only include bins with more than 10 galaxies that satisfy our selection
function (Section~\ref{sec:obs}). 
We include, for reference, the $M_* - \ssfr$ relation of the simulations in
black dashed contours.

In each panel, we find that SIMBA, TNG, and EAGLE galaxies with higher
$M_*$ have higher dust attenuation --- consistent with the literature.
\cite{burgarella2005}, for instance, found significant positive $M_*$
dependence in $FUV$ attenuation in NUV-selected and FIR-selected samples. 
\cite{garn2010} and \cite{battisti2016} also find higher attenuation in
more massive SDSS star-forming galaxies. 
Most recently, \cite{salim2018} find higher $V$ and $FUV$ attenuation for
more massive star-forming galaxies in GSWLC2. 
\chedit{ 
    For the $\ssfr$ dependence, we find that galaxies with higher $\ssfr$ have
    higher $A_{1500}$ (top) and $A_V$ (bottom) as well as steeper slopes. 
    We note that the $\ssfr$ dependence is not as prominent in EAGLE (see also
    Table~\ref{tab:posterior}). 
    Compared to SIMBA or TNG, EAGLE has a narrower $\ssfr$ distribution with no
    starburst galaxies or quiescent galaxies with $\ssfr < 10^{-12}yr^{-1}$. 
    As a result, EAGLE has fewer intrinsically luminous star-forming galaxies
    or UV red galaxies (Figure~\ref{fig:obs}). 
    This means that EAGLE has a narrower intrinsic $\gr$ and $\fnuv$ color
    distributions that require an overall attenuation and reddenning less
    dependent on $\ssfr$. 
    Nevertheless, in all simulations, star-forming galaxies have slopes that
    are consistent with observations (Section~\ref{sec:reproduce}) while
    quiescent galaxies with the lowest $\ssfr$ have nearly flat attenuation
    curves. 
    Since observations have only focused on star-forming galaxies due to the
    difficulty in measuring dust attenuation in quiescent galaxies, the
    \eda~predictions provide new insight into the $\ssfr$ dependence of dust
    attenuation. 
    In summary,we find that \emph{SIMBA, TNG, and EAGLE galaxies with higher
    $M_*$ require overall higher dust attenuation and galaxies with higher
    $\ssfr$ require steeper attenuation curves}.
} 

