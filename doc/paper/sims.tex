\begin{figure}
\begin{center}
    \includegraphics[width=\textwidth]{figs/smf_m_sfr.pdf}
    \caption{\label{fig:smf_msfr}
    The stellar mass functions, $\Phi_{M_*}$ (left-most panel), and $M_*-\sfr$
    relation (right panels) of galaxies in three cosmological hydrodynamic 
    simulations: SIMBA (orange), TNG (blue), and EAGLE (green). 
    For reference, we include $\Phi_{M_*}$ (black) and the $M_*-\sfr$ relation 
    (black dashed) of our observational sample from SDSS. Uncertainties for the 
    SDSS $\Phi_{M_*}$ are derived using jackknife resampling. We describe the
    simulations and observations above in Section~\ref{sec:sims}. \emph{Differences
    in $\Phi_{M_*}$ and the $M_*-\sfr$ relations among the hydrodynamic simulations 
    highlight how they predict galaxy populations with significantly different
    the physical properties.} 
    }
\end{center}
\end{figure}

\section{Data}\label{sec:sims}
In this paper, we present the Empirical Dust Attenuation (\eda) model and
demonstrate how it can be used in a forward modeling approach to compare galaxy
populations in simulations and observations. For our simulations, we use three
cosmological hydrodynamical simulations: the Illustris TNG (hereafter TNG),
EAGLE, and SIMBA. For our observations, we use a galaxy sample derived from
SDSS. Below, we briefly describe the hydrodynamical simulations and the SDSS
observations used throughout this work.

In Figure~\ref{fig:smf_msfr}, we present the stellar mass functions,
$\Phi_{M_*}$ left-most panel, and $M_*-\sfr$ relations (right panels) of
galaxies in the  SIMBA (orange), TNG (blue), and EAGLE (green) cosmological
hydrodynamic simulations.  For reference, we include $\Phi_{M_*}$ and the 
$M_*-\sfr$ relation for our observational SDSS galaxy sample. For the
simulations, $M_*$ is the total stellar mass within the host halo, excluding
any stellar mass in subhalos; $\sfr$ is the instantaneous $\sfr$ derived from
dense and cold gas. For SDSS, $M_*$ is estimated using $\mathtt{kcorrect}$~\citep{blanton2007a} 
assuming a~\cite{chabrier2003} initial mass function and $\sfr$ is from the
current release of \cite{brinchmann2004}\footnote{\url{http://www.mpa-garching.mpg.de/SDSS/DR7/}}.
The uncertainties for the SDSS SMF are derived from jackknife resampling.
Figure~\ref{fig:smf_msfr} illustrates that the hydrodynamical simulations
predict significantly different SMFs and $M_*-\sfr$ relations. 
This difference, which was also recently highlighted in \cite{hahn2019c}, 
demonstrates that \emph{the hydrodynamical simulations predict galaxy
populations with significantly different physical properties}.

\subsection{IllustrisTNG100} \label{sec:tng}
The IllustrisTNG100, hereafter TNG, simulation\footnote{\url{https://www.tng-project.org/}}
is a cosmological hydrodynamic simulation of comoving volume 
$(110.7\,\mpc)^3$, with a particle mass resolution of $7.6 \times 10^{6}\ M_\odot$ for dark matter and $1.4 \times 10^{6}\ M_\odot$ for baryonic particles~\citep{nelson2018, pillepich2018, springel2018}. It improves on
the original Illustris simulation\footnote{\url{http://www.illustris-project.org}}~(\citealt{vogelsberger2014, genel2014};
public data release by~\citealt{nelson2015}), by including
magneto-hydrodynamics and updated treatments for galactic winds, metal
enrichment, and AGN feedback. Most notably, TNG uses a new implementation for
feedback from SMBH~\citep{weinberger2018}, where feedback energy is injected in
the form of a kinetic AGN-driven wind at low SMBH accretion rates. This new
implementation has been shown to alleviate discrepancies found between the
original Illustris and observations for $> 10^{13-14} M_\odot$ massive halos. 
%TNG has a baryonic mass resolution of $1.4\times10^6M_{\sun}$ \ch{temporal resolution?}\tks{why would you need to know temporal resolution? for the SFH?}.
%\todo{details on the following properties that we use in the paper: SFH, ZH}

\subsection{EAGLE} \label{sec:eagle} 
The Virgo Consortium's EAGLE
project\footnote{\url{http://www.eaglesim.org}}~\citep{schaye2015, crain2015,
mcalpine2016} is a publicly available suite of cosmological hydrodynamic
simulations constructed using {\sc Anarchy} (Dalla Vecchia et al. in prep.; 
see also Appendix A of \citealt{schaye2015}), a modified version of the 
{\sc GADGET-3} code~\citep{springel2005}. We use the L0100Ref simulation,
which has a comoving volume of $(100\,\mpc)^3$, and a baryonic mass resolution of $1.81\times 10^6M_{\sun}$. %It 
%that includes a conservative pressure-entropy formulation for the smoothed particle hydrodynamics calculation, artificial viscosity, artificial conduction and the time limiter that improve hydrodynamic computation performance. 
EAGLE has subgrid models for star formation, stellar mass loss, metal enrichment
and stellar feedback that stochastically inject thermal energy in the ISM as
in~\citep{dallavecchia2012}. The feedback energy from AGN is also added to
surrounding gas stochastically~\citep{booth2009}. Parameters of the stellar 
feedback and SMBH accretion are calibrated to broadly reproduce the $z=0$ 
stellar mass function and galaxy stellar size-stellar mass relation. Meanwhile, 
the AGN feedback efficiency is calibrated to match the SMBH-galaxy mass relation. 
%\todo{details on the following properties that we use in the paper: SFH, ZH}

\subsection{SIMBA} \label{sec:simba}
The {\sc Simba} simulation suite~\citep{dave2019}, the successor to {\sc
Mufasa}~\citep{dave2016, dave2017, dave2017a}, is a cosmological hydrodynamical
simulation construted using {\sc Gizmo}, a meshless finite mass hydrodynamics 
code~\citep{hopkins2015, hopkins2017}. Of the simulations, we use
`m100n1024', which has a box size of $(100\,h^{-1}\,\mpc)^3$ and baryonic 
mass resolution of $1.82 \times 10^7\ M_\odot$. The simulation uses the same
subgrid models as {\sc Mufasa} for $\rm H_2$ based star formation, decoupled
two-phase winds for star formation driven galactic winds, and feedback from 
Type I supernovae and AGB stars. Meanwhile, it uses updated models for AGN
feedback and on-the-fly dust model. {\sc Simba} uses a two-mode SMBH accretion 
model, torque-limited accretion for cold gas~\citep{angles-alcazar2017} and 
Bondi-based accretion for hot gas, and two-mode AGN feedback. %which ejects bipolar kinetic winds with $\sim 10^3 \kms$ for high SMBH accretion rate, and launches winds with increased velocity of $\sim 8000 \kms$ for low SMBH accretion of Eddington ratio below 2 \%.
%\todo{details on the following properties that we use in the paper: SFH, ZH}

\subsection{SDSS Galaxies} \label{sec:obs} 
For our observations, we use a galaxy sample derived from SDSS observations. We
start with the $M_* > 10^{9.4} h^{-2}M_\odot$ complete \cite{tinker2011}
volume-limited sample and then impose a $M_r < -20$ completeness cut. 
The original \cite{tinker2011} sample is derived from the SDSS DR7~\citep{abazajian2009} NYU
Value-Added Galaxy Catalog~\citep[VAGC;][]{blanton2005}. In this work, we focus 
on observables that can be consistently defined and derived in both simulations 
and observables: the $r$-band absolute magnitude, $M_r$, the optical $\gr$
color, and the $\fnuv$ color. For the SDSS sample, we use $FUV$,
$NUV$, $r$ and $g$ band absolute magnitudes from the NASA-Sloan
Atlas\footnote{\url{http://nsatlas.org/}}, which is re-reduction of SDSS DR8
\citep{aihara2011} that includes an improved background subtraction~\citep{blanton2011} 
and near and far UV photometry from GALEX. These absolute magnitudes are
derived using $\mathtt{kcorrect}$~\citep{blanton2007a}, assuming
a~\cite{chabrier2003} initial mass function. 

\subsection{Forward Modeling Observations} \label{sec:fm} 
One of the main goals of this work is to conduct an ``apples-to-apples'' comparison
between the simulations and observations using a forward modeling
approach. A crucial step is to forward model the observables for our
simulations --- in this work, we use $r$-band luminosity ($M_r$), optical color ($\gr$), and 
UV color ($\fnuv$). As the first step, we construct SEDs for all
of the simulated galaxies based on their star formation and metallicity
histories (SFH and ZH) using the Fleixble Stellar Population Synthesis model~\citep[$\mathtt{FSPS}$;][]{conroy2009, conroy2010}. 

For each simulated galaxy, we bin the total stellar mass formed by age ($t$)
and metallicity ($Z$). We use a consistent $t$, $Z$ grid for all of the simulations
to account for the variable time and mass resolutions. For each point in the
$t, Z$ grid, we generate a spectrum assuming a simple stellar population (SSP)
using $\mathtt{FSPS}$ and take the mass-weighted linear combination of them to
produce the galaxy SED. We use a
\cite{chabrier2003} initial mass function. For further details on how we
construct the SEDs, we refer readers to Starkenberg et al. (in prep.).

Next in the forward model, we apply the \eda~model to assign dust attenuation
curves for each of simulated galaxy based on its physical properties and apply
dust attenuation to its SEDs. We describe the \eda~in detail in Section~\ref{sec:dem}. 
From the dust attenuated SEDs, we measure the observables by convolving them
with transmission curves of the GALEX $FUV$, GALEX $NUV$, SDSS $g$, and SDSS $r$
broadband filters. Lastly, we add realistic noise to $M_r$, $\gr$, and $\fnuv$ 
by sampling from the observed uncertainty distributions of the NASA
Sloan-Atlas. 

In Figure~\ref{fig:obs} we present the forward modeled optical and UV color-magnitude
relations, $(\gr)-M_r$ (top) and $(\fnuv)-M_r$ (bottom), for simulated galaxies
of SIMBA (left), TNG (center) and EAGLE (right). The observables for the
simulations do not yet include any prescription for dust attenuation.
Comparison to SDSS observations (black dashed) clearly demonstrate
that {\em without dust attenuation, the hydrodynamical simulations cannot
reproduce the observed optical or UV color-magnitude relations.}

\begin{figure}
\begin{center}
\includegraphics[width=0.9\textwidth]{figs/observables.pdf} 
    \caption{\label{fig:obs}
    We present the forward modeled optical and UV color-magnitude relations of 
    simulated galaxies in SIMBA (left), TNG (center), and EAGLE (right). The
    simulations above do {\em not} yet include any prescription for dust
    attenuation.  $(\gr)-M_r$ (top) and $(\fnuv)-M_r$ (bottom) are the
    main observables used throughout the paper. They are derived using SEDs
    constructed from the star-formation and metallicity of the simulated
    galaxies. These observables are consistently defined and measured as our
    SDSS sample (Section~\ref{sec:fm}), which we include for comparison (black
    dashed). {\em Without dust, the hydrodynamical simulations cannot reproduce
    the observed optical or UV color-magnitude.}
    }
\end{center}
\end{figure}
