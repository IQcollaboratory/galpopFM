\section{Empirical Dust Modeling} \label{sec:methods}

\begin{equation} 
    F_o (\lambda) = F_i (\lambda) 10^{-0.4 A(\lambda)}
\end{equation}
where $F_o$ is the observed flux, $F_i$ is the intrinsic flux, and $A(\lambda)$
is the attenuation curve. 


\begin{equation} 
    A(\lambda) = A_V \frac{k(\lambda)}{k_V} 
\end{equation}
Throughout we use the slab model \citep{somerville1999} for the $V$ band
attenuation:  
\begin{equation} 
    A_V = -2.5 \log \left[ \frac{1 - e^{-\tau_V\,\sec i}}{\tau_V\,\sec i} \right]
\end{equation}
$i$ is the inclication, which we uniformly sample 

\subsection{Naive Model} 
We use \cite{calzetti2001} 
\[
    k_{\rm Cal}(\lambda) = 
    \begin{cases} 
        2.659 (-1.857 + 1.040/\lambda) + R_V, & 6300 \AA \le \lambda \le
        22000 \AA \\ 
        2.659 (-2.156 + 1.509/\lambda - 0.198/\lambda^2 + 0.011/\lambda^3) +
        R_V & 1200 \AA \le \lambda \le 6300 \AA
    \end{cases}
\]

and 

\begin{equation}
    \tau_V = m_\tau \log \left(\frac{M_*}{10^{10} M_\odot}\right) + c_\tau
\end{equation} 

\subsection{Less Naive Model} 
We use the attenuation curve from \cite{noll2009} 
\begin{equation}
    k(\lambda) = \left(k_{\rm Cal}(\lambda) + D(\lambda)\right) \left(
    \frac{\lambda}{\lambda_V} \right)^\delta
\end{equation}
$\lambda_V$ is the $V$ band wavelength. $D(\lambda)$ is the bump. 
\begin{equation}
    D(\lambda) = \frac{E_b(\lambda \Delta \lambda)^2}{(\lambda^2 -
    \lambda_0^2)^2 + (\lambda \Delta \lambda)^2}
\end{equation}
we assume fixed $\lambda_0 = 2175 \AA$ and $\Delta \lambda = 350\AA$. $E_b$ is
the strength of the bump.
$\delta$, the slope of the attenuation curve, also correlates with galaxy
properties.
\cite{kriek2013}, and \cite{naranyanan2018} more recently with simulations, 
demonstrated $E_b$ correlates with the slope of the attenuation curve.
So we parameterize $\delta$ and $E_b$: 
\begin{align}
    \delta(M_*) &= m_\delta \log \left(\frac{M_*}{10^{10}
    M_\odot}\right) + c_\delta \\
    E_b &= m_E~\delta + c_E
\end{align}

\subsection{Less Less Naive Model} 
We use the attenuation curve from \cite{noll2009} 


\begin{align}
    \tau_V &= m_{\tau,1} \log \left(\frac{M_*}{10^{10} M_\odot}\right) +
    m_{\tau,2} \log\,{\rm SFR} + c_\tau \\ 
    \delta(M_*, {\rm SFR}) &= m_{\delta,2} \log \left(\frac{M_*}{10^{10}
    M_\odot}\right) + m_{\delta,2} \log\,{\rm SFR} + c_\delta \\
    E_b &= m_E~\delta + c_E
\end{align}


\begin{table}
    table of free parameters 
\end{table} 


\subsection{Likelihood-Free Inference} 
Approximate Bayesian Computation with Population Monte Carlo \cite{hahn2017a},

discussion of observables and distance metric 
\cite{ishida2015} 

