\begin{figure}
\begin{center}
    \includegraphics[width=0.6\textwidth]{figs/abc_slope_AV_all.pdf}
    \caption{\label{fig:slope}
    The attenuation-slope relation of the attenuation curves predicted by the
    \eda~model for median posterior parameter values of TNG (blue) and (EAGLE). 
    For attenuation we use $A_V$; for slope we use $S = A(3000\AA)/A_V$. For
    comparison, we include the observed attenuation-slope relation from 
    GSWLC2~\citep{salim2020}. We also include the Milky Way (star) and mark the
    slope of the \cite{calzetti2001} curve (dashed). Even though, we do not fit
    any observed dust attenuation measurements, the attenuation-slope relation
    predicted by the \eda~is in excellent agreement with observations. 
    }
\end{center}
\end{figure}

\subsection{Reproducing Dust Observations} 
We demonstrated above that we can use the \eda~model in a forward model to
accurately reproduce the observed color-magnitude relations. Beyond its use in
a forward model, the \eda~model also assigns dust attenuation to simulated
galaxies. This means that we can compare the dust attenuation assigned to
simulated galaxies to the dust attenuation measured from observed galaxies.  
In Figure~\ref{fig:slope}, we present the attenuation-slope relation of the
\eda~dust attenuation curves using the median posterior parameter values 
of TNG (blue) and EAGLE (green). For attenuation, we use $A_V$; for slope, we
use the UV-optical slope, $S = A(3000\AA)/A_V$, used in the literature. For
comparison, we include the
attenuation-slope relations of GSWLC2 galaxies~\citep[grey;][]{salim2020}, the
Milky Way (star), and mark the slope of the \cite{calzetti2001} curve (dashed). 

\emph{The attenuation-slope relation predicted by the \eda~is in excellent agreement
with observations --- galaxies with higher attenuation have shallower slopes}. 
This relation is a result of the fact that dust scattering dominates absoprtion
at low attenuation while dust absorption dominates at high 
attenuation~\citep{gordon1994, witt2000, draine2003, chevallard2013}. We
emphasize that constraints on the \eda~parameters were derived soley from 
comparing the UV and optical color-magnitude relations. This underscores the 
advantages of our forward modeling approach --- without fitting to any dust 
measurements, we were able to reproduce the observed properties of dust using
the \eda.
\todo{details about the rnage of slope and amplitude} 

\begin{figure}
\begin{center}
    \includegraphics[width=0.85\textwidth]{figs/abc_attenuation.pdf}
    \caption{\label{fig:atten}
    Attenuation curves of the DEM for TNG (blue) and EAGLE (green) for 
    low (top) and high $M_*$ (bottom), star-forming (left) and
    quiescent galaxies (right). The attenuation curves are normalized at
    $3000\AA$: $A(\lambda)/A(3000\AA)$. We mark the $1\sigma$ standard
    deviation of the attenuation curves with the shaded region. For comparison,
    we include measurements of $A(\lambda)/A(3000\AA)$ from 
    observations~\citep{calzetti2000, battisti2017, salim2018} as well as
    from simulations~\citep{narayanan2018}. For star-forming galaxies, the 
    \cite{calzetti2000} and \cite{battisti2017} attenuation curves are shallower 
    than the DEM attenuation curves. However, this is primarily driven by the
    differences in $M_*$ ranges. For \cite{salim2018}, which probe a similar
    $M_*$ range, we find goood agreement. We also find good agreement with 
    median attenuation curve of star-forming galaxies in \cite{narayanan2018}.
    With DEM, we can also constrain the attenuation curves of quiescent galaxies,
    which are challenging to directly constrain from observations. Quiescent 
    galaxies have significantly shallower attenuation curves and larger
    variations than star-forming galaxies. 
    }
\end{center}
\end{figure}

In addition to the attenuation-slope relation, we more closely compare the
attenuation curves predicted by the \eda~to observations. In
Figure~\ref{fig:atten}, we present the normalized attenuation curves predicted
by the \eda~model for the median posterior parameter values of TNG (blue) and
EAGLE (green) for star-forming galaxies (left panels). In the top panel, we
present galaxies with $\log M_* < 10^{11} M_\odot$; in the bottom we present 
galaxies with $\log M_* > 10^{11} M_\odot$. We define galaxies with 
$\log {\rm SSFR}$ as star-forming. The attenuation curves are normalized at
$3000\AA$. We present the variation in the attenuation curves in the shaded
region, 1$\sigma$ standard deviation about the median. For comparison,
we include $A(\lambda)/A(3000\AA)$ from observations: \cite{calzetti2000},
\cite{battisti2017}, and \cite{salim2018}. We also include $A(\lambda)/A(3000\AA)$
from the \cite{narayanan2018} simulation. 

The \eda~attenuation curves for both TNG and EAGLE are in good agreement with the
observed \cite{salim2018} attenuation curves. They are also consistent with the
median curve of \cite{narayanan2018}. We note that the UV bump in the
\eda~parameter is fixed. On the other hand, the \eda attenuation curves are
steeper than the \cite{calzetti2000} and \cite{battisti2017} curves. These
attenuation curves, however, are derived from $M_* < 10^{9.9}M_\odot$ star-forming 
galaxies --- below our $M_*$ range. Since we find $\mdeltam < 0$ for both
the TNG and EAGLE posteriors, the \eda~attenuation curves are consistent with 
\cite{calzetti2000} and \cite{battisti2017}. Overall, the \eda~predicts
normalized attenuation curves that are in good agreement with both observations and
radiative transfer simulations for star-forming galaxies. Again, the fact that
we reproduce the dust attenuation curves for star-forming galaxies in
observations and simulations with the \eda~highlights the benefits of a forward
modeling approach. 

\todo{do we want to include the unnormalized attenuation curves?} 

%At low attenuation, dust scattering dominates absoprtion so the 
%attenuation curve steepens because red light scatters isotropically while blue light
%scatters forward~\citep{gordon1994, witt2000, draine2003}. %, which causes more optical-to-IR light to escape the galaxy than UV light
%At high attenuation dust absorption is dominant and the attenuation curve is
%shallower~\citep{chevallard2013}. For the $A_V$ range probed by the DEM, the
%$A_V$--slope relation is in good agreement with GSWLC2 galaxies~\citep[black shaded][]{salim2020}.
%They are also consistent with \cite{leja2017}. We also compare our results to
%theoretical predictions from radiative transfer models, \cite{inoue2005}
%(dotted), the radiative transfer models considered in \cite{chevallard2013}
%(dot dashed), and \cite{trayford2020} (light shaded), which all predict shallower 
%attenuation curves than observations. This is also the case for the
%\cite{narayanan2018} attenuation curves (not included). 
%\emph{The attenuation curve slopes from the DEM for are in excellent
%agreement with observations and better reproduces the observed
%attenuation--slope relation than radiative transfer models.}





