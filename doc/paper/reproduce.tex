\begin{figure}
\begin{center}
    %\includegraphics[width=0.4\textwidth]{figs/abc_slope_AV_all.pdf}
    \caption{\label{fig:slope}
    The attenuation-slope relation of the attenuation curves predicted by the
    \eda~model for median posterior parameter values of TNG (blue) and EAGLE
    (green) simulations. We use $A_V$ and $S = A(3000A)/A_V$ for attenuation
    and slope, respectively. For comparison, we include the observed 
    attenuation-slope relation from GSWLC2~\citep{salim2020}. We also include 
    the Milky Way (star) and mark the slope of the \cite{calzetti2001} curve
    (dashed). We derived the posteriors of the \eda~model from comparing the
    UV and optical color-magnitude relation and do not fit any observed dust
    attenuation measurements. Yet, using a forward modeling approach, \emph{we find 
    excellent agreement between the attenuation-slope relation predicted by
    the \eda~and observations.}
    }
\end{center}
\end{figure}

\subsection{Reproducing Dust Observations} 
We demonstrated above that we can use the \eda~model in a forward model to
accurately reproduce the observed color-magnitude relations. The \eda~assigns dust attenuation curves to each simulated galaxy. This
means that we can compare the dust attenuation assigned to the simulated
galaxies to the dust attenuation measured from 
observed galaxies. In Figure~\ref{fig:slope}, we present the attenuation-slope 
relation of the dust attenuation curves predicted by the \eda~for the median posterior parameter 
values of TNG (blue) and EAGLE (green) simulations. For attenuation we use
$A_V$ and for slope we use the UV-optical slope, $S = A(3000A)/A_V$, commonly
found in the literature. For comparison, we include the observed
attenuation-slope relations of GSWLC2 galaxies~\citep[grey;][]{salim2020}, the
Milky Way (star), and mark the slope of the \cite{calzetti2001} curve (dashed). 

\emph{The attenuation-slope relation predicted by the \eda~is in excellent 
agreement with observations}. In both the \eda~and observations, galaxies with 
higher attenuation have shallower slopes. We note that the difference in the
$A_V$ range is due to the stellar mass completeness limt of our sample, $M_* \gtrsim 
10^{10} M_\odot$. \cite{salim2020} GSWLC2 sample extends down to $M_* \sim 10^{8.5}M_\odot$. 
The attenuation-slope relation is a consequence of the fact that dust scattering 
dominates absoprtion at low attenuation while dust absorption dominates at high 
attenuation~\citep{gordon1994, witt2000, draine2003, chevallard2013}. With the
\eda, we are able to reproduce this physical relation solely from comparing the
UV and optical color-magnitude relations. This not only demonstrates the
robustness of the \eda, but it also highlights the advantages of 
our forward modeling approach. 

\begin{figure}
\begin{center}
    \includegraphics[width=0.85\textwidth]{figs/abc_attenuation.pdf}
    \caption{\label{fig:atten}
    Attenuation curves predicted by the \eda~for median posterior parameter
    values of the TNG (blue) and EAGLE (green) simulations. In the left and
    right panels, we present star-forming and quiescent galaxies classified
    using a $\log {\rm SSFR} = -11$ cut. In the top and bottom panels, we
    present galaxies with $M_* < 10^{11} M_\odot$ and $M_* > 10^{11} M_\odot$,
    respectively. The attenuation curves are normalized at
    $3000A$ and we mark the $1\sigma$ standard deviation of the attenuation 
    curves with the shaded region. For comparison, we include
    $A(\lambda)/A(3000A)$ measurements from observations~\citep{calzetti2000,
    battisti2017, salim2018} and simulations~\citep{narayanan2018}. 
    For star-forming galaxies, the \cite{calzetti2000} and \cite{battisti2017}
    attenuation curves are shallower than the \eda~attenuation curves, but 
    they probe less massive galaxies than our sample. For \cite{salim2018}, 
    which probe a similar $M_*$ range, we find goood agreement. We also find good agreement with 
    median attenuation curve of star-forming galaxies in the radiative transfer
    simulations of \cite{narayanan2018}.
    With \eda, we can also constrain the attenuation curves of quiescent 
    galaxies, which are challenging to directly measure in observations. 
    Quiescent galaxies have significantly shallower attenuation curves than
    star-forming galaxies.
    }
\end{center}
\end{figure}

We can more closely compare the attenuation curves predicted by the \eda~to 
observations. In the left panels of Figure~\ref{fig:atten}, we present the 
attenuation curves of star-forming galaxies predicted by the \eda~model for 
the median posterior parameter values of TNG (blue) and EAGLE (green). In 
the top panel, we present galaxies with $M_* < 10^{11} M_\odot$; in 
the bottom we present galaxies with $M_* > 10^{11} M_\odot$. We define
galaxies with $\log {\rm SSFR} > -11$ as star-forming. The attenuation curves 
are normalized at
$3000A$ and we present the variation in the attenuation curves in the shaded
region, 1$\sigma$ standard deviation about the median. For comparison,
we include $A(\lambda)/A(3000A)$ from observations: \cite{calzetti2000},
\cite{battisti2017}, and \cite{salim2018}. We also include $A(\lambda)/A(3000A)$
measured from the \cite{narayanan2018} simulation. The left panels of
Figure~\ref{fig:raw_atten} are the same as in Figure~\ref{fig:atten}, except
with attenuation curves that are not normalized. The normalized attenuation
curves in Figure~\ref{fig:atten} emphasize the slope while the curves in
Figure~\ref{fig:raw_atten} emphasize the amplitude. 

Based on the \eda, star-forming galaxies have higher dust attenuation at lower
wavelengths. This is driven by both TNG and EAGLE predicting star-forming
galaxies that are bluer than observations in the optical and UV wavelengths
(Figure~\ref{fig:obs}). As a result, the \eda~significantly redden star-forming galaxies. This is also why TNG, which
predicts an overall bluer star-forming population, has higher attenuation. 
In Figure~\ref{fig:raw_atten}, we also find that more massive star-forming
galaxies have higher attenuation. This is because the simulations overpredict 
luminous blue star-forming galaxies, which must be attenuated to reproduce
observations. 

The \eda~attenuation curves for both TNG and EAGLE are in good agreement with the
observed \cite{salim2018} attenuation curves. They are also consistent with the
median curve of \cite{narayanan2018}. We ignore the discrepancies between the
\eda~and \cite{narayanan2018} attenuation curves in the amplitude of UV dust
bump, since we do not vary the UV bump in the \eda. The \eda~attenuation curves 
are noticeably steeper than the \cite{calzetti2000} and \cite{battisti2017} curves. 
These attenuation curves, however, are derived from $M_* < 10^{9.9}M_\odot$ 
star-forming galaxies --- below our $M_*$ range. %Since we find $\mdeltam < 0$ for both the TNG and EAGLE posteriors, the \eda~attenuation curves are consistent with \cite{calzetti2000} and \cite{battisti2017}. 
Overall, the \eda~predicts attenuation curves that are in good agreement with 
both observations and radiative transfer simulations for star-forming galaxies. 
Again, the fact that we reproduce the detailed dust attenuation curves of star-forming 
galaxies in observations and simulations with the \eda~without fitting for
them, highlights the advantages of a forward modeling approach. 

%At low attenuation, dust scattering dominates absoprtion so the 
%attenuation curve steepens because red light scatters isotropically while blue light
%scatters forward~\citep{gordon1994, witt2000, draine2003}. %, which causes more optical-to-IR light to escape the galaxy than UV light
%At high attenuation dust absorption is dominant and the attenuation curve is
%shallower~\citep{chevallard2013}. For the $A_V$ range probed by the DEM, the
%$A_V$--slope relation is in good agreement with GSWLC2 galaxies~\citep[black shaded][]{salim2020}.
%They are also consistent with \cite{leja2017}. We also compare our results to
%theoretical predictions from radiative transfer models, \cite{inoue2005}
%(dotted), the radiative transfer models considered in \cite{chevallard2013}
%(dot dashed), and \cite{trayford2020} (light shaded), which all predict shallower 
%attenuation curves than observations. This is also the case for the
%\cite{narayanan2018} attenuation curves (not included). 
%\emph{The attenuation curve slopes from the DEM for are in excellent
%agreement with observations and better reproduces the observed
%attenuation--slope relation than radiative transfer models.}
