\begin{figure}
\begin{center}
    \includegraphics[width=0.9\textwidth]{figs/abc_observables.pdf}
    \caption{\label{fig:dem}
    The optical and UV color-magnitude relations predicted by the DEM with 
    the median ABC posteriors for the SIMBA (orange), TNG (blue), and EAGLE
    (green) hydrodynamical simulations. For comparison, we include the 
    $(\gr) - M_r$ (top panels) and $(\fnuv) - M_r$ (bottom panels) relations
    for SDSS (black dashed). With the DEM, the simulations produce dramatically 
    different observables than without any dust prescription (Figure~\ref{fig:obs}). 
    Hence, dust must be account for when interpreting and comparing simulations.
    Moreover, with the DEMs, all three simulations produce color-magnitude
    relations consistent with SDSS. Since different simulations can 
    produce reproduce observations by varying dust, dust significantly limits
    our ability to constrain the physical processes that go into galaxy
    simulations. 
    }
\end{center}
\end{figure}

\section{Results} \label{sec:results}
% dust model can reproduce color magnitude relation
Without dust attenuation, all of the hydrodynamical simulations struggle to 
reproduce the $(\gr) - M_r$ and $(\fnuv) - M_r$ relations of SDSS (Figure~\ref{fig:obs}). 
In optical, the simulations predict significantly bluer colors and find  
broader differences in color, ${\sim}0.5~mag$, between star-forming and
quiescent galaxies. In the UV, they predict a subpopulation of galaxies that
have red $\fnuv$ colors, $\fnuv{\sim}3~mag$, beyond the SDSS observations. 
These are quiescent galaxies with particularly low SSFR, $\log\,{\rm SSFR} <
-12\,yr^{-1}$. For the rest of the galaxies, the simulations predict
significantly bluer UV colors than SDSS. 

On the other hand, \emph{with the \eda~model all of the simulations are able to
reproduce the color-magnitude relations of SDSS.} In Figure~\ref{fig:dem}, 
we present the optical and UV color-magnitude relations predicted by the 
\eda~for the SIMBA (orange), TNG (blue), and EAGLE (green) simulations. 
For the \eda~parameters, we use the median of the posterior distributions 
inferred using ABC (Figure~\ref{fig:abc}). The contours mark the $68\%$ and 
$95\%$ confidence intervals of the distributions.

%By comparing the color-magnitude relations produced with and without the \eda, we can examine how dust attenuation can produce observations. 

Dust dramatically impacts the observables of simulations. The \eda~affects the 
the optical and UV color-magnitude relations in three major ways to produce 
agreement with SDSS. First, the \eda~significantly reddens simulated galaxies in both 
the optical and UV.  Overall, $\gr$ colors are ${\sim}0.25~mag$ redder and 
$\fnuv$ colors ${\sim}0.5~mag$ redder. Second, the \eda~significantly attenuates, by
${\sim}0.5~mag$, intrinsically blue galaxies (\ie~star-forming galaxies with
$\log~{\rm SSFR} > -10.5$). As a result, there are no luminous optically blue 
galaxies ($M_r < -21$ and $\gr < 0.5$) --- consistent with observations.
Lastly, in the UV, the \eda~attenuates low SSFR quiescent galaxies that are 
intrinsically red in the UV, $\fnuv{\sim}3$, by ${\sim}1~mag$. Thus, unlike in
Figure~\ref{fig:obs}, the UV color-magnitude relations from the \eda~does not
produce galaxies with high $\fnuv$ and is in good agreement with SDSS. 

% Do we want to comment on the variation in the eda attenuation curve and how
% that mixes up the color for a given SSFR? 

% paragraph on SIMBA discrepancy
For SIMBA, the \eda~substantially improves the agreement with observations;
however, there are still some significant discrepancies in the color-magnitude
relations with SDSS. SIMBA produces a $\gr$ and $\fnuv$ color distributions that 
are respectively narrower and broader than SDSS. Also, the inferred \eda~parameters 
for SIMBA differ significantly from the parameters of TNG and EAGLE
(Figure~\ref{fig:abc}). These discrepancies are primarily driven by the
large population of high $\log{\rm SSFR} > -9.5$ ``starburst'' galaxies that
lie above the SFS, which are {\em only} present in SIMBA (Figure~\ref{fig:smf_msfr}). 
Without dust attenuation, these starburst galaxies are blue $\gr\sim0.1$ and
have high luminosity, $M_r < -22$. Such high luminosity blue galaxies are not
present in observations, so these SIMBA galaxies need to be both strongly 
attenuated and reddened. This would, however, contradict the attenuation-slope 
relation established in both observations and simulations where galaxies with 
higher attenuation have shallower
slopes~\citep{inoue2005,chevallard2013,salim2018,salim2020,trayford2020}.
To further examine the impact of the starburst galaxies in SIMBA, we try
running the \eda for SIMBA excluding galaxies with $\log{\rm SSFR} > -9.5$ and 
with median parameter values of the TNG and EAGLE posteriors. Without starburst
galaxies, SIMBA is able to produce color-magnitude relations in good agreement 
with observations. {\em Hence, we conclude that the excess starburst population in 
SIMBA is in tension with observations}. For the rest of the paper, we focus on
the TNG and EAGLE simulations.  

% comparison to literature 
Previous works in the literature have also compared simulations with different
dust prescriptions to observations in color-magnitude space. For EAGLE, 
\cite{trayford2015} calculate colors and luminosities with the {\sc Galaxev}
population synthesis models and a two-component screen model for dust. More
recently, \cite{trayford2017} calculated optical colors for EAGLE using {\sc
Skirt}, a Monte Carlo radiative transfer code~\citep{camps2015}, to model the
dust. At stellar masses and luminosities comparable to our SDSS sample, both 
\cite{trayford2015} and \cite{trayford2017} produce red sequences bluer than 
in GAMA observations. Also, \cite{trayford2015} predict an excess of luminous 
blue galaxies. Although a detailed comparison is difficult since both works 
compare to different observations, we note that with the \eda, EAGLE is able 
to successfully reproduce the position of the SDSS red sequence and does not 
predict a significant excess of luminous blue galaxies. Also using EAGLE and 
{\sc Skirt}, \cite{baes2019} find that they overesimtate the observed cosmic 
SED (CSED) in the UV regime and produce significantly higher $\fnuv$ color 
than GAMA. The \eda predict $\fnuv$ in good agreement with SDSS. 
%\cite{baes2019}: EAGLE+SKIRT SED compparison with GAMA Far UV is not attenuated enough. underrestimates optical and NIR
For TNG, \cite{nelson2018} calculate optical colors using a dust model that
includes attenuation due to dense gas birth clouds surrounding young stellar
populations and also due to simulated distribution of neutral gas and metals.
They find bluer red sequence peaks and a narrower blue cloud compared to SDSS.
We find neither of these discrepancies for the TNG with the~\eda. The
\eda~is a simpler prescription for applying dust attenuation than the dust
models in these works. Yet, with its flexibility, we are able to produce
optical and UV color-magnitude relations that are in good agreement with
observations. 
%better agreement than observations than these works.

% without being fit, the EDA reproduces the attenuation-slope relation and SF
% attenuation of SDSS observations  
\begin{figure}
\begin{center}
    %\includegraphics[width=0.9\textwidth]{figs/abc_slope_AV_starforming.pdf}
    \caption{\label{fig:slope}
    The attenuation-slope relation of star-forming galaxies ($\ssfr >
    10^{-11}yr^{-1}$), using the attenuation curves predicted by our  
    \eda~prescription for the median posterior parameter values of SIMBA
    (left), TNG (center) and EAGLE (right). 
    For comparison, we include the observed attenuation-slope relation
    from GSWLC2~\citep{salim2020}. 
    We use $A_V$ and $S = A(1500\AA)/A_V$ as measurements of attenuation and
    slope, respectively. 
    \chedit{
        \emph{The \eda~does not predict $A_V < 0.3$ because star-forming galaxies in
        the simulations are too lumnious and require significant attenuation to
        reproduce observations.}
        Beyond $A_V > 0.3$, however, there is good agreement between the
        attenuation-slope relation predicted by the \eda~and observations. 
    }
    }
\end{center}
\end{figure}

\subsection{Comparison to Dust Observations} \label{sec:reproduce}
With our \eda~prescription, we are able to accurately reproduce the
observed optical and UV color-magnitude relations with our simulations. 
In addition to  reproducing observations, since the \eda~assigns dust
attenuation curves to each simulated galaxy, we can compare the
\eda~attenuation curves to dust attenuation measured from
observations. 
We begin with the well-established attenuation-slope relation: star-forming
galaxies with higher dust attenuation have shallower attenuation curves. 
This relation is a consequence of dust scattering dominating absorption at
low attenuation while dust absorption dominates at high
attenuation~\citep{gordon1994, witt2000, draine2003, chevallard2013}. 
In Figure~\ref{fig:slope}, we present the attenuation-slope relation of
star-forming galaxies with $\ssfr > 10^{-11}yr^{-1}$ based on the
dust attenuation curves predicted by the \eda~for the median posteriors of
SIMBA (left), TNG (center) and EAGLE (right).
For comparison, we include the observed attenuation-slope relations of
GSWLC2 galaxies~\citep[grey shaded;][]{salim2020}.
For attenuation we use $A_V$ and for slope we use the UV-optical slope, $S
= A(1500\AA)/A_V$, commonly found in the literature. 
The contours mark the 68 and 95 percentiles. 
Most noticably, we find that the \eda~does not predict $A_V < 0.3$. 
\chedit{
    This is not due to the selection function imposed by our forward model. 
    The attenuation-slope relation of star-forming galaxies in GSWLC2 does not
    change significantly if we impose similar selection cuts as our
    observational sample.
    Instead, the lack of star-forming galaxies with $A_V < 0.3$ is a
    consequence of SIMBA, TNG, and EAGLE predicting star-forming galaxies that
    are more luminous than observations.  
}
All of the simulations have star-forming galaxies with intrinsic $M_r <
-21$ and $\gr < 0.5$ (Figure~\ref{fig:obs}). 
This is further corroborated by the $\sfr-M*$ relations in
Figure~\ref{fig:smf_msfr}, where the simulations all have star-forming
galaxies with $M_* > 10^{11}M_\odot$, not found in SDSS. 
To reproduce the SDSS optical color-magnitude relation these galaxies would
need to be significantly reddened and attenuated. 
Hence, any dust prescription for the simulations would require high 
$A_V$ for star-forming galaxies.
Nevertheless, for $A_V > 0.3$, we find good agreement between the 
attenuation-slope relation predicted by the \eda~and observations. 

%We note that the difference in the $A_V$ ranges is due to the $M_r$
%completeness limit imposed by our forward model (Section~\ref{sec:fm}).
%The GSWLC2 sample in \cite{salim2020} extends down to $M_* \sim
%10^{8.5}M_\odot$; however, the TNG and EAGLE samples do not extend below
%$M_* \sim10^{10}M_\odot$.
%\emph{The \eda~predicts attenuation-slope relations for TNG and EAGLE that
%are in excellent agreement with observations.} 

\begin{figure}
\begin{center}
    \includegraphics[width=0.5\textwidth]{figs/abc_sf_attenuation.pdf}
    \caption{\label{fig:sfatten}
    The normalized attenuation curves of star-forming galaxies predicted by
    the \eda~for median posterior parameter values of SIMBA (orange), TNG
    (blue), and EAGLE (green).  
    Galaxies with $\log \ssfr > -11~yr^{-1}$ are classified as star-forming. 
    The attenuation curves are normalized at $3000\AA$ and we mark the
    68 percentile of the attenuation curves with the shaded region.
    For comparison, we include $A(\lambda)/A(3000\AA)$ measurements from
    the~\cite{narayanan2018} radiative transfer simulation (dashed) and
    \cite{salim2018} observations (dotted).
    %The \cite{calzetti2000} and \cite{battisti2017} attenuation curves are shallower than the \eda~attenuation curves; however, they probe lower $M_*$ galaxies than our forward modeled TNG and EAGLE samples.  For attenuation curve from \cite{salim2018}, which probe a similar $M_*$ range, we find goood agreement. 
    {\em The \eda~predict attenuation curves of star-forming galaxies that
    are in good agreement with the attenuation curves measured from
    the simulation and observations in the literature.}
    %We also find good agreement with median attenuation curve of star-forming galaxies in the radiative transfer simulations of \cite{narayanan2018}.
    }
\end{center}
\end{figure}

In addition to the attenuation-slope relation, we can also directly compare
the attenuation curves predicted by the \eda~to measurements from
observations for star-forming galaxies. 
In Figure~\ref{fig:sfatten}, we present the normalized attenuation curves
of star-forming galaxies predicted by the \eda~for the median posterior
parameter values of SIMBA(orange), TNG (blue), and EAGLE (green).
We again define galaxies with $\ssfr > 10^{-11}{yr}^{-1}$ as star-forming.
The attenuation curves are normalized at $3000\AA$ and we present the
variation in the attenuation curves in the shaded region, 68 percentile. 
For comparison, we include $A(\lambda)/A(3000\AA)$ from the
\cite{narayanan2018} radiative transfer simulation (dashed) and 
observations~\citep[][dotted]{salim2018}. 
The attenuation curve from \cite{salim2018} corresponds to star-forming
galaxies with $M_* > 10^{10.5}M_\odot$, a similar $M_*$ range as our
forward modeled samples. 
Since we do not vary the UV bump in our \eda~prescription, we ignore any
discrepancies in the amplitudes of the bump. 
\emph{Overall, we find good agreement between the \eda~attenuation curves for
star-forming galaxies and the attenuation curves from observations and
simulations in the literature.}

%Again, the fact that we reproduce the detailed dust attenuation curves of star-forming galaxies in observations and simulations with the \eda~without fitting for them, highlights the advantages of a forward modeling approach. 

%The \eda~attenuation curves are slightly steeper than the \cite{calzetti2000} and \cite{battisti2017} curves. 
%These attenuation curves, however, are derived from $M_* < 10^{9.9}M_\odot$
%star-forming galaxies, which lie below the $M_*$ limit of our forward
%modeled TNG and EAGLE samples. 
%Meanwhile, the TNG and EAGLE \eda~attenuation curves are in good agreement
%with the \cite{salim2018} attenuation curve for $M_* > 10^{10.5}M_\odot$ star-forming galaxies. 
%They are also consistent with the median curve of \cite{narayanan2018}. 

%The \eda~attenuation curves are noticeably steeper than the \cite{calzetti2000} and \cite{battisti2017} curves.  These attenuation curves, however, are derived from $M_* < 10^{9.9}M_\odot$ star-forming galaxies --- below our $M_*$ range. 
%Since we find $\mdeltam < 0$ for both the TNG and EAGLE posteriors, the \eda~attenuation curves are consistent with \cite{calzetti2000} and \cite{battisti2017}. 


%\chedit{ 
%    The \eda~predicts higher dust attenuation at lower wavelenghts for
%    star-forming galaxies.
%    Without dust attenuation, both TNG and EAGLE predict star-forming galaxies
%    that are bluer in the optical and UV than observations
%    (Figure~\ref{fig:obs}).
%    To reproduce the SDSS, the \eda~significantly reddens star-forming galaxies.
%}
%In Figure~\ref{fig:raw_atten}, we also find that more massive star-forming
%galaxies have higher attenuation. This is because the simulations overpredict 
%luminous blue star-forming galaxies, which must be attenuated to reproduce
%observations. 


%At low attenuation, dust scattering dominates absoprtion so the 
%attenuation curve steepens because red light scatters isotropically while blue light
%scatters forward~\citep{gordon1994, witt2000, draine2003}. %, which causes more optical-to-IR light to escape the galaxy than UV light
%At high attenuation dust absorption is dominant and the attenuation curve is
%shallower~\citep{chevallard2013}. For the $A_V$ range probed by the DEM, the
%$A_V$--slope relation is in good agreement with GSWLC2 galaxies~\citep[black shaded][]{salim2020}.
%They are also consistent with \cite{leja2017}. We also compare our results to
%theoretical predictions from radiative transfer models, \cite{inoue2005}
%(dotted), the radiative transfer models considered in \cite{chevallard2013}
%(dot dashed), and \cite{trayford2020} (light shaded), which all predict shallower 
%attenuation curves than observations. This is also the case for the
%\cite{narayanan2018} attenuation curves (not included). 
%\emph{The attenuation curve slopes from the DEM for are in excellent
%agreement with observations and better reproduces the observed
%attenuation--slope relation than radiative transfer models.}
 

\subsection{The Galaxy -- Dust Attenuation Connection}  
Next, in Figure~\ref{fig:atten}, we present the normalized attenuation curves 
of the TNG (blue) and EAGLE (green) DEM for low (top) and high $M_*$ (bottom),
star-forming (left) and quiescent galaxies (right). The attenuation curves are
normalized at $3000\AA$ in order to emphasize their slopes. The DEM includes
significant variations in the dust attenuation through the slab model
and the dependence on galaxy properties. We present this variation in the
shaded region (1$\sigma$ standard deviation about the median). For comparison,
we include $A(\lambda)/A(3000\AA)$ from observations~\citep{calzetti2000, battisti2017, salim2018} 
as well as from simulations~\citep{narayanan2018}. The DEM attenuation curves 
for star-forming centrals in TNG and EAGLE are in good agreement with the
\cite{salim2018} attenuation curves for $10^{9.5} < M_* < 10^{10.5}M_\odot$ 
(top left) and $10^{10.5} < M_*M_\odot$ star-forming galaxies (bottom
left). They are also consistent with the median curve of \cite{narayanan2018} 
star-forming galaxies. On the other hand, they are steeper than the
\cite{calzetti2000} and \cite{battisti2017} curves. However, we note that 
the \cite{calzetti2000} and \cite{battisti2017} curves are derived from
$M_* < 10^{9.9}M_\odot$ star-forming galaxies (below the our $M_*$ range).
We find that lower $M_*$ star-forming galaxies have shallower attenuation 
curves, especially for TNG. If this $M_*$ dependence continues below our $M_*$
range, we expect a better agreement the \cite{calzetti2000} and \cite{battisti2017} 
curves as well. In \cite{salim2018} and \cite{narayanan2018}, they find 
large variations in the slopes of the attenuation curve. This is, however, 
over their entire $M_*$ range. For a narrow $M_*$ range of star-forming
galaxies, we find significantly less variation. 
\emph{Overall, the DEM attenuation curves for star-forming galaxies are in good
agreement with attenuation curves in the literature from both observations and
simulations.}

So far, we have demonstrated that with the DEM we successfully reproduce the 
observed color-magnitude relation as well as the attenuation--slope relation.
We also find good agreement between the attenuation curves of the DEM and in
the literature for star-forming galaxies. Besides reproducing observations and
trends in the literature, the constraints on the DEM provide insights into dust
attenuation in galaxies. In fact, the parameterization of the DEM makes it
especially easy to interpret correlation between dust attenuation and galaxy
physical properties. In Figure~\ref{fig:abc}, for all three simulations we find
we find significant positive $M_*$ dependence of $\tau_V$ ($\mtaum \sim 2$) 
consistent with previous works in the literature. \cite{burgarella2005}, for instance, found significant
positive $M_*$ dependence in $FUV$ attenuation in NUV-selected and FIR-selected
samples. \cite{garn2010} and \cite{battisti2016} also find positive $M_*$ 
dependence in SDSS star-forming galaxies. Most recently, \cite{salim2018} 
find higher $V$ and $FUV$ attenuation for more masssive star-forming galaxies
in GSWLC2. In addition to the $M_*$ dependence, the DEM posteriors also reveal the
correlation between dust attenuation and star formation. Ignoring SIMBA, which
flips the color versus $\sfr$ relation, we find $\mtaus\sim-1$ --- galaxies 
with lower $\sfr$ have higher attenuation. Observations that examine the
relationship between dust attenuation and $\sfr$~\citep[\eg][]{garn2010,
reddy2015, battisti2016, battisti2017, salim2018} have thus far focused
primarily on star-forming galaxies. With the DEM, we confirm that 
\emph{galaxies with higher $M_*$ have overall higher dust attenuation
and find that galaxies with lower $\sfr$ have overall higher dust attenuation}. 


In addition to the correlation with galaxy properties, the DEM also reveals
dust attenuation in quiescent galaxies (left panels of Figure~\ref{fig:atten}). 
This is particularly valuable since there are many challenges to measuring
attenuation curves for quiescent galaxies directly from observations. For
instance, methods that rely on IR luminosities can be contaminated by MIR 
emission from AGN heating nearby dust~\cite{kirkpatrick2015}. Even SED 
fitting methods require accounting for AGN MIR emission~\citep{salim2016,
leja2018, salim2018}. SED fitting methods also struggle to tightly constrain
dust attenuation due to the degeneracies with star formation history and 
metallicity (\ch{cite?}). By forward modeling the optical and UV photometry
with the DEM, we do not face these issues. For both TNG and EAGLE, we find 
that quiescent galaxies have significantly shallower attenuation curve than
star-forming galaxies. They also have significantly larger variations than
star-forming galaxies. % \ch{explanation for why DEM model does this} 

%They find that
%star-forming galaxies with higher SFR have higher attenuation; however, this
%trend is driven by the $M_*$ dependence since star-forming galaxies lie on 
%the star-forming sequence~\citep{garn2010, battisti2017}. At fixed $M_*$,
%observations find no strong $\sfr$ dependence for the star-forming population. 
%Since previous works do not include quiescent galaxies, %\cite{tress2018} find positive dependence between E(B-V) (color excess) with M* and negative dependence with
%SSFR for 1753 star-forming galaxies within 1.5 < z < 3.  

%For instance, \cite{garn2010} find that SDSS star-forming galaxies with higher
%SFR have higher H$\alpha$ attenuation. \cite{battisti2016} find a consistent
%correlation for the Balmer optical depth of star-forming galaxies in GALEX and SDSS 
%using 10000 SF galaxies GALEX-SDSS. Although at higher redshifts,  
%\cite{reddy2015} also find this correlation among $z{\sim}2$ star-forming galaxies of
%MOSFIRE Deep Evolution Field.

%\cite{battisti2017} using 5000 SF galaxies from found little stellar mass dependence in the opposite direction (less attenuation at higher stellar masses). But they have big error bars and only probe up to 9.7
%\cite{reddy2015} SF galaxies from $z\sim2$ MOSFIRE Deep Evolution Field survey find strong correlation with SFR. %ionized gas is more reddened relative to the stellar continuum with increasing SFR 

%We can also examine the correlation between attenuation curve slopes and galaxy properties with the DEM. For TNG and EAGLE, we find that 
%\cite{leja2017} find composite and AGN galaxies generally have shallower  
%slopes, although their sample is limited to only 129 galaxies, In contrast,
%\cite{salim2018} find that quiescent galaxies in the GALEX-SDSS-WISE Legacy
%Catalog 2~\citep[GSWLC2;][]{salim2019} have significantly steeper curves. They,
%however, also find significantly steeper curves for the starburst population 
%(\ie~galaxies above the SFS). With no consensus in the literature and few
%observations that examine the correlation between $\delta$ and galaxy
%properties, the lack of $M_*$ and $\sfr$ dependence we find on $\delta$ is an
%interesting prediction for future observations. 


%\cite{calzetti2000} find  slopes of $2.3 < S < 2.9$ for low-redshift starburst galaxies, 
%\cite{burgarella2005} find $2.5 < S < 6.2$ for 50 UV and 100 IR selected galaxies,
%\cite{johnson2007} find $S\sim2.5$ for 1000 nearby galaxies, 
%\cite{conroy2010} find $S\sim4.5$ for 3400 $10^{9.5} < M_* < 10^{10} M_\odot$ disk galaxies,
%\cite{wild2011} find $2.5 < S < 4.5$ for 23,000 $z{\sim}0.07$ star-forming galaxies, 
%\cite{battisti2016, battisti2017} find slopes consistent with Calzetti (S=2.4) for SF galaxies 
%\cite{leja2017} 130 relatively massive galaxies 2 < S < 15
%\cite{salim2018} 230,000 SDSS galaxies 2 < S < 15 with median S = 5.4
% high z
%\cite{kriek2013}  using stacked SEDs of medium- and broadband photometry of
%galaxies at 0.5 < z < 2 find an average slope of delta=-0.2  but they restrict
%to galaxies with moderate to high optical attenuations (AV> 0.5), 
%\cite{salmon2016} is also spot on. 


% attenuation-slope relation 

%\cite{trcka2020}: EAGLE+SKIRT with CIGALE to get physical properties of
%galaxies. trcka2020 compares to dustpedia~\citep{davies2017} They lok at
%IRX-beta relation.

\begin{figure}
\begin{center}
    \includegraphics[width=0.85\textwidth]{figs/abc_attenuation.pdf}
    \caption{\label{fig:atten}
    Attenuation curves of the DEM for TNG (blue) and EAGLE (green) for 
    low (top) and high $M_*$ (bottom), star-forming (left) and
    quiescent galaxies (right). The attenuation curves are normalized at
    $3000\AA$: $A(\lambda)/A(3000\AA)$. We mark the $1\sigma$ standard
    deviation of the attenuation curves with the shaded region. For comparison,
    we include measurements of $A(\lambda)/A(3000\AA)$ from 
    observations~\citep{calzetti2000, battisti2017, salim2018} as well as
    from simulations~\citep{narayanan2018}. For star-forming galaxies, the 
    \cite{calzetti2000} and \cite{battisti2017} attenuation curves are shallower 
    than the DEM attenuation curves. However, this is primarily driven by the
    differences in $M_*$ ranges. For \cite{salim2018}, which probe a similar
    $M_*$ range, we find goood agreement. We also find good agreement with 
    median attenuation curve of star-forming galaxies in \cite{narayanan2018}.
    With DEM, we can also constrain the attenuation curves of quiescent galaxies,
    which are challenging to directly constrain from observations. Quiescent 
    galaxies have significantly shallower attenuation curves and larger
    variations than star-forming galaxies. 
    }
\end{center}
\end{figure}

% variation of attenuation curve  
%\ch{what do we learn about quiescent galaxy attenuation?} 
%In \cite{leja2017}, they similar find composite and AGN galaxies
%to have shallow attenuation curves with higher $A_V$; however, the comparison
%is limited due to ther smaller sample size (129 galaxies). In contrast,  
%\cite{salim2018} find that quiescent galaxies in GSWLC2 have significantly
%steeper curves; they, however, focus their analysis mainly on star-forming 
%galaxies.

% Kartheik on why dust is hard to constrain for quiescent galaxies: 
%firstly because quiescent galaxies don't have much recent SFR and therefore not much dust (their most recently produced dust has likely happened more than a dust destruction timescale ago), 
% secondly because the continuum for quiescent galaxies is super hard to fit (and break degeneracies with SFH & metallicity), which leads to poor(er) constraints. You could basically think of this in terms of dust/total SNR, which drops sharply for this population.

The DEM produces optical and UV color-magnitude relations overall consistent
with SDSS. There are, however, still a few discrepancies between the DEM
observables and SDSS.  For instance, the DEM produces broader distributions
overall than observations.
Galaxies in SDSS sharply cut-off above the red sequence, while some galaxies in
the DEM broadly extend beyond the cut-off. The DEM also produces galaxies 
more luminous galaxies than SDSS. Nevertheless, the DEM better reproduces
observations than other works. Furthermore, we chose linear parameterization
for $\tau_V$ and $\delta$ in the DEM for simplicity. However, the
empirical framework of the DEM can easily be extended to more flexible
parameterizations that better reproduce observations --- the only challenge
would be to find a well-motivated parameterization from observations.

In the DEM, we make a few other assumptions and choices. For simulated galaxies
with $\sfr=0$, we directly sample their observables from the distributions of 
SDSS quiescent galaxies. These $\sfr=0$ galaxies do not have recent
star-formation and also have 0 gas mass\todo{CHH: @tjitske is this for all
sims} so we would expect them to also have no dust. However, without
attenuating these galaxies, the simulations struggle to reproduce observations.
Our prescription for $\sfr = 0$ galaxies ensures that $\sfr=0$ galaxies do not
impact our results, without delving into the issue further. In
Appendix~\ref{sec:res}, we describe our prescription in detail and discuss the
limitations of the simulations near the mass and temporal resolutions or their
gas prescriptions. Another assumption in the DEM is how we assign $A_V$ using 
the slab model (Eq.~\ref{eq:slab}). The slab model is consistent with the
correlation between attenuation and inclination in
observations~\citep{conroy2010b, wild2011, battisti2017, salim2020} and
simulations~\citep[\eg][]{chevallard2013, narayanan2018, trayford2020}.
Furthermore, it is able to reproduce the SDSS $A_V$ distribution
(Figure~\ref{fig:av_dist}). When we further test the robustness of our results
by replacing the slab model with a more flexible truncated normal distribution 
in Appendix~\ref{sec:nonslab}. We find that our results are not significant
impacted. We therefore conclude that our results are robust to the assumptions and choices we make in the DEM. 

% Salim(2020): Chevallard et al. (2013), who aggregated and analyzed a diverse series of theoretical attenuation law studies by Pierini et al. (2004), Tuffs et al. (2004), Silva et al. (1998) and Jonsson et al. (2006), and showed that all the stud- ies predict, with some normalization differences, a relationship between the optical depth AV and attenuation law slope.

% Salmon+(2016): There is evidence that galaxy inclination correlates with the strength of Lyα emission, such that we observe less Lyα equivalent width for more edge-on galaxies (Charlot & Fall 1993; Laursen & Sommer-Larsen 2007; Yajima et al. 2012; Verhamme et al. 2012; U et al. 2015)
% Therefore, based on physi- cal models, one expects that galaxies with “greyer” dust laws and larger overall attenuation should have higher inclinations
% Salim+(2018): Chevallard et al. (2013) furthermore show that the depend- ence of the slope on AV is the same irrespective of whether the AV is driven by different levels of intrinsic (face-on) attenuation or is the result of inclined viewing geometry. 

% recap what we learn from the  
In this work, we present the DEM, a framework for statistically assigning
attenuation curves to galaxy populations. We apply the DEM to the SIMBA,
TNG, and EAGLE hydrodynamical simulations and forward model optical and UV
color-magnitude relations. %Then we compare the forward modeled DEM outputs to SDSS observations to reveal insights into the simulations as well as dust attenuation in galaxies.
With the DEM, we reproduce the color-magnitude relations of the SDSS
observations for all three simulations. Also, the attenuation curves of the DEM
closely reproduce the observed attenuation--slope relation, better than
radiative transfer models. Furthermore, the DEM produces attenuation curves
that are in good agreement with the literature for star-forming galaxies.

Focusing on the DEM for TNG and EAGLE, we find significant $M_*$ and $\sfr$
dependences in the amplitude of dust attentuation, $A_V$. More massive
galaxies have higher dust attenuation; galaxies with lower 
$\sfr$ have  higher dust attenuation. The DEM is also able to constrain the
attenuation curves for quiescent galaxies, which have few constraints from
observations. We find that quiescent galaxies have shallower attenuation
curves with larger variation than star-forming galaxies.

By reproducing SDSS observations with the DEM for different hydrodynamical
simulations, we demonstrate that accounting for dust attenuation is essential 
to reproduce observations. However, based on our current understanding of dust
there is enough flexiblity to reproduce observations even for simulations that
predict galaxy populations with significantly different physical properties. 
In fact, for SIMBA the inferred dust attenuation reverses the established 
relationship between color and $\sfr$ in order to account for SIMBA
overpredicting a large starburst population at $<10^{10}M_\odot$. Since
adjusting dust alone can reproduce observations, dust is highly degenerate with the variations in
subgrid physics across simulations. In other words, if we were to marginalize
over dust we would not be able to differentiate between the various
hydrodyanmical models using observations. So detailed comparisons across
simulations and to observations likely overinterpret the differences and
similarities found in simulations. Therefore, the current limitations in our
understanding of dust is a major bottleneck for investigating galaxy formation
using simulations.

%The DEM provides a simple empirical model for applying dust attenuation to
%galaxies. With only a few parameters, it allows $M_*$ and $\sfr$ dependences in
%dust attenuation curves, reproduces the attenuation--slope relation, and
%produces attenuation curves with significant variation. Combined with a
%simulation-based inference method such as ABC-PMC, it provides a direct
%framework for inferring dust attenuation from observations. For those 
%uninterested in dust, the DEM provides a
%straightforward framework for marginalizing over dust. \todo{IQ3 description?} 


Our results also highlight another key point. Even for three
simulations that produce significantly different SMFs and $M_*-\sfr$ relations
(Figure~\ref{fig:smf_msfr}), the DEM is able to produce observables that agree
with observations. 



Our current understanding of dust, which is
encapsulated in the DEM, has enough flexiblity to reproduce observations 
for simulations that predict galaxy populations with different physical
properties, even if it contradicts the established relationship between color
and $\sfr$. This means that marginalizing over dust would leave little 
constraining power on the subgrid galaxy physics of the simulations. 
Therefore, \emph{current limitations in our understanding of dust in 
galaxies significant impedes our ability to investigate galaxy 
formation from simulations.}



