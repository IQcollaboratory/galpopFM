\begin{figure}
\begin{center}
    \includegraphics[width=\textwidth]{figs/abc.pdf}
    \caption{\label{fig:abc}
    Posterior distributions of our DEM parameters for the SIMBA (orange), TNG
    (blue), and EAGLE (green) hydro simulations. The contours mark the $68\%$
    and $95\%$ confidence intervals. We describe the parameters in
    Section~\ref{sec:dem} and Table~\ref{tab:free_param} and derive these
    posteriors using Approximate Bayesian Computation (Section~\ref{sec:abc}). 
    In all simulations, dust attenuation increases for higher $M_*$ galaxies 
    ($m_{\tau,M_*} \sim 2$). The simulations also have consistent optical 
    depth amplitudes ($c_\tau$). However, the ${\rm SFR}$ dependence of
    $\tau_V$ is different among the simulations. For TNG and EAGLE,
    star-forming galaxies have lower $\tau_V$; for SIMBA quiescent galaxies
    have lower $\tau_V$. Meanwhile, for the slope offset of the attenuation
    curve, $\delta$, we find little $M_*$ and ${\rm SFR}$ dependence in the
    simulations and that the amplitude ($c_\tau$) is consistent with 0. 
    }
\end{center}
\end{figure}

\begin{figure}
\begin{center}
    \includegraphics[width=0.9\textwidth]{figs/abc_observables.pdf}
    \caption{\label{fig:dem}
    $(G-R) - M_r$ color-magnitude (top panels) and $(FUV-NUV) - M_r$ (bottom
    panels) relations predicted by the median DEM posteriors for the SIMBA
    (orange), TNG (blue), and EAGLE (green) hydro simulations. For comparison, 
    we include the observables for SDSS in the left-most panel
    (Section~\ref{sec:obs}). The median posterior DEMs produce dramatically 
    different observables than when we do not include any dust prescription
    (Figure~\ref{fig:obs}). Hence, dust must be account for when interpreting 
    and comparing simulations. Moreover, with the DEMs, all three simulations
    produce observables consistent with SDSS. Since different simulations can 
    produce reproduce observations by varying dust, dust significantly limits
    our ability to constrain the physical processes that go into galaxy
    simulations. 
    }
\end{center}
\end{figure}


\begin{figure}
\begin{center}
    \includegraphics[width=0.85\textwidth]{figs/abc_observables_mr_bin.pdf}
    \caption{\label{fig:demcloseup}
    $G-R$ (left) and $FUV-NUV$ (right) number density distributions for the DEM
    models of the SIMBA (orange), TNG (blue), and EAGLE (green) simulations in
    two $M_r$ bins: $-20 > M_r > -21$ (top panels) and $-21 > M_r > -22.5$
    (bottom panels).  Each of the DEM models are run using the median posterior
    parameter values. In comparison to SDSS (grey shaded), the DEM models predict 
    consistent red sequence and blue cloud positions in the $G-R$ distributions, 
    $FUV-NUV$ peak positions, and number density. {\em Overall the DEM
    models for SIMBA, TNG, and EAGLE produce observables are in good agreement 
    with SDSS.}
    }
\end{center}
\end{figure}



\begin{figure}
\begin{center}
    \includegraphics[width=0.85\textwidth]{figs/abc_attenuation.pdf}
    \caption{\label{fig:atten}
    }
\end{center}
\end{figure}

\section{Results} \label{sec:results}
We present the posterior distributions of the DEM parameters for the SIMBA
(orange), TNG (blue), and EAGLE (green) hydro simulations in
Figure~\ref{fig:abc}. The DEM parameters include $\mtaum$, $\mtaus$, and
$\ctau$, which parameterize the $M_*$ dependence, $\sfr$ dependence, and 
amplitude of $\tau_V$, the $V$-band optical depth. $\tau_V$ dictates the
overall strength of the dust attenuation. They also include $\mdeltam$,
$\mdeltas$, and $\cdelta$, which parameterize the $M_*$ dependence, $\sfr$ dependence,
and amplitude of $\delta$, the slope offset of the attenuation curve
(Section~\ref{sec:dem} and Table~\ref{tab:free_param}). The posteriors 
are derived using ABC (Section~\ref{sec:abc}) and the contours mark the 
$68\%$ and $95\%$ confidence intervals. 

In addition, we present the observables predicted by the DEM with median of the
posteriors for the SIMBA (orange), TNG (blue), and EAGLE (green) simulations 
in Figure~\ref{fig:dem}. We include the SDSS observables for comparison
(black dashed; Section~\ref{sec:obs}). The top panels present the $(G-R) - M_r$ 
color-magnitude relations while the bottom panels present the $(FUV-NUV) - M_r$
relations. Without any dust prescriptions, we found that simulations predict
dramatically different observables than SDSS (Figure~\ref{fig:obs}). 
\ch{something about clear bimodality with the red sequence and blue cloud} 
In contrast, {\em using DEMs we produce $(G-R) - M_r$ and $(FUV-NUV) - M_r$
relations consistent with SDSS for all of the simulations}. 

We examine the observables more closely in Figure~\ref{fig:demcloseup}, where
we present the $G-R$ (left) and $FUV-NUV$ (right) distributions in $M_r$ 
bins for the DEM models of the SIMBA (orange), TNG (blue), and EAGLE (green) 
simulations. The top panels contain galaxies with $-20 > M_r > -21$ and the 
bottom panels contain galaxies with $-21 > M_r > -22.5$. In each panel we 
include the SDSS distributions for comparison (grey shaded). These distributions are 
number densities. The positions of the red sequence and blue cloud in the $G-R$
distributions for the DEM models are consistent with the SDSS distribution. 
They also produce $FUV_NUV$ distributions with peaks consistent with
observations. We also find good agreement in the overall normalization
(number density), especially for the $-20 > M_r > -21$ bin. 

There are also a few discrepancies between the DEM models and SDSS observations. 
First, the red sequence in the observed $G-R$ distribution has a sharp red color 
cut-off.  In contrast, we do not find as sharp of a $G-R$ cut off in the DEM models, 
and find a small fraction of galaxies redder than observations. 
Furthermore, while the DEM models for TNG and EAGLE agree with SDSS, the DEM
model for SIMBA overpredicts blue galaxies. \ch{what do we want to do with
SIMBA?} We also find that the TNG and EAGLE DEM models slightly overpredict 
galaxies in the higher luminosity bin and produce a broader $G-R$ distribution.
Nonetheless, Figure~\ref{fig:demcloseup} demonstrates that overall the DEM
models for SIMBA, TNG, and EAGLE produce observables are in good agreement 
with SDSS.

% comparison to literature 
% First EAGLE
Previous works in the literature have also presented models that predict colors
and luminosities for different simulations and dust models and compared them to
observations. For instance, \cite{trayford2015} calculate colors and luminosities 
for $z=0.1$ galaxies using EAGLE with the {\sc GALAXEV} population synthesis models 
and a two-component screen model for dust. Compared to GAMA observations, their
model produces a bluer red sequence and overpredicts luminous blue galaxies~\citep[][Figure
5]{trayford2015}. Although a detailed comparison is difficult since they
compare all galaxies, not just centrals, we note that the DEM models find good
agreement in the positions of the red sequences. Furthermore, for TNG and EAGLE, 
the DEM models do not overpredict blue galaxies. Even for SIMBA, the DEM model
overpredict blue galaxies by a smaller amount.

More recently, \cite{trayford2017} calculated optical colors for the EAGLE simulation using
{\sc SKIRT}, a Monte Carlo radiative transfer code~\citep{camps2015}, to model the dust. 
\cite{trayford2017} compares all galaxies so again, we do not include a direct 
comparison. Compared to GAMA, while they find good agreement with observations 
at intermediate masses, $10^{10.5} < M_* < 10^{10.8} M_\odot$, they again find
a bluer red-sequence at $10^{11.2} < M_* < 10^{11.5}$. While we do not present
comparisons in $M_*$ bins, which compare SED derived $M_*$ to the predicted
$M_*$ from simulations, we find that the position of the red sequence in the
DEM models are in good agreement with SDSS even at $M_* > 10^{11.2}$. Our models, 
however, predict an overall broader color distribution at the high mass end. 
Using the same \cite{trayford2017} EAGLE and {\sc SKIRT} framework,
\cite{baes2019} compared the predicted cosmic spectral energy distributions
(CSED) to observations. While this comparison averages over the galaxy
populations, they find the EAGLE-{\sc SKIRT} CSED overestimates the observed
CSED in the UV regime. Moreover, the $FUV-NUV$ color of their CSED is
significantly higher than GAMA $FUV-NUV$. The DEM models on the other hand, 
predict $FUV-NUV$ in good agreement with observations. 

%\cite{baes2019}: EAGLE+SKIRT SED compparison with GAMA Far UV is not attenuated enough. underrestimates optical and NIR
% then TNG
Besides with EAGLE, \cite{nelson2018} calculated optical colors for the
TNG simulations with a dust model that includes attenuation due to dense gas 
birth clouds surrounding young stellar populations and also attenuation due to 
simulated distribution of neutral gas and metals.
Although they compare the color distribution for all galaxies in bins of $M_*$,
so we cannot directly compare to the DEM models, compared to SDSS they find 
bluer red sequence peak position and narrower blue cloud. We find neither of
these discrepancies between the DEM models and SDSS. 
\ch{restatement of how the DEM models have the flexibility to reproduce the
optical and UV color-magnitude relationship.}

The simulations with DEMs predict observables in agreement with observations 
despite the significant differents in the SMFs and $M_*$-SFR relations 
(Figure~\ref{fig:smf_m_sfr}). In other words, the DEM has the 
flexiblity to reproduce observations even when simulations predict galaxy
populations with significantly different physical properties. We emphasize that
the DEM is based on the standard prescriptions for dust attenuation and, thus,
serve as a flexible parameterization within the bounds of our current
understanding of dust in galaxies.

Figures~\ref{fig:dem} and~\ref{fig:demcloseup} highlights two key points. First, any comparison of
simulations must account for dust. Dust entirely changes the predictions of
simulations in observables-space. Without dust, we did not find bimodality in
the color-magnitude relation.
Fortunately, the DEM provides a simple framework
for including dust motivated by attenuation laws and correlations with the
physical properties of galaxies. 
Second, the current limiations in our understanding of dust in galaxies 
significant impedes our ability to understand galaxy formation from simulations. 
To robustly interpret any comparison of simulations, we would need to
marginalize over dust (\emph{e.g.} DEM parameters). Since DEMs can produce
consistent observables for a range of simulations, marginalizing over dust
would leave little constraining power on the subgrid prescriptions (\emph{i.e.}
galaxy physics) of the simulations. 

% what we learn about AV - galaxy property connection  
The DEMs demonstrate that simulations can closely match observations by varying
dust attenuation. They therefore illustrate how dust is a major bottleneck for
directly interpreting galaxy simulations for insights into galaxy formation. In
addition, DEMs also provide some insight into dust. Given our parameterization
(Section~\ref{sec:dem}), it is especially easy to interpret correlation between
dust attenuation and galaxy physical properties. For instance, the posteriors of
DEM parameters in Figure~\ref{fig:abc} reveal a
number of consistent trends across the simulations. In all three simulations, we find significant positive
$M_*$ dependence of $\tau_V$: $\mtaum \sim 2$. Regardless of the underlying
hydro simulations, we find that {\em galaxies with higher $M_*$ have overall higher 
dust attenuation}.

This $M_*$ dependence is consistent with prevoius works in the literature.
The seminal work of \cite{burgarella2005}, for instance, found significant
positive $M_*$ dependence in $FUV$ attenuation in NUV-selected and FIR-selected
samples of \cite{buat2005} and \cite{iglesias-paramo2006}. \cite{garn2010} also
find positive $M_*$ dependence in Balmer decrement-based H$\alpha$ attenuation
in ${\sim}90,000$ SDSS star-forming galaxies. \cite{battisti2016} similarly
find higher Balmer optical depth for higher $M_*$ in ${\sim}10,000$
star-forming galaxies from GALEX and SDSS. Most recently, \cite{salim2018} 
find higher $V$ and $FUV$ attenuation for more masssive star-forming galaxies in the
GALEX-SDSS-WISE Legacy Catalog 2 (GSWLC2). \todo{citation is a bit SDSS heavy.
Anything else in the literature?}

%\ch{SFR dependence of attenuation curves} 
In addition to the $M_*$ dependence, the DEM posteriors also allow us to
examine the correlation bteween dust attenuation and star formation. For TNG
and EAGLE, we infer DEM posteriors with $\mtaus\sim-1$ (Figure~\ref{fig:abc}). 
This means that TNG and EAGLE require higher attenuation for galaxies with
lower SFR --- \ie quiescent galaxies have higher dust attenuation overall.
While previous works that have examined the relationship between dust attenuation
and SFR in observations~\citep[\eg][]{garn2010, reddy2015, battisti2016,
battisti2017}, they focus solely on star-forming galaxies. While they find that
star-forming galaxies with higher SFR have higher attenuation, much of this
trend is driven by the star-forming sequence~\citep[more massive star-forming
galaxies have higher SFR;][]{garn2010, battisti2017}. At fixed $M_*$,
observations find no strong SFR dependence for the SF population. 

For SIMBA, unlike for TNG and EAGLE, we infer $\mtaus\sim 3$: galaxies with
higher SFR have higher dust attenuation (Figure~\ref{fig:abc}). In fact, the
attenuation is so high for star-forming galaxies that they populate the red
sequence rather than the blue cloud in the color-magnitude relation. This
extreme SFR dependence in the dust attenuation that results in a contradiction of
established color-SFR relations, is due to the fact that SIMBA predicts a population of
star-forming with exceptionally high SFR, that seemingly lie above the SFS
(Figure~\ref{fig:smf_msfr}). In a SIMBA DEM model with $\mtaus < 0$, these high
SFR galaxies would be high luminosity blue galaxies, not found in observations. 
\ch{If we impose a $\mtaus < 0$ prior for the SIMBA DEM model, we struggle to
reproduce observables consistent with SDSS.} The difference in $\mtaus$ 
among the hydro simulations demonstrates that, in addition to insights on dust
in galaxies, our DEM approach can also highlights differences and limiations among 
simulations. Moreover, it further highlights that dust attenuation can be
adjusted, within priors set by observations, so that any simulation can 
match observations. 

%For instance, \cite{garn2010} find that SDSS star-forming galaxies with higher
%SFR have higher H$\alpha$ attenuation. \cite{battisti2016} find a consistent
%correlation for the Balmer optical depth of star-forming galaxies in GALEX and SDSS 
%using 10000 SF galaxies GALEX-SDSS. Although at higher redshifts,  
%\cite{reddy2015} also find this correlation among $z{\sim}2$ star-forming galaxies of
%MOSFIRE Deep Evolution Field.

%\cite{battisti2017} using 5000 SF galaxies from found little stellar mass dependence in the opposite direction (less attenuation at higher stellar masses). But they have big error bars and only probe up to 9.7
%\cite{reddy2015} SF galaxies from $z\sim2$ MOSFIRE Deep Evolution Field survey find strong correlation with SFR. %ionized gas is more reddened relative to the stellar continuum with increasing SFR 
\ch{delta - galaxy property connection}
In addition to the dependence of $A_V$ on $M_*$ and SFR, our results also shed light 
on the correlation between the slope of dust attenuation, parameterized by
slope deviation $\delta$, and galaxy properties. Overall, 

We also find overall little $M_*$ and ${\rm SFR}$ dependence in $\delta$. In
fact, the amplitude of $\delta$ is roughly consistent with 0.  
This is consistent with \cite{salim2020}, where they measured the attenuation
curve slopes of $23,000$ galaxies from GALEX-SDSS-WISE Legacy Catalog 2
(\ch{cite}). 
comparison to \cite{leja2017} (only 129 nearby galaxies though) 
\cite{viaene2017} (one galaxy, jesus)  
%salim2018: d = -0.38 + 0.29(log M* - 10),
\cite{kriek2013}  using stacked SEDs of medium- and broadband photometry of
galaxies at 0.5 < z < 2 find an average slope of delta=-0.2 . % but they
%restrict to galaxies with moderate to high optical attenuations (AV> 0.5), 

% Salim calims that the slope is steeper for satr burst galaxies 


% In brief, at low optical depths, red light is scattered more isotropically and escapes the galaxy, while blue light is more forward- scattered and is subjected to more absorption. This results in a net steepening of the attenuation curve. At high optical depths, in a mixed star-dust geometry, the observed radiation primarily comes from stars at an optical depth less than unity. Redder photons travel more physical distance than bluer photons before absorption or scattering; the net effect is that larger optical depths result in a grayer attenuation curve.
% Do we agree with the standard interpretion of dust? 

%\cite{trcka2020}: EAGLE+SKIRT with CIGALE to get physical properties of
%galaxies. trcka2020 compares to dustpedia~\citep{davies2017} They lok at
%IRX-beta relation.

% parameter degeneracies? 
\ch{can we say anything about the $A_V - \delta$ relation?}
% Both Leja et al. (2017) and Salim et al. (2018) find a strong relationship between the optical opacity and curve slope, such that the galaxies with higher AV have shallower slopes.

% \cite{salim2018} find that Galaxies with low AV values having steep curves and the ones with high AV being shallow.
% At fixed AV there is no trend of slope versus mass
% strong correlation between the attenuation curve slope and optical depth ( ctau -- cdelta) 
% important implications, and it also impacts the underlying assumption of the comparison method,
% Chevallard et al. (2013) (radiative transfer models combined with realistic
% dust geometries and the two-component (birth clouds/diffuse ISM) model
% (Charlot & Fall 2000))
% Leja et al. (2017) also finds this correlation 

% according to Chevallard et al. (2013) the steepness of an attenuation curve
% at small optical depths is the result of the dominance of scattering over
% absorption, coupled with the fact that scattering is more forward directed at
% shorter wavelengths whereas it is more isotropic at longer wavelengths.As the
% optical depth increases, absorption becomes more dominant than scattering,
% and the curve becomes shallower (grayer).


% variation of attenuation curve  
\ch{variation of attenuation curves}
While an empirical prescription like DEM doesn't allow explicit modeling of the
complex dust-star geometry, it does a good job at mimicking it. 
\ch{how does this compare to the literature?} 
comparison to \cite{narayanan2018} paper 
comparison to \cite{salim2018} for SF population  
comparison to \cite{leja2017} (only 129 nearby galaxies though) 



dust attenuation in quiescent galaxies, which is difficult to measure from
observations. For non-star-forming galaxies, MIR emission from active galactic nuclei (AGN)
heating nearby dust complicate methods that rely on IR luminosity to measure
dust attenuation. Even SED fitting methods require accounting for AGN MIR 
emission~\citep{salim2016, leja2018, salim2018}. 
Since we utilize a forward modeling approach with optical and UV data, we don't
have this issue. 
\ch{what do we learn about quiescent galaxy attenuation?} 

%\cite{salim2018}: The IR luminosities of AGN may be affected by nonstellar dust heating, especially since they are based on the mid-IR data.

%\cite{leja2017} Inferring total IR luminosities with only MIR broadband photometry can be strongly affected by galaxy-to- galaxy variability in polycyclic aromatic hydrocarbon (PAH) emission in the MIR (Draine et al. 2007) and by an active galactic nucleus (AGN) contribution (Kirkpatrick et al. 2015). The interpretation of LIR itself is complicated by the contribution of evolved stars to the IR luminosity (Cortese et al. 2008; Hayward et al. 2014; Utomo et al. 2014). UV luminosities are sensitive to stellar metallicity and recent SFH, and very sensitive to the amount of dust attenuation, which can be estimated from the UV slope β but with serious limitations (Viaene et al. 2016).
%\cite{leja2018}:  Full SED models will naively attribute AGN emission to dust heated by star formation, resulting in full SED models overestimating sSFRs by up to 0.6 dex for galaxies with AGN (Salim et al. 2016)

% For comparison methods, IR lumninosity from AGN can affect IR based attenuation curves~\citep{cite}. Even for SED
%\cite{salim2018} does look at attenuatiion for quiesscent 


\ch{limitations of DEM and potential improvements}
\begin{itemize}
    \item too many luminous galaxies
    \item color distribution isnt' perfect. 
    \item There isn't a whole lot of flexibility for SFR=0 galaxies predicted by
    simulations and they do not agree well with observations \ref{sec:res}. 
\end{itemize}


\ch{How robust are our results?}
We fix the UV bump to reduce the number of parameters. But when run our
analysis without fixing the UV bump, we find it does not impact our results.
We also get no constraints on the UV bump parameters. 

We rely on the slab model. But nothing changes when we use a more flexible
truncated normal distribution in Appendix~\ref{sec:nonslab}. tnorm DEM model allows us to also vary the
scatter of the attenuation curve 

\ch{How about our prior choice?}

\ch{restate what we learn about dust through DEMs}  
paragraph on restating how we can learn about dust through DEMs based on trends we see
across all simulations. summarize main findings again. 

\ch{If we marginalize over dust, can we learn anything about galaxy evolution
from the simulation?}

Are there observables that hydro sims + DEMs cannot reproduce? What does that say about the hydro sims?

What observables are unaffected by DEMs? We should chase those observables. 

We clearly have to becareful with overinterpreting hydro sims because modifying
dust allows us to reproduce whatever we want. 
\begin{itemize}
    \item Should we bother calibrating our empirical and semi-analytic models
        to hydrodynamic simulations when the hydro sims also require
        marginalizing over dust parameters? Does this mean that if our goal is
        to make realistic mocks, we can be relatively careless about 
\end{itemize}


\ch{What are some applications for DEMs?}
Realistic mock catalogs that reproduce observations in observable-space rather
than physical parameter space.  
