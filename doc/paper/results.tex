\begin{figure}
\begin{center}
    \includegraphics[width=0.9\textwidth]{figs/abc_observables.pdf}
    \caption{\label{fig:dem}
    The optical and UV color-magnitude relations predicted by the DEM with 
    the median ABC posteriors for the SIMBA (orange), TNG (blue), and EAGLE
    (green) hydrodynamical simulations. For comparison, we include the 
    $(\gr) - M_r$ (top panels) and $(\fnuv) - M_r$ (bottom panels) relations
    for SDSS (black dashed). With the DEM, the simulations produce dramatically 
    different observables than without any dust prescription (Figure~\ref{fig:obs}). 
    Hence, dust must be account for when interpreting and comparing simulations.
    Moreover, with the DEMs, all three simulations produce color-magnitude
    relations consistent with SDSS. Since different simulations can 
    produce reproduce observations by varying dust, dust significantly limits
    our ability to constrain the physical processes that go into galaxy
    simulations. 
    }
\end{center}
\end{figure}

\section{Results} \label{sec:results}
% dust model can reproduce color magnitude relation
Without dust attenuation, all of the hydrodynamical simulations struggle to 
reproduce the $(\gr) - M_r$ and $(\fnuv) - M_r$ relations of SDSS (Figure~\ref{fig:obs}). 
In optical, the simulations predict significantly bluer colors and find  
broader differences in color, ${\sim}0.5~mag$, between star-forming and
quiescent galaxies. In the UV, they predict quiescent galaxies with $\ssfr <
10^{-12}yr^{-1}$ that have redder $\fnuv$ colors beyond SDSS observations. 
Meanwhile, for the rest of the galaxies, the simulations predict significantly
bluer UV colors than SDSS. 

\emph{With the \eda~model, all of the simulations are able to
reproduce the color-magnitude relations of SDSS.} In Figure~\ref{fig:dem}, 
we present the optical and UV color-magnitude relations predicted by the 
\eda~for the SIMBA (orange), TNG (blue), and EAGLE (green) simulations. 
For the \eda~parameters, we use the median of the posterior distributions 
inferred using ABC (Figure~\ref{fig:abc}). The contours mark the $68\%$ and 
$95\%$ confidence intervals of the distributions.

%By comparing the color-magnitude relations produced with and without the \eda, we can examine how dust attenuation can produce observations. 

Dust dramatically impacts the observables of simulations. The \eda~affects the 
the optical and UV color-magnitude relations in three major ways to produce
good agreement with SDSS. First, the \eda~significantly reddens simulated galaxies in both 
the optical and UV.  Overall, $\gr$ colors are ${\sim}0.25~mag$ redder and 
$\fnuv$ colors ${\sim}0.5~mag$ redder. Second, the \eda~significantly attenuates
(${\sim}0.5~mag$) intrinsically blue galaxies (\ie~star-forming galaxies with
$\log~{\rm SSFR} > -10.5$). As a result, there are no longer luminous optically blue 
galaxies ($M_r < -21$ and $\gr < 0.5$) --- consistent with observations.
Lastly, in the UV, the \eda~attenuates low SSFR quiescent galaxies that are 
intrinsically red in the UV ($\fnuv{\sim}3~mag$) by ${\sim}1~mag$. Thus, unlike in
Figure~\ref{fig:obs}, the UV color-magnitude relations from the \eda~do not
have galaxies with high $\fnuv$ as in SDSS.

% Do we want to comment on the variation in the eda attenuation curve and how
% that mixes up the color for a given SSFR? 

% paragraph on SIMBA discrepancy
For SIMBA, the \eda~substantially improves the agreement with observations;
however, there are still some significant discrepancies in the color-magnitude
relations with SDSS. The \eda~products a $\gr$ color distribution that is too
narrow and a $\fnvu$ distribution that is too wide. Also, the inferred \eda~parameters 
for SIMBA differ significantly from the parameters of TNG and EAGLE
(Figure~\ref{fig:abc}). These discrepancies are primarily driven by the
large population of high $\log{\rm SSFR} > -9.5$ ``starburst'' galaxies that
lie above the SFS, which are {\em only} present in SIMBA (Figure~\ref{fig:smf_msfr}). 
Without dust attenuation, these starburst galaxies are blue $\gr\sim0.1$ and
have high luminosity, $M_r < -22$. Such high luminosity blue galaxies are not
present in observations, so these SIMBA galaxies need to be both strongly 
attenuated and reddened. This would, however, contradict the attenuation-slope 
relation established in both observations and simulations where galaxies with 
higher attenuation have shallower
slopes~\citep{inoue2005,chevallard2013,salim2018,salim2020,trayford2020}.
To further examine the impact of the starburst galaxies in SIMBA, we ran the \eda~for SIMBA excluding galaxies with $\log{\rm SSFR} > -9.5$ and 
with median parameter values of the TNG and EAGLE posteriors. Without its starburst
galaxies, SIMBA produces color-magnitude relations in good agreement 
with observations. {\em We conclude that the excess starburst population in 
SIMBA is in tension with observations}. Examining the prescriptions that
produce the starburst in SIMBA is beyond the scope of this paper. Therefore,
for the rest of the paper, we focus on the TNG and EAGLE simulations.  

% comparison to literature 
Previous works in the literature have also compared simulations with different
dust prescriptions to observations in color-magnitude space. For EAGLE, 
\cite{trayford2015} calculate colors and luminosities with the {\sc Galaxev}
population synthesis models and a two-component screen model for dust. More
recently, \cite{trayford2017} calculated optical colors for EAGLE using {\sc
Skirt}, a Monte Carlo radiative transfer code~\citep{camps2015}, to model the
dust. At stellar masses and luminosities comparable to our SDSS sample, both 
\cite{trayford2015} and \cite{trayford2017} produce red sequences bluer than 
in GAMA observations. Also, \cite{trayford2015} predict an excess of luminous 
blue galaxies. Although a detailed comparison is difficult since both works 
compare to different observations, we note that with the \eda, EAGLE is able 
to successfully reproduce the position of the SDSS red sequence and does not 
predict a significant excess of luminous blue galaxies. Also using EAGLE and 
{\sc Skirt}, \cite{baes2019} find that they overesimtate the observed cosmic 
SED (CSED) in the UV regime and produce significantly higher $\fnuv$ color 
than GAMA. The \eda~predicts $\fnuv$ in good agreement with SDSS. 
%\cite{baes2019}: EAGLE+SKIRT SED compparison with GAMA Far UV is not attenuated enough. underrestimates optical and NIR
For TNG, \cite{nelson2018} calculate optical colors using a dust model that
includes attenuation due to dense gas birth clouds surrounding young stellar
populations and also due to simulated distribution of neutral gas and metals.
They find bluer red sequence peaks and a narrower blue cloud compared to SDSS.
We find neither of these discrepancies for the TNG with the~\eda. The
\eda~is a simpler prescription for applying dust attenuation than the dust
models in these works. Yet, with its flexibility, we are able to produce
optical and UV color-magnitude relations that are in good agreement with
observations. Furthermore, with its low computation costs we were able o fully
explore our dust parameters. 
%better agreement than observations than these works.

% without being fit, the EDA reproduces the attenuation-slope relation and SF
% attenuation of SDSS observations  
\begin{figure}
\begin{center}
    %\includegraphics[width=0.9\textwidth]{figs/abc_slope_AV_starforming.pdf}
    \caption{\label{fig:slope}
    The attenuation-slope relation of star-forming galaxies ($\ssfr >
    10^{-11}yr^{-1}$), using the attenuation curves predicted by our  
    \eda~prescription for the median posterior parameter values of SIMBA
    (left), TNG (center) and EAGLE (right). 
    For comparison, we include the observed attenuation-slope relation
    from GSWLC2~\citep{salim2020}. 
    We use $A_V$ and $S = A(1500\AA)/A_V$ as measurements of attenuation and
    slope, respectively. 
    \chedit{
        \emph{The \eda~does not predict $A_V < 0.3$ because star-forming galaxies in
        the simulations are too lumnious and require significant attenuation to
        reproduce observations.}
        Beyond $A_V > 0.3$, however, there is good agreement between the
        attenuation-slope relation predicted by the \eda~and observations. 
    }
    }
\end{center}
\end{figure}

\subsection{Comparison to Dust Observations} \label{sec:reproduce}
With our \eda~prescription, we are able to accurately reproduce the
observed optical and UV color-magnitude relations with our simulations. 
In addition to  reproducing observations, since the \eda~assigns dust
attenuation curves to each simulated galaxy, we can compare the
\eda~attenuation curves to dust attenuation measured from
observations. 
We begin with the well-established attenuation-slope relation: star-forming
galaxies with higher dust attenuation have shallower attenuation curves. 
This relation is a consequence of dust scattering dominating absorption at
low attenuation while dust absorption dominates at high
attenuation~\citep{gordon1994, witt2000, draine2003, chevallard2013}. 
In Figure~\ref{fig:slope}, we present the attenuation-slope relation of
star-forming galaxies with $\ssfr > 10^{-11}yr^{-1}$ based on the
dust attenuation curves predicted by the \eda~for the median posteriors of
SIMBA (left), TNG (center) and EAGLE (right).
For comparison, we include the observed attenuation-slope relations of
GSWLC2 galaxies~\citep[grey shaded;][]{salim2020}.
For attenuation we use $A_V$ and for slope we use the UV-optical slope, $S
= A(1500\AA)/A_V$, commonly found in the literature. 
The contours mark the 68 and 95 percentiles. 
Most noticably, we find that the \eda~does not predict $A_V < 0.3$. 
\chedit{
    This is not due to the selection function imposed by our forward model. 
    The attenuation-slope relation of star-forming galaxies in GSWLC2 does not
    change significantly if we impose similar selection cuts as our
    observational sample.
    Instead, the lack of star-forming galaxies with $A_V < 0.3$ is a
    consequence of SIMBA, TNG, and EAGLE predicting star-forming galaxies that
    are more luminous than observations.  
}
All of the simulations have star-forming galaxies with intrinsic $M_r <
-21$ and $\gr < 0.5$ (Figure~\ref{fig:obs}). 
This is further corroborated by the $\sfr-M*$ relations in
Figure~\ref{fig:smf_msfr}, where the simulations all have star-forming
galaxies with $M_* > 10^{11}M_\odot$, not found in SDSS. 
To reproduce the SDSS optical color-magnitude relation these galaxies would
need to be significantly reddened and attenuated. 
Hence, any dust prescription for the simulations would require high 
$A_V$ for star-forming galaxies.
Nevertheless, for $A_V > 0.3$, we find good agreement between the 
attenuation-slope relation predicted by the \eda~and observations. 

%We note that the difference in the $A_V$ ranges is due to the $M_r$
%completeness limit imposed by our forward model (Section~\ref{sec:fm}).
%The GSWLC2 sample in \cite{salim2020} extends down to $M_* \sim
%10^{8.5}M_\odot$; however, the TNG and EAGLE samples do not extend below
%$M_* \sim10^{10}M_\odot$.
%\emph{The \eda~predicts attenuation-slope relations for TNG and EAGLE that
%are in excellent agreement with observations.} 

\begin{figure}
\begin{center}
    \includegraphics[width=0.5\textwidth]{figs/abc_sf_attenuation.pdf}
    \caption{\label{fig:sfatten}
    The normalized attenuation curves of star-forming galaxies predicted by
    the \eda~for median posterior parameter values of SIMBA (orange), TNG
    (blue), and EAGLE (green).  
    Galaxies with $\log \ssfr > -11~yr^{-1}$ are classified as star-forming. 
    The attenuation curves are normalized at $3000\AA$ and we mark the
    68 percentile of the attenuation curves with the shaded region.
    For comparison, we include $A(\lambda)/A(3000\AA)$ measurements from
    the~\cite{narayanan2018} radiative transfer simulation (dashed) and
    \cite{salim2018} observations (dotted).
    %The \cite{calzetti2000} and \cite{battisti2017} attenuation curves are shallower than the \eda~attenuation curves; however, they probe lower $M_*$ galaxies than our forward modeled TNG and EAGLE samples.  For attenuation curve from \cite{salim2018}, which probe a similar $M_*$ range, we find goood agreement. 
    {\em The \eda~predict attenuation curves of star-forming galaxies that
    are in good agreement with the attenuation curves measured from
    the simulation and observations in the literature.}
    %We also find good agreement with median attenuation curve of star-forming galaxies in the radiative transfer simulations of \cite{narayanan2018}.
    }
\end{center}
\end{figure}

In addition to the attenuation-slope relation, we can also directly compare
the attenuation curves predicted by the \eda~to measurements from
observations for star-forming galaxies. 
In Figure~\ref{fig:sfatten}, we present the normalized attenuation curves
of star-forming galaxies predicted by the \eda~for the median posterior
parameter values of SIMBA(orange), TNG (blue), and EAGLE (green).
We again define galaxies with $\ssfr > 10^{-11}{yr}^{-1}$ as star-forming.
The attenuation curves are normalized at $3000\AA$ and we present the
variation in the attenuation curves in the shaded region, 68 percentile. 
For comparison, we include $A(\lambda)/A(3000\AA)$ from the
\cite{narayanan2018} radiative transfer simulation (dashed) and 
observations~\citep[][dotted]{salim2018}. 
The attenuation curve from \cite{salim2018} corresponds to star-forming
galaxies with $M_* > 10^{10.5}M_\odot$, a similar $M_*$ range as our
forward modeled samples. 
Since we do not vary the UV bump in our \eda~prescription, we ignore any
discrepancies in the amplitudes of the bump. 
\emph{Overall, we find good agreement between the \eda~attenuation curves for
star-forming galaxies and the attenuation curves from observations and
simulations in the literature.}

%Again, the fact that we reproduce the detailed dust attenuation curves of star-forming galaxies in observations and simulations with the \eda~without fitting for them, highlights the advantages of a forward modeling approach. 

%The \eda~attenuation curves are slightly steeper than the \cite{calzetti2000} and \cite{battisti2017} curves. 
%These attenuation curves, however, are derived from $M_* < 10^{9.9}M_\odot$
%star-forming galaxies, which lie below the $M_*$ limit of our forward
%modeled TNG and EAGLE samples. 
%Meanwhile, the TNG and EAGLE \eda~attenuation curves are in good agreement
%with the \cite{salim2018} attenuation curve for $M_* > 10^{10.5}M_\odot$ star-forming galaxies. 
%They are also consistent with the median curve of \cite{narayanan2018}. 

%The \eda~attenuation curves are noticeably steeper than the \cite{calzetti2000} and \cite{battisti2017} curves.  These attenuation curves, however, are derived from $M_* < 10^{9.9}M_\odot$ star-forming galaxies --- below our $M_*$ range. 
%Since we find $\mdeltam < 0$ for both the TNG and EAGLE posteriors, the \eda~attenuation curves are consistent with \cite{calzetti2000} and \cite{battisti2017}. 


%\chedit{ 
%    The \eda~predicts higher dust attenuation at lower wavelenghts for
%    star-forming galaxies.
%    Without dust attenuation, both TNG and EAGLE predict star-forming galaxies
%    that are bluer in the optical and UV than observations
%    (Figure~\ref{fig:obs}).
%    To reproduce the SDSS, the \eda~significantly reddens star-forming galaxies.
%}
%In Figure~\ref{fig:raw_atten}, we also find that more massive star-forming
%galaxies have higher attenuation. This is because the simulations overpredict 
%luminous blue star-forming galaxies, which must be attenuated to reproduce
%observations. 


%At low attenuation, dust scattering dominates absoprtion so the 
%attenuation curve steepens because red light scatters isotropically while blue light
%scatters forward~\citep{gordon1994, witt2000, draine2003}. %, which causes more optical-to-IR light to escape the galaxy than UV light
%At high attenuation dust absorption is dominant and the attenuation curve is
%shallower~\citep{chevallard2013}. For the $A_V$ range probed by the DEM, the
%$A_V$--slope relation is in good agreement with GSWLC2 galaxies~\citep[black shaded][]{salim2020}.
%They are also consistent with \cite{leja2017}. We also compare our results to
%theoretical predictions from radiative transfer models, \cite{inoue2005}
%(dotted), the radiative transfer models considered in \cite{chevallard2013}
%(dot dashed), and \cite{trayford2020} (light shaded), which all predict shallower 
%attenuation curves than observations. This is also the case for the
%\cite{narayanan2018} attenuation curves (not included). 
%\emph{The attenuation curve slopes from the DEM for are in excellent
%agreement with observations and better reproduces the observed
%attenuation--slope relation than radiative transfer models.}
 

\begin{figure}
\begin{center}
    \includegraphics[width=0.85\textwidth]{figs/abc_attenuation_unormalized.pdf}
    \caption{\label{fig:raw_atten}
    Same as Figure~\ref{fig:atten} but without normalizing the attenuation
    curves at 3000$\AA$. Based on \eda, quiescent galaxies have significant 
    dust attenuation. Compared to star-forming galaxies, they have 
    significantly lower attenuation in the UV and significantly shallower 
    attenuation curves overall.
    }
\end{center}
\end{figure}
\subsection{The Attenuation Curves of Quiescent Galaxies}  
We have demonstrated so far that the \eda~is able reproduce the observed UV and
optical color-magnitude relations and also predict dust attenuation curves
of star-forming galaxies consistent with observations and radiative transfer 
simulations. In addition, the \eda~also predicts dust attenuation curves of
quiescent galaxies. This is particularly valuable since there are many challenges 
to measuring attenuation curves for quiescent galaxies directly from observations. 
Methods that rely on IR luminosities can be contaminated by MIR emission from AGN
heating nearby dust~\cite{kirkpatrick2015}. Even SED fitting methods require 
accounting for AGN MIR emission~\citep{salim2016, leja2018, salim2018}. SED 
fitting methods also struggle to tightly constrain dust attenuation for quiescent 
galaxies since they are limited by the degeneracies with star formation history and 
metallicity.

With a forward modeling approach, we circumvent these challenges. We derive the 
attenuation curves necessary for quiescent galaxy population in simulations to
reproduce the observed optical and UV photometry.  In right panels of
Figure~\ref{fig:atten}, we present the attenuation curves of quiescent galaxies 
predicted by the \eda~model for the median posterior parameter values of TNG (blue) 
and EAGLE (green). In the top and bottom panels, we present galaxies with 
$\log M_* < 10^{10.5} M_\odot$ and $\log M_* > 10^{10.5} M_\odot$, respectively. 
We define galaxies with $\log {\rm SSFR} < -11$ as quiescent. The right panels 
of Figure~\ref{fig:raw_atten} are the same as in Figure~\ref{fig:atten}, except
the attenuation curves in Figure~\ref{fig:raw_atten} are not normalized. 

For both TNG and EAGLE posteriors, the \eda~predicts significant dust attenuation 
in quiescent galaxies. Compared to star-forming galaxies, however, they have 
lower attenuation in the UV and much shallower attenuation curves. The amplitude 
of the \eda~quiescent galaxy attenuation curves is driven by the fact that both
TNG and EAGLE --- without dust --- predict quiescent galaxies that are too 
luminous compared to observations. Hence, significant attenuation is necessary 
to lower their luminosity. Meanwhile, the shallow slope is driven by the
simulations predicting quiescent galaxies that are bluer in the optical but
redder in the UV than observations. The \eda~optically reddens the quiescent
galaxies but maintains a shallow enough slope to reproduce the UV
color-magnitude relation. This is also why TNG has a shallower slope than
EAGLE: TNG has an optically redder quiescent population and more quiescent
galaxies with high $\fnuv$ color. 

Given the challenges in observationally measuring attenuation curves of quiescent
galaxies, the predictions of the bestfit \eda~models for TNG and EAGLE
highlight the advantages of a forward modeling approach and provide valuable 
insights into dust attenuation in quiescent galaxies. \emph{Quiescent galaxies
have significant UV and optical attenuation with shallow attenuation curves.} 

 


\begin{figure}
\begin{center}
    \includegraphics[width=0.9\textwidth]{figs/abc_av_mssfr.pdf}
    \caption{\label{fig:avmsfr}
    \chedit{ 
        $M_*$ and $\ssfr$ dependence of dust attenuation at $1500 \AA$, $A_{1500}$,
        and at $5500\AA$, $A_{V}$.
    }
    }
\end{center}
\end{figure}

\subsection{The Galaxy -- Dust Connection}  
\chedit{
    With the \eda~framework, we can also shed light on the connection between
    the physical properties of galaxies and dust attenuations.
    In our \eda~prescription, we included a flexible $M_*$ and $\ssfr$ dependence in both the
    amplitude and slope of the attenuation curve~(Eqs~\ref{eq:tauv}
    and~\ref{eq:delta}. 
    Hence, we can reveal the $M_*$ and $\ssfr$ dependence of dust attenuation
    through the \eda~parameter constraints (Figure~\ref{fig:abc}) and the
    predicted attenuation curves. 
}


\chedit{
    Focusing first on the amplitude of dust attenuation, we find that TNG has
    little $M_*$ dependence in $\tau_V$: 
    $\mtaum = 0.14\substack{+0.64 \\ -0.58}$. 
    EAGLE has a more significant positive $M_*$ dependence:
    $\mtaum = 0.53\substack{+0.36 \\ -0.36}$.
    Though neither TNG nor EAGLE has a strong dependence, $V$-band dust
    attenuation is higher for more massive galaxies.  
    Meanwhile, we find significant $\ssfr$ dependences in both TNG 
    ($\mtaus = -0.42\substack{+0.2 \\ -0.18}$)
    and EAGLE
    ($\mtaus = -0.24\substack{+0.22 \\ -0.19}$): galaxies with higher $\ssfr$
    have lower $V$-band dust attenuation. 
    For the slope of the dust attenuation, we find significant $M_*$
    dependence in both TNG 
    ($\mdeltam = -0.36\substack{+0.23\\-0.19}$)
    and EAGLE
    ($\mdeltam = -0.2\substack{+0.16\\-0.16}$). 
    More massive galaxies have steeper attenuation curves. 
    We also find strong $\ssfr$ dependence in both TNG 
    ($\mdeltas = -0.55\substack{+0.08 \\ -0.08}$)
    and EAGLE
    ($\mdeltas = -0.43\substack{+0.08 \\ -0.08}$).
    Galaxies with higher $\ssfr$ have steeper attenuation curves.

}

\chedit{
    Next, we take a closer look at the $M_*$ and $\ssfr$ dependence of the
    attenuation curve in Figure~\ref{fig:avmsfr}. We present dust attenuation
    at $1500\AA$ ($A_{1500}$; top) and $5500\AA$ ($A_V$; bottom) as a function
    of $\log M_*$ and $\log {\rm SFR}$ predicted by the \eda~for TNG (left) and
    EAGLE (right). For each hexagonal bin, the colormap represents the median
    attenuation for all simulated galaxies in the bin. We only include
    $A_{1500}$ or $A_V$ values for galaxies that satisfy our $M_r < -20$
    completeness limit. The bottom panels of Figure~\ref{fig:avmsfr}
    corroborate our conclusions on $V$-band dust attenuation from the
    \eda~parameter posteriors (Figure~\ref{fig:abc}): $A_V$ has a slight $M_*$
    dependence but a significant $\ssfr$ dependence. 
    Furthermore, by comparing the top to the bottom panels, we can confirm the
    $M_*$ and $\ssfr$ dependence of the  attenuation curve slopes. 
}


The $M_*$ dependence in $\tau_V$ and $\delta$
are further illustrated in the attenuation curves in Figure~\ref{fig:raw_atten}.
For TNG, there is little difference in the $V$-band attenuation between the 
top and bottom panels. However, more massive galaxies have steeper slopes 
with similar $A_V$ so they have significantly higher UV attenuation. For
EAGLE, more massive galaxies have both higher $V$-band and UV attenuation.
Overall, we find that more massive galaxies have higher attenuation,
which is consistent with the literature. \cite{burgarella2005}, for instance,
found significant positive $M_*$ dependence in $FUV$ attenuation in
NUV-selected and FIR-selected samples. \cite{garn2010} and \cite{battisti2016}
also find higher attenuation in more massive SDSS star-forming galaxies. Most
recently, \cite{salim2018} find higher $V$ and $FUV$ attenuation for more
masssive star-forming galaxies in GSWLC2. 

From the attenuation curves in
Figure~\ref{fig:raw_atten}, we similarly find that star-forming galaxies have
attenuation curves with slightly lower $A_V$ but substantially higher UV 
attenuation. Quiescent galaxies, on the other hand, have significantly 
shallower attenuation curves. Although observations have examined the
SSFR dependence of dust attenuation, they cannot be compared to our findings since
they focus only on star-forming galaxies, due to the difficulty in
observationally constraining the attenuation curve in quiescent
galaxies~\citep[\eg][]{garn2010, reddy2015, battisti2016, battisti2017, salim2018}. 
In summary, from the bestfit \eda~models of TNG and EAGLE, we find that
\emph{galaxies with higher $M_*$ have overall higher dust attenuation and
galaxies with higher SSFR have steeper attenuation curves}.


\subsection{Discussion}  
We make a number of assumptions and choices in our \eda~prescription. 
\ch{
    First, we use the slab model (Eq.~\ref{eq:slab}) to assign $A_V$ as a
    function of the randomly sampled $i$ and $\tau_V$. 
    This choice is based on the fact that the slab model reproduces the 
    correlation between attenuations and inclination found in 
    observations~\citep{conroy2010b, wild2011, battisti2017, salim2020} as well
    as simulations~\citep[\eg][]{chevallard2013, narayanan2018, trayford2020}.
}
It can also reproduce the SDSS $A_V$ distribution (Figure~\ref{fig:av_dist}). If
we replace the slab model with a more flexible model for sampling $A_V$ using
truncated normal distributions, we find that our results are not significantly
impacted (see Appendix~\ref{sec:slab} for details). Therefore, we conclude
that our results do not sigificantly depend on our choice of the slab model. 
\ch{
    In our \eda, we also use a parameterization of $\tau_V$ and $\delta$ that
    depend linearly on $\log M_*$ and $\log {\rm SSFR}$. 
    While the $M_*$ and $\ssfr$ dependence of $A_V$ is well-motivated and is
    found in, for instance, the \cite{salim2018} GSWLC2 catalog (Appendix~\ref{sec:slab}), 
    the linear dependence was chosen primarily for its simplicity.
    The \eda~framework can be easily extended to more flexible
    parameterizations. 
}
In fact, a more flexible parameterization would likely reduce 
some of the discrepancies with the SDSS color-magnitude relations. The
\eda~produces broader distributions of optical colors than SDSS. Few galaxies
in SDSS have $\gr > 1.$ while some galaxies in the \eda~broadly extend beyond
this cut-off. In the UV, the \eda~struggles to accurately reproduce the redder
portions ($\fnuv > 1.5$) of the UV color-magnitude relation. The main
challenges for a more flexible parameterization would be model selection and
finding a well-motivated parameterization. 
\ch{
    Notwithstanding, for SDSS observations, our \eda~prescription using
    parameter values from the TNG and EAGLE posteriors find good agreement.
}

\ch{
    We demonstrate in this work that accounting for dust attenuation is
    essential when comparing simulations to observations. 
}
None of the simulations reproduce the UV and optical color-magnitude relation
without dust attenuation (Figure~\ref{fig:obs}). 
\ch{ 
    Furthermore, the fact that we can use the \eda~to reproduce SDSS observations 
    for different hydrodynamical simulations highlights how our current lack of 
    understanding of dust limits our ability to closely compare galaxy
    formation models. 
    Our \eda~prescription is built on our current understanding of dust
    attenuation in galaxies: \eg~the \citealt{noll2009} parameterization, the
    UV bump, the slab model, etc.
}
Yet with the \eda, two simulations that predict galaxy populations with
significantly different physical properties (Figure~\ref{fig:smf_msfr}) can
reproduce the same SDSS observations. 
This suggests that dust is highly degenerate with the differences between simulations. 
Put another way --- if we were to marginalize over dust in our comparison to observations, we would not
be able to differentiate between the different galaxy physics prescriptions in
the simulations. 
Hence, current limitations in our understanding of dust is 
a major bottleneck for investigating galaxy formation using simulations.
\ch{
    In the next paper of the series, Starkenburg et al. (in preparation), we
    will examine whether we can compare the prescriptions for star formation
    quenching in different galaxy formation models once we include the
    \eda~framework.
} 


\begin{figure}
\begin{center}
    \includegraphics[width=0.45\textwidth]{figs/abc_Lir.pdf}
    \caption{\label{fig:lir}
    IR dust emission luminosity predicted by the \eda~with median parameter
    values of the TNG (blue) and EAGLE (green) posteriors as a function of
    $M_r$. The dust emission is estimated assuming the \cite{dacunha2008}
    energy balance.  Despite reproducing the same SDSS UV and optical
    color-magnitude relations, \emph{the \eda~predicts significantly different
    IR dust emission for TNG and EAGLE}. Therefore, including IR
    observations will significantly improve the constraints on \eda~parameters
    and allow us to better differentiate galaxy formation models.
    }
\end{center}
\end{figure}

%There's hope! 
Fortunately, there are many avenues for improving our understanding of dust
with a forward modeling approach. In this work, we used a restrictive $M_r <
-20$ complete SDSS galaxy sample. 
\ch{
    Figure~\ref{fig:avmsfr} illustrates that our completeness cut limits the
    number of quiescent galaxies below $M_* < 10^{11}M_\odot$.  
}
Instead of imposing a completeness limit, 
we can include the actual SDSS selection function in the forward 
model~\citep[\eg~][]{dickey2020}. 
\ch{
    This would allow us to compare the simulations with \eda~to the entire SDSS
    sample, a substantially larger sample with a wider range of galaxies. 
}
Upcoming surveys, such as the Bright Galaxy Survey (BGS) of the Dark Energy
Spectroscopic Instrument~\citep[DESI;][]{desicollaboration2016, ruiz-macias2020} 
and galaxy evolution survey of the Prime Focus
Spectrograph~\citep[PFS;][]{takada2014,tamura2016}, will also vastly expand galaxy
observations. 
\ch{ 
    BGS, for instance, will measure $10\times$ the number of galaxy spectra as
    SDSS out to $z\sim0.4$  and with its $r\sim20$ magnitude limit will probe
    a significant number of low redshift dwarf galaxies. 
    Such an observational sample will allow us to place tigher constraints on
    the \eda~parameters, which may enable comparisons of the underlying galaxy
    formation models, 
    and shed light on dust in a broader range of galaxies. 
}

\ch{
    In this work, we also only used observables derived from UV and optical
    photometry, which means that we have only examined one side of the impact
    that dust has on galaxy spectra.
}
While dust attenuates light in the optical and UV, it emits light in
IR. In fact, even though the TNG and EAGLE simulations reproduce the same UV and
optical color-magnitude relations with the \eda, they predict significantly 
different dust emission in the IR. In Figure~\ref{fig:lir}, we present IR dust
emission luminosity, $L_{\rm IR}$, predicted by the \eda~with median parameter values of 
the TNG (blue) and EAGLE (green) posteriors as as a function of the $r$-band 
absolute magnitude, $M_r$. The dust emissions are estimated using the standard
energy balance assumption --- \ie~all starlight attenuated by dust is reemitted 
in the IR~\citep{dacunha2008}. 

Despite reproducing the same SDSS UV and optical color-magnitude relations, the
\eda~predicts significantly different IR dust emission for TNG and EAGLE. For
TNG, the \eda~predicts an overall ${\sim}0.3$ dex ($2\times$) higher dust
emissions than for EAGLE. Higher dust emission for TNG
is consistent with the higher $\ctau$ we infer for TNG (Figure~\ref{fig:abc}).
It is also consistent with the fact that TNG predicts bluer galaxies and more
luminous quiescent galaxies with red $\fnuv$ color than EAGLE
(Figure~\ref{fig:obs}). Since IR dust emission measures the total dust
attenuation, IR observations would specifically constrain the \eda~and
therefore break degeneracies between dust and the galaxy physics in simulations.
\ch{
    Upcoming will provide crucial observation on this front.  
    BGS, for instance, will have IR photometry from NEOWISE~\citep{meisner2018}. 
    \emph{James Webb Space Telecope (JWST)} will also provide valuable IR
    observations.
}

% Salim(2020): Chevallard et al. (2013), who aggregated and analyzed a diverse series of theoretical attenuation law studies by Pierini et al. (2004), Tuffs et al. (2004), Silva et al. (1998) and Jonsson et al. (2006), and showed that all the stud- ies predict, with some normalization differences, a relationship between the optical depth AV and attenuation law slope.

% Salmon+(2016): There is evidence that galaxy inclination correlates with the strength of Lyα emission, such that we observe less Lyα equivalent width for more edge-on galaxies (Charlot & Fall 1993; Laursen & Sommer-Larsen 2007; Yajima et al. 2012; Verhamme et al. 2012; U et al. 2015)
% Therefore, based on physi- cal models, one expects that galaxies with “greyer” dust laws and larger overall attenuation should have higher inclinations
% Salim+(2018): Chevallard et al. (2013) furthermore show that the depend- ence of the slope on AV is the same irrespective of whether the AV is driven by different levels of intrinsic (face-on) attenuation or is the result of inclined viewing geometry. 




%They find that
%star-forming galaxies with higher SFR have higher attenuation; however, this
%trend is driven by the $M_*$ dependence since star-forming galaxies lie on 
%the star-forming sequence~\citep{garn2010, battisti2017}. At fixed $M_*$,
%observations find no strong $\sfr$ dependence for the star-forming population. 
%Since previous works do not include quiescent galaxies, %\cite{tress2018} find positive dependence between E(B-V) (color excess) with M* and negative dependence with
%SSFR for 1753 star-forming galaxies within 1.5 < z < 3.  

%For instance, \cite{garn2010} find that SDSS star-forming galaxies with higher
%SFR have higher H$\alpha$ attenuation. \cite{battisti2016} find a consistent
%correlation for the Balmer optical depth of star-forming galaxies in GALEX and SDSS 
%using 10000 SF galaxies GALEX-SDSS. Although at higher redshifts,  
%\cite{reddy2015} also find this correlation among $z{\sim}2$ star-forming galaxies of
%MOSFIRE Deep Evolution Field.

%\cite{battisti2017} using 5000 SF galaxies from found little stellar mass dependence in the opposite direction (less attenuation at higher stellar masses). But they have big error bars and only probe up to 9.7
%\cite{reddy2015} SF galaxies from $z\sim2$ MOSFIRE Deep Evolution Field survey find strong correlation with SFR. %ionized gas is more reddened relative to the stellar continuum with increasing SFR 

%We can also examine the correlation between attenuation curve slopes and galaxy properties with the DEM. For TNG and EAGLE, we find that 
%\cite{leja2017} find composite and AGN galaxies generally have shallower  
%slopes, although their sample is limited to only 129 galaxies, In contrast,
%\cite{salim2018} find that quiescent galaxies in the GALEX-SDSS-WISE Legacy
%Catalog 2~\citep[GSWLC2;][]{salim2019} have significantly steeper curves. They,
%however, also find significantly steeper curves for the starburst population 
%(\ie~galaxies above the SFS). With no consensus in the literature and few
%observations that examine the correlation between $\delta$ and galaxy
%properties, the lack of $M_*$ and $\sfr$ dependence we find on $\delta$ is an
%interesting prediction for future observations. 


%\cite{calzetti2000} find  slopes of $2.3 < S < 2.9$ for low-redshift starburst galaxies, 
%\cite{burgarella2005} find $2.5 < S < 6.2$ for 50 UV and 100 IR selected galaxies,
%\cite{johnson2007} find $S\sim2.5$ for 1000 nearby galaxies, 
%\cite{conroy2010} find $S\sim4.5$ for 3400 $10^{9.5} < M_* < 10^{10} M_\odot$ disk galaxies,
%\cite{wild2011} find $2.5 < S < 4.5$ for 23,000 $z{\sim}0.07$ star-forming galaxies, 
%\cite{battisti2016, battisti2017} find slopes consistent with Calzetti (S=2.4) for SF galaxies 
%\cite{leja2017} 130 relatively massive galaxies 2 < S < 15
%\cite{salim2018} 230,000 SDSS galaxies 2 < S < 15 with median S = 5.4
% high z
%\cite{kriek2013}  using stacked SEDs of medium- and broadband photometry of
%galaxies at 0.5 < z < 2 find an average slope of delta=-0.2  but they restrict
%to galaxies with moderate to high optical attenuations (AV> 0.5), 
%\cite{salmon2016} is also spot on. 


% attenuation-slope relation 

%\cite{trcka2020}: EAGLE+SKIRT with CIGALE to get physical properties of
%galaxies. trcka2020 compares to dustpedia~\citep{davies2017} They lok at
%IRX-beta relation.

% variation of attenuation curve  
%\ch{what do we learn about quiescent galaxy attenuation?} 
%In \cite{leja2017}, they similar find composite and AGN galaxies
%to have shallow attenuation curves with higher $A_V$; however, the comparison
%is limited due to ther smaller sample size (129 galaxies). In contrast,  
%\cite{salim2018} find that quiescent galaxies in GSWLC2 have significantly
%steeper curves; they, however, focus their analysis mainly on star-forming 
%galaxies.

% Kartheik on why dust is hard to constrain for quiescent galaxies: 
%firstly because quiescent galaxies don't have much recent SFR and therefore not much dust (their most recently produced dust has likely happened more than a dust destruction timescale ago), 
% secondly because the continuum for quiescent galaxies is super hard to fit (and break degeneracies with SFH & metallicity), which leads to poor(er) constraints. You could basically think of this in terms of dust/total SNR, which drops sharply for this population.

