\subsection{Discussion}  
We make a number of assumptions and choices in our \eda~prescription. 
\ch{
    First, we use the slab model (Eq.~\ref{eq:slab}) to assign $A_V$ as a
    function of $\tau_V$ and randomly sampled $i$. 
    This choice is loosely motivated by the fact that the slab model reproduces
    the correlation between attenuations and inclination found in star-forming
    galaxies from observations~\citep{conroy2010b, wild2011, battisti2017,
    salim2020} as well as simulations~\citep[\eg][]{chevallard2013,
    narayanan2018, trayford2020}.
    More importantly, the slab model can reproduce the $A_V$ distribution of
    SDSS star-forming galaxies as well as the GSWLC2 sample, which includes
    quiescent galaxies (Appendix~\ref{sec:slab}).
    We therefore conclude that we can sample $A_V$ in our \eda~with sufficient
    flexibility using the slab model. 
}
%It can also reproduce the SDSS $A_V$ distribution (Figure~\ref{fig:av_dist}). If we replace the slab model with a more flexible model for sampling $A_V$ using truncated normal distributions, we find that our results are not significantly impacted (see Appendix~\ref{sec:slab} for details). Therefore, we conclude that our results do not sigificantly depend on our choice of the slab model. 
\ch{
    In our \eda, we also use a parameterization of $\tau_V$ and $\delta$ that
    depend linearly on $\log M_*$ and $\log {\rm SSFR}$. 
    While the $M_*$ and $\ssfr$ dependence of $A_V$ is well-motivated and is
    found in, for instance, the \cite{salim2018} GSWLC2 catalog (Appendix~\ref{sec:slab}), 
    the linear dependence was chosen primarily for its simplicity.
    The \eda~framework can be easily extended to more flexible
    parameterizations. 
}
In fact, a more flexible parameterization would likely produce even better
agreement with the SDSS color-magnitude relations. The \eda~produces broader
distributions of optical colors than SDSS. Few galaxies in SDSS have $\gr > 1.$
while some galaxies in the \eda~broadly extend beyond this cut-off. In the UV,
the \eda~struggles to accurately reproduce the redder portions ($\fnuv > 1.5$)
of the UV color-magnitude relation. The main challenges for a more flexible
parameterization would be model selection and finding a well-motivated
parameterization. 
\ch{
    Notwithstanding, for SDSS observations, our \eda~prescription using
    parameter values from the TNG and EAGLE posteriors find good agreement.
}

% do we want to mention the FUV and NUV absolute magnitude measurements for
% SDSS? 

\chedit{
    Figure~\ref{fig:avmsfr} also highlights the selection function imposed by
    our forward model. 
    We impose a $M_r < -20$, $M_{FUV} < 13.5$, and $M_{NUV} < -14$
    absolute magnitude completeness limits of our SDSS sample to the
    observables of each simulation. 
    The forward modeled samples expectedly include lower mass star-forming
    galaxies that are more luminous, while they exclude lower mass quiescent
    galaxies that are less luminous. 
    While the forward modeled samples of TNG and EAGLE have comparable $M_*$
    and $\ssfr$ coverage, we find that SIMBA includes a significant population
    of lower mass star-forming and quiescent galaxies. 
    One reason for this is that SIMBA has much fewer massive galaxies above
    $M_* > 10^{11}M_\odot$ than TNG or EAGLE (see SMFs in
    Figure~\ref{fig:smf_msfr}). 
    With fewer massive galaxies, SIMBA requires less attenuation for quiescent
    galaxies and, thus, its forward model includes lower mass quiescent
    galaxies.
    Meanwhile, the $M_* < 10^{10}M_\odot$ lower mass star-forming galaxies included in
    SIMBA are high $\ssfr > 10^{-9.5}yr^{-1}$ starburst galaxies that lie
    significantly above the SFS (Figure~\ref{fig:smf_msfr}).
    This starburst population has also been identified in \cite{dave2019} (see
    their Figures~5 and 6) and is caused by large gas in low-mass galaxies at $z=0$. 
} 




\ch{
    We demonstrate that accounting for dust attenuation is essential when
    comparing simulations to observations. 
}
None of the simulations reproduce the UV and optical color-magnitude relation
without dust attenuation (Figure~\ref{fig:obs}). 
\ch{ 
    Furthermore, the fact that we can use the \eda~to reproduce SDSS observations 
    for different hydrodynamical simulations highlights how our current lack of 
    understanding of dust limits our ability to closely compare galaxy
    formation models. 
    Our \eda~prescription is built on what we currently know about dust
    attenuation in galaxies: \eg~the \citealt{noll2009} parameterization, the
    UV bump, the slab model, etc.
}
Yet with the \eda, two simulations that predict galaxy populations with
significantly different physical properties (Figure~\ref{fig:smf_msfr}) can
reproduce the same SDSS observations. 
This suggests that dust is highly degenerate with the differences between simulations. 
Put another way --- if we were to marginalize over dust in our comparison to observations, we would not
be able to differentiate between the different galaxy physics prescriptions in
the simulations. 
Hence, current limitations in our understanding of dust is 
a major bottleneck for investigating galaxy formation using simulations.
\ch{
    In the next paper of the series, Starkenburg et al. (in preparation), we
    will examine whether we can compare the prescriptions for star formation
    quenching in different galaxy formation models once we include the
    \eda~framework.
} 


\begin{figure}
\begin{center}
    \includegraphics[width=0.45\textwidth]{figs/abc_Lir.pdf}
    \caption{\label{fig:lir}
    IR dust emission luminosity predicted by the \eda~with median parameter
    values of the TNG (blue) and EAGLE (green) posteriors as a function of
    $M_r$. The dust emission is estimated assuming the \cite{dacunha2008}
    energy balance.  Despite reproducing the same SDSS UV and optical
    color-magnitude relations, \emph{the \eda~predicts significantly different
    IR dust emission for TNG and EAGLE}. Therefore, including IR
    observations will significantly improve the constraints on \eda~parameters
    and allow us to better differentiate galaxy formation models.
    }
\end{center}
\end{figure}

%There's hope! 
Fortunately, there are many avenues for improving our understanding of dust
with a forward modeling approach. In this work, we used a restrictive $M_r <
-20$ complete SDSS galaxy sample. 
\ch{
    Figure~\ref{fig:avmsfr} illustrates that our completeness cut limits the
    number of quiescent galaxies below $M_* < 10^{11}M_\odot$.  
}
Instead of imposing a completeness limit, 
we can include the actual SDSS selection function in the forward 
model~\citep[\eg~][]{dickey2020}. 
\ch{
    This would allow us to compare the simulations with \eda~to the entire SDSS
    sample, a substantially larger sample with a wider range of galaxies. 
}
Upcoming surveys, such as the Bright Galaxy Survey (BGS) of the Dark Energy
Spectroscopic Instrument~\citep[DESI;][]{desicollaboration2016, ruiz-macias2020} 
and galaxy evolution survey of the Prime Focus
Spectrograph~\citep[PFS;][]{takada2014,tamura2016}, will also vastly expand galaxy
observations. 
\ch{ 
    BGS, for instance, will measure $10\times$ the number of galaxy spectra as
    SDSS out to $z\sim0.4$  and with its $r\sim20$ magnitude limit will probe
    a significant number of low redshift dwarf galaxies. 
    Such an observational sample will allow us to place tighter constraints on
    the \eda~parameters, which may enable comparisons of the underlying galaxy
    formation models, 
    and shed light on dust in a broader range of galaxies. 
}

\ch{
    In this work, we also only used observables derived from UV and optical
    photometry, which means that we have only examined one side of the impact
    that dust has on galaxy spectra.
}
While dust attenuates light in the optical and UV, it emits light in
IR. In fact, even though the TNG and EAGLE simulations reproduce the same UV and
optical color-magnitude relations with the \eda, they predict significantly 
different dust emission in the IR. In Figure~\ref{fig:lir}, we present IR dust
emission luminosity, $L_{\rm IR}$, predicted by the \eda~with median parameter values of 
the TNG (blue) and EAGLE (green) posteriors as a function of the $r$-band 
absolute magnitude, $M_r$. The dust emissions are estimated using the standard
energy balance assumption --- \ie~all starlight attenuated by dust is reemitted 
in the IR~\citep{dacunha2008}. 

Despite reproducing the same SDSS UV and optical color-magnitude relations, the
\eda~predicts significantly different IR dust emission for TNG and EAGLE. For
TNG, the \eda~predicts an overall ${\sim}0.3$ dex ($2\times$) higher dust
emissions than for EAGLE. Higher dust emission for TNG
is consistent with the higher $\ctau$ we infer for TNG (Figure~\ref{fig:abc}).
It is also consistent with the fact that TNG predicts bluer galaxies and more
luminous quiescent galaxies with red $\fnuv$ color than EAGLE
(Figure~\ref{fig:obs}). Since IR dust emission measures the total dust
attenuation, IR observations would specifically constrain the \eda~and
therefore break degeneracies between dust and the galaxy physics in simulations.
\ch{
    While some upcoming surveys, such as BGS, will have existing near-IR
    photometry from NEOWISE~\citep{meisner2018}, future observations will
    dramatically expand the information we have in IR.
    \emph{Nancy Grace Roman Space Telescope} and \emph{James Webb Space
    Telescope (JWST)}, for instance, will provide valuable near and
    mid-IR observations. 
    Meanwhile, IR observations at even longer wavelengths will came from
    Atacama Large Millimeter/submillimeter Array (ALMA) or future facilities
    such as the Next-Generation Very Large Array (ngVLA) and Origins Space Telescope.
}

% Salim(2020): Chevallard et al. (2013), who aggregated and analyzed a diverse series of theoretical attenuation law studies by Pierini et al. (2004), Tuffs et al. (2004), Silva et al. (1998) and Jonsson et al. (2006), and showed that all the stud- ies predict, with some normalization differences, a relationship between the optical depth AV and attenuation law slope.

% Salmon+(2016): There is evidence that galaxy inclination correlates with the strength of Lyα emission, such that we observe less Lyα equivalent width for more edge-on galaxies (Charlot & Fall 1993; Laursen & Sommer-Larsen 2007; Yajima et al. 2012; Verhamme et al. 2012; U et al. 2015)
% Therefore, based on physi- cal models, one expects that galaxies with “greyer” dust laws and larger overall attenuation should have higher inclinations
% Salim+(2018): Chevallard et al. (2013) furthermore show that the depend- ence of the slope on AV is the same irrespective of whether the AV is driven by different levels of intrinsic (face-on) attenuation or is the result of inclined viewing geometry. 


