\documentclass{aastex63}
\usepackage[T1]{fontenc}
\usepackage{color}
\usepackage{amsmath}
\usepackage{natbib}
\usepackage{ctable}
\usepackage{bm}
\usepackage{xspace}

\usepackage{graphicx}

% math shih
\newcommand{\setof}[1]{\left\{{#1}\right\}}
\newcommand{\given}{\,|\,}
\newcommand{\lss}{{\small{LSS}}\xspace}

\newcommand{\Om}{\Omega_{\rm m}} 
\newcommand{\Ob}{\Omega_{\rm b}} 
\newcommand{\OL}{\Omega_\Lambda}
\newcommand{\smnu}{M_\nu}
\newcommand{\sig}{\sigma_8} 
\newcommand{\mmin}{M_{\rm min}}
\newcommand{\BOk}{\widehat{B}_0} 
\newcommand{\hmpc}{\,h/\mathrm{Mpc}}
\newcommand{\bfi}[1]{\textbf{\textit{#1}}}
\newcommand{\parti}[1]{\frac{\partial #1}{\partial \theta_i}}
\newcommand{\partj}[1]{\frac{\partial #1}{\partial \theta_j}}
\newcommand{\mpc}{{\rm Mpc}}
\newcommand{\eg}{\emph{e.g.}}
\newcommand{\ie}{\emph{i.e.}}

\let\oldAA\AA
\renewcommand{\AA}{\text{\normalfont\oldAA}}
% cmds for this paper 
\newcommand{\gr}{g{-}r}
\newcommand{\fnuv}{FUV{-}NUV}
\newcommand{\sfr}{{\rm SFR}}
\newcommand{\ssfr}{{\rm SSFR}}
\newcommand{\mtaum}{m_{\tau,M_*}}
\newcommand{\mtaus}{m_{\tau,{\rm SSFR}}}
\newcommand{\ctau}{c_\tau}
\newcommand{\mdeltam}{m_{\delta,M_*}}
\newcommand{\mdeltas}{m_{\delta,{\rm SFR}}}
\newcommand{\cdelta}{c_\delta}
\newcommand{\eda}{EDA}


\newcommand{\specialcell}[2][c]{%
  \begin{tabular}[#1]{@{}c@{}}#2\end{tabular}}
% text shih
\newcommand{\foreign}[1]{\textsl{#1}}
\newcommand{\etal}{\foreign{et~al.}}
\newcommand{\opcit}{\foreign{Op.~cit.}}
\newcommand{\documentname}{\textsl{Article}}
\newcommand{\equationname}{equation}
\newcommand{\bitem}{\begin{itemize}}
\newcommand{\eitem}{\end{itemize}}
\newcommand{\beq}{\begin{equation}}
\newcommand{\eeq}{\end{equation}}

%% collaborating
\newcommand{\todo}[1]{\marginpar{\color{red}TODO}{\color{red}#1}}
\definecolor{orange}{rgb}{1,0.5,0}
\newcommand{\ch}[1]{#1}%{\color{orange}#1}}
\newcommand{\chedit}[1]{{\color{orange}#1}}
\newcommand{\tks}[1]{{\color{blue}#1}}
\newcommand{\tksedit}[1]{{\color{blue}#1}}

\shorttitle{The Empirical Dust Attenuation Framework}
\shortauthors{Hahn et al.}


\begin{document} 

\title{IQ Collaboratory III: The Empirical Dust Attenuation Framework ---
Taking Hydrodynamical Simulations with a Grain of Dust}

\author[0000-0003-1197-0902]{ChangHoon Hahn}
\correspondingauthor{ChangHoon Hahn}
\email{changhoon.hahn@princeton.edu}
\affil{Department of Astrophysical Sciences, Princeton University, Peyton Hall, Princeton NJ 08544, USA} 

\author{Tjitske K. Starkenburg}
\affil{Center for Interdisciplinary Exploration and Research in Astrophysics (CIERA) and \\Department of Physics and Astronomy, 1800 Sherman Ave, Evanston IL 60201, USA}

\author{Daniel Angl\'es-Alc\'azar}
\affil{Department of Physics, University of Connecticut, 196 Auditorium Road,
U-3046, Storrs, CT 06269-3046, USA}
\affil{Center for Computational Astrophysics, Flatiron Institute, 162 5th Ave,
New York, NY 10010, USA}

\author[0000-0002-8131-6378]{Ena Choi}
\affil{Quantum Universe Center, 
        Korea Institute for Advanced Study, 
        Hoegiro 85, Seoul 02455, Korea}

\author{Romeel Dav\'e}
\affil{Institute for Astronomy, Royal Observatory, Univ. of Edinburgh, Edinburgh EH9 3HJ, UK}
\affil{University of the Western Cape, Bellville, Cape Town 7535, South Africa}
\affil{South African Astronomical Observatories, Observatory, Cape Town 7925, South Africa}

\author[0000-0002-1081-3991]{Claire Dickey} 
\affil{Department of Astronomy, Yale University, New Haven CT 06520 USA}

%\suppressAffiliations
\author[0000-0001-9298-3523]{Kartheik G. Iyer}
\affil{Dunlap Institute for Astronomy and Astrophysics, University of Toronto,
50 St George St, Toronto, ON M5S 3H4, Canada}

\author[0000-0003-2060-8331]{Ariyeh H. Maller}
\affil{Department of Physics, New York City College of Technology, City
University of New York, 300 Jay St., Brooklyn, NY 11201, USA}

\author{Rachel S. Somerville}
\affil{Department of Physics and Astronomy, Rutgers University, 136
Frelinghuysen Road, Piscataway, NJ 08854, USA}
\affil{Center for Computational Astrophysics, Flatiron Institute, 162 5th Ave,
New York, NY 10010, USA}

\author{Jeremy L. Tinker}
\affil{Center for Cosmology and Particle Physics, Department of Physics, New York University, 4 Washington Place, New York, NY 10003}

\author{L. Y. Aaron Yung}
\affil{Astrophysics Science Division, NASA Goddard Space Flight Center, 8800 Greenbelt Rd, Greenbelt, MD 20771, USA}

%\collaboration{2}{IQ Collaboratory}

%\AuthorCallLimit=3

\begin{abstract}
    We present the Empirical Dust Attenuation (\eda) framework --- a flexible
    prescription for 
    %for applying dust attenuation to galaxies from theoretical models. \eda~provides a flexible prescription 
    assigning realistic dust attenuation to simulated galaxies based on their
    physical properties. % ($M_*$, $\ssfr$). %  and using the \cite{noll2009} attenuation curve parameterization. 
    We use the \eda~to forward model synthetic observations for three state-of-the-art large-scale cosmological
    hydrodynamical simulations: SIMBA, IllustrisTNG, and EAGLE. 
    We then compare the optical
    and UV color-magnitude relations, $(\gr) - M_r$ and $(\fnuv)-M_r$, of the
    simulations to a $M_r < -20$ complete SDSS galaxy sample using
    likelihood-free inference. 
    Without dust, none of the simulations match observations, as expected.
    %As expected, the simulations cannot match observations without dust. % for all three simulations, dust attenuation is {\em necessary} to match the observations.
    With the \eda, however, we can reproduce the observed color-magnitude
    with all three simulations. 
    %For SIMBA and TNG, we find an excess of luminous UV-red galaxies that dust cannot fix because the simulations quench star formation too efficiently in the most massive galaxies. 
    Furthermore, the attenuation curves predicted by our dust prescription are
    in good agreement with the observed attenuation--slope relations and
    attenuation curves of star-forming galaxies. 
    However, the \eda~does not predict star-forming galaxies with low
    $A_V$ since simulated star-forming galaxies are intrinsically much
    brighter than observations.
    Additionally, the \eda~provides, for the first time, predictions
    on the attenuation curves of quiescent galaxies, which are challenging to
    measure observationally. 
    Simulated quiescent galaxies require shallower attenuation curves
    with lower amplitude than star-forming galaxies.
    %Overall, we find that more massive galaxies in the simulations require higher dust attenuation, while galaxies with higher specific star formation rates have steeper attenuation curves.
    The \eda, combined with forward modeling, provides an effective
    approach for shedding light on  dust in galaxies and
    probing hydrodynamical simulations. 
    This work also illustrates a major limitation in comparing galaxy
    formation models: by adjusting dust attenuation, simulations that predict
    significantly different galaxy populations can reproduce the same UV
    and optical observations. 
\end{abstract}

\keywords{
    galaxies: formation -- galaxies: evolution -- galaxies: statistics -- methods: numerical 
}
\NewPageAfterKeywords

% --- intro ---  
\section{Introduction} \label{sec:intro} 
% paragraph on why dust is important 
\ch{ 
    Dust in the interstellar medium of a galaxy can dramatically impact its
    spectral energy distribution (SED). % over the full wavelength range. % of the electromagnetic spectrum. 
    The combined effect of dust on a galaxy's SED is typically described using
    an attenuation curve, $A(\lambda)$, which has now been broadly characterized 
    by observations. 
    In the UV, attenuation curves steeply rise due to absorptions by small grains.
    At $2175\AA$, in the near-UV (NUV), there is an absorption bump referred to as
    the ``UV dust bump''. 
    At longer wavelengths, the curves take on a power-law shape. 
    Finally, dust reemits the attenuated light in the optical and UV in the
    infrared~\citep[for an overview see][]{calzetti2001, draine2003,
    galliano2018}.
    By impacting the SED, dust also affects the physical properties that are
    inferred from the SED, such as star formation rate ($\sfr$), stellar
    mass ($M_*$), or stellar ages~\citep[see reviews by][]{walcher2011, conroy2013}. 
    Assumptions on dust attenuation can dramatically vary the derived physical
    properties~\citep{kriek2013, reddy2015, salim2016, salim2020}.
}
Since these properties are the building blocks to our understanding of
galaxies and how they evolve, a better understanding of dust not only provides
insights into dust, but also underpins all galaxy studies.  

% paragraph on why dust is hard with backwards modeling observations and RT
\ch{
    To better understand dust in galaxies, many observational works have
    examined trends between dust attenuation and galaxy properties.
    For example, UV and optical attenuation are found to correlate with 
    $M_*$, $\sfr$, and metallicity in star-forming
    galaxies~\citep{garn2010, battisti2016}. 
    The slope of the attenuation curves in star-forming galaxies also correlate
    with galaxy properties, such as $M_*$, specific $\sfr$ ($\ssfr$),
    metallicity, and axial ratio~\citep{wild2011, battisti2017}. 
    Recently, \cite{salim2018} argue that these correlations stem from the
    underlying ``attenuation-slope relation'', a trend between the amplitude of
    attenuation and slope. 
    Despite the progress, there is still no clear consensus on the connection
    between attenuation curves and galaxy properties.
}
Also, studies so far have focused solely on star-forming galaxies and little is known
about dust attenuation in quiescent galaxies. 
\ch{
    Furthermore, galaxy properties, including dust attenuation, measured from
    galaxy SEDs are subject to variations, inconsistencies, and biases of
    different methodologies, which can be significant even for the same
    observations~\citep[\eg][]{speagle2014, katsianis2020}.
    SED fitting can also impose undesirable priors on derived galaxy
    properties~\citep{carnall2018, leja2019} and suffer from parameter
    degeneracies that are poorly understood. 
}

\ch{
    Significant progress has also been made in theoretically modeling dust. 
    Simulations can now model the radiative transfer of stellar light
    through a dusty ISM for a wide range of configurations:
    from simple slab-like dust geometries illuminated
    by stellar radiation~\citep[\eg][]{witt1996, witt2000, seon2016} to 
    3D hydrodynamic simulations of entire galaxies~\citep[\eg][]{jonsson2006,
    rocha2008, hayward2015, natale2015, hou2017}. 
    Radiative transfer models have even been applied to cosmological
    hydrodynamical simulations~\cite[\eg][]{camps2015, narayanan2018,
    trayford2020}. 
    Dust has also been examined in a cosmological context using 
    semi-analytic models ~\citep[SAMs; \eg][]{granato2000, fontanot2009, wilkins2012,
    gonzalez-perez2013, popping2017}. 
    Yet there are still major limitations in modeling dust. 
    Dust models in cosmological simulations currently do not reproduce the
    redshift evolution of dust properties~\citep{somerville2012, yung2019,
    vogelsberger2020}. 
    Radiative transfer models produce attenuation-slope relations that are
    significantly steeper than observations. 
}
Many models also require significant hand-tuning (\eg~propagating rays/photons into
particular cells) and make assumptions on the underlying dust grain models~\citep[see][for a review]{steinacker2013}. 
%Lastly, radiative transfer models are computationally expensive.  
%Applying a range of radiative transfer dust models to multiple simulations for
%comparisons would require huge computational resources.
%Using Markov Chain Monte Carlo to sample them for parameter exploration or
%to marginalize over the impact of dust would be prohibitive.  

\ch{
    We take a different approach from the observational and theoretical works
    above --- \emph{we investigate dust attenuation using a forward modeling
    approach to compare simulations to observations}.
    Our ``forward model'' starts with three major large-scale hydrodynamical
    simulations: EAGLE, IllustrisTNG, and SIMBA. 
    We use their outputs (\eg~star formation history) to build SEDs for each
    simulated galaxy.
    %We take three major large-scale hydrodynamical simulations (EAGLE, IllustrisTNG, and SIMBA) and use their outputs (\eg~star formation history) to build SEDs for the simulated galaxies. 
    We then apply dust attenuation to all the SEDs using an Empirical Dust
    Attenuation framework, which we describe shortly.
    Finally, we construct synthetic photometric observations from the
    attenuated SEDs by applying a realistic noise model and survey selection
    function. 
    Afterwards, we compare the synthetic observations from the forward model to
    actual observations. 
    The comparisons are made in observational space, so they are not impacted
    by the inconsistencies of observational methods for measuring galaxy
    properties.
    Furthermore, since the forward models can directly include the selection
    functions and observational systematic effects, forward modeling makes it
    easier to account for these effects and to exploit the full observational
    data set.
}

\ch{ 
    An essential step in our forward model is applying the Empirical Dust
    Attenuation (\eda) framework, which provides a flexible and computationally
    inexpensive prescription for applying dust attenuation.  
    The \eda~first assigns attenuation curves to every simulated galaxy. 
    In this work, we use the \cite{noll2009} parameterization for the
    attenuation curves and determine the amplitude and slope of the curves from
    the simulated galaxy's stellar mass ($M_*$) and specific star formation
    rate ($\ssfr$) as well as a randomly sampled inclination ($i$). 
    We use the same parameterization as observational studies so that the
    \eda~attenuation curves can easily be compared to observational constraints. 
    Also, the $M_*$ and $\ssfr$ dependence is motivated by
    observations~\citep[\eg~][]{garn2010, wild2011, battisti2016, leja2017,
    salim2018, salim2020}.
    After the assignment, we simply apply the attenuation curves to the SEDs of
    the simulated galaxies.
    The \eda, as we later demonstrate, can produce realistic distributions of
    dust attenuation for galaxy populations. 
    Unlike radiative transfer models, however, the \eda~does not produce
    realistic dust attenuation for individual galaxies.  
    The \eda~provides an empirical mapping framework for dust attenuation,
    analogous to the halo occupation or abundance matching frameworks in 
    galaxy formation~\citep[for a review see ][]{wechsler2018}. 
}

\ch{
    In principle, a radiative transfer model can be used instead of the \eda;
    however, radiative transfer models are computationally expensive.  
    Applying a range of radiative transfer dust models to multiple simulations
    for comparisons would require huge computational resources.
    Using them with Monte Carlo methods for parameter exploration would be
    prohibitive.  
    On the other hand, with the \eda, we can apply a wide range of realistic
    dust attenuation to simulated galaxies in a matter of seconds. 
    We can easily explore and sample the dust parameter space and infer the
    relationship between dust attenuation and galaxy properties. 
    That is the focus of this paper. 
    Beyond investigating dust, the \eda~also provides a framework where we can
    treat dust as {\em nuisance} parameters and tractably marginalize over dust
    attenuation. 
    In the subsequent paper of the IQ series, Starkenburg et al. (in
    preparation), we will use the \eda~framework to compare star formation
    quenching in cosmological galaxy formation models after marginalizing over
    dust attenuation. 
}

\ch{
    In Section~\ref{sec:sims}, we describe the three state-of-the-art
    cosmological large-scale hydrodynamical simulations (SIMBA, IllustrisTNG,
    and EAGLE) that we use in our forward model.
    We also describe the observed SDSS galaxy sample used for the comparison. 
    Next, we present the specific \eda~prescription used in this work in
    Section~\ref{sec:dem} and the simulation-based inference method for
    comparing the simulations to observations in Section~\ref{sec:abc}. 
    Finally, in Section~\ref{sec:results}, we present the results of our
    comparison and discuss their implications on dust attenuation and its
    connection to galaxy properties. 
}

%motivation for an empirical dust attenuation model
%\ch{
%    We take a different approach from the observational and theoretical works above.  
%    In the standard observational approach, galaxy properties, including dust
%    attenuation, are measured from galaxy SEDs. 
%    The measurements, however, are subject to variations, inconsistencies, and
%    biases of different methodologies, which can be significant even for the
%    same observations~\citep[\eg][]{salim2007, kennicutt2012, speagle2014,
%    flores2020, katsianis2020}.
%    SED fitting can also impose undesirable priors on derived galaxy
%    properties~\citep{carnall2018, leja2019} and suffer from parameter
%    degeneracies that are poorly understood. 
%    Meanwhile, radiative transfer models are computationally expensive
%    Applying a range of radiative transfer dust models to multiple simulations
%    for comparisons would require huge computational resources.
%    Using Markov Chain Monte Carlo to sample them for parameter exploration or
%    to marginalize over the impact of dust would be prohibitive.  
%}


%\ch{
%    \emph{Instead, in this paper we investigate dust attenuation using a
%    forward modeling approach to compare simulations to observations}.
%    Our ``forward model'' starts with three major large-scale hydrodynamical
%    simulations: EAGLE, IllustrisTNG, and SIMBA. 
%    We use their outputs (\eg~star formation history) to build SEDs for each
%    simulated galaxy.
%    %We take three major large-scale hydrodynamical simulations (EAGLE, IllustrisTNG, and SIMBA) and use their outputs (\eg~star formation history) to build SEDs for the simulated galaxies. 
%    We then apply dust attenuation to all the SEDs using an Empirical Dust
%    Attenuation (\eda) framework.
%    Finally, we construct synthetic photometric observations from the
%    attenuated SEDs by applying a realistic noise model and survey selection
%    function. 
%    Afterwards, we compare the synthetic observations from the forward model to
%    actual observations. 
%
%}
    

%The \eda~model, with its flexibility and speed, provides a crucial step in a
%{\em forward modeling approach} to comparing simulations to
%observations~\citep[\eg][]{nelson2018, baes2019, trcka2020, dickey2020}.
%In the standard approach, the galaxy properties (\eg~SFR, $M_*$) predicted by
%simulations are compared to those derived from observations. 
%In the forward
%modeling approach, observable quantities (\eg~magnitude, color) are {\em forward
%modeled} for each galaxy in the simulations; then the simulations are compared 
%to observations in observational space. With a forward modeling approach, 
%comparisons are not limited by variations, inconsistencies, and biases of different
%observational methods for measuring galaxy properties (\eg~different tracers of
%SFR or $M_*$). Selection functions and systematic effects can also be accounted
%for in the forward model. 


%and present the Empirical Dust
%Attenuation (\eda) model, a flexible framework for statistically applying dust
%attenuation to galaxy populations. The \eda~model assigns, to every galaxy, a
%different dust attenuation curve. We present attenuation curves that are parameterized with
%a functional-form used in observational studies~\citep{noll2009} and the
%parameters of the curve for each galaxy (\eg~optical depth, slope) are sampled
%from distributions set by the \eda~model parameters and the galaxy's properties.  
%By sampling the attenuation curve parameters, it produces realistic variations 
%among the attenuation curves. The \eda~does not seek to produce realistic dust 
%attenuation for individual galaxies, like radiative transfer models. Instead, 
%it aims to produce realistic dust attenuations for a large ensemble of galaxies
%that enables direct comparison across galaxy populations. 

%There are a number of advantages to our \eda~approach. The \eda~uses the same
%functional-form for the attenuation curves as observational studies, so
%predictions of the model can easily be compared to observational constraints. 
%We also formulate the \eda~parameters so that they are easily interpretable and
%directly reveal correlations between dust attenuation and galaxy properties.
%Lastly, the \eda~model is computationally inexpensive. 
%
%The \eda~model, with its flexibility and speed, provides a crucial step in a
%{\em forward modeling approach} to comparing simulations to
%observations~\citep[\eg][]{nelson2018, baes2019, trcka2020, dickey2020}.
%In the standard approach, the galaxy properties (\eg~SFR, $M_*$) predicted by
%simulations are compared to those derived from observations. In the forward
%modeling approach, observable quantities (\eg~magnitude, color) are {\em forward
%modeled} for each galaxy in the simulations; then the simulations are compared 
%to observations in observational space. With a forward modeling approach, 
%comparisons are not limited by variations, inconsistencies, and biases of different
%observational methods for measuring galaxy properties (\eg~different tracers of
%SFR or $M_*$). Selection functions and systematic effects can also be accounted
%for in the forward model. 
%
%Of course, to produce realistic observables of simulated galaxies, the forward
%model must include some prescription for applying dust attenuation. In
%principle, a radiative transfer model can be used for this purpose; however,
%its computational cost would severely limit any exploration of the ``dust
%parameters''. With the \eda, however, we can apply a wide range of realistic
%dust attenuation to simulated galaxies in a matter of seconds and easily
%explore and sample the \eda~parameter space to infer the 
%relationship between dust attenuation and galaxy properties. {\em For readers 
%uninterested in dust}, the \eda~provides a way to treat dust parameters as
%{\em nuisance} parameters and tractably marginalize over dust attenuation. 

%In this work, we present a simple \eda~model that uses the \cite{noll2009}
%attenuation curve parameterization and includes correlations between dust
%attenuation and galaxy $M_*$ and $\sfr$ (Section~\ref{sec:dem}). We apply 
%the \eda~separately to three state-of-the-art cosmological large-scale hydrodynamical 
%simulations (SIMBA, IllustrisTNG, and EAGLE), which we describe in
%Section~\ref{sec:sims}, and compare them to a volume-limited galaxy sample from SDSS
%and GALEX observations (Section~\ref{sec:obs}). In Section~\ref{sec:dem}, we
%describe our \eda~model in detail. Finally, in Section~\ref{sec:results}, we
%present the results of our comparison and discuss their implications for our
%understanding of dust attenuation as well as of its connection to galaxy properties. 

% --- data ---  
\section{Simulations}\label{sec:sims}

\subsection{Illustris TNG} \label{sec:tng}
\todo{describe what galaxy properties (SFH, ZH, etc) are available} 

\subsection{SIMBA} \label{sec:tng}
\todo{describe what galaxy properties (SFH, ZH, etc) are available} 

\subsection{Spectral Energy Distributions} \label{sec:sed}
\todo{describe how the SED is generated using the SFH and ZHs} 

\subsection{Forward Modeling SDSS Photometry and Spectra} \label{sec:fm} 
 
% --- methods ---  
\section{The Empirical Dust Attenuation Framework} \label{sec:dem}
\tks{how do you feel now about alternative names? 
Rachel proposed EDAM; in our discussion of names I think the best I came with was DEAP 
(Dust Empirical Attenuation Prescription); looking at this section title we could also go for FADE which I really like in the context 
(Framework to Attenuate by Dust Empirically).}
\chedit{
    In this section, we describe the Empirical Dust Attenuation (\eda)
    framework and present one particular \eda~prescription. The \eda~is a
    flexible framework for applying dust attenuation curves to simulated galaxy
    populations. For each simulated galaxy, the \eda~assigns a dust attenuation
    curve assigned that is parameterized as a function of the galaxy's
    properties (\eg~$M_*$, ${\rm SSFR}$), the \eda~parameters, and randomly
    sampled inclincation. With the \eda, we can apply a wide variety of dust
    attenuation that include correlation between dust attenuation and physical
    galaxy properties. 
}
% Later, we demonstrate that we can accurately reproduce SDSS observations with the \eda~and use it to test galaxy formation models and shed light on dust in galaxies. 

We begin by defining the dust attenuation curve, $A(\lambda)$, as 
\begin{equation} \label{eq:full_atten}
    F_o (\lambda) = F_i (\lambda) 10^{-0.4 A(\lambda)}
\end{equation}
where $F_o$ is the observed flux and $F_i$ is the intrinsic flux. We normalize
the attenuation to the $V$ band attenuation, 
\begin{equation} 
    A(\lambda) = A_V \frac{k(\lambda)}{k_V}
\end{equation}
so that $A_V$ determines the amplitude of the attenuation, while $k(\lambda)$
determines the wavelength dependence. 

\chedit{
    The \eda~framework assigns $A(\lambda)$ to every galaxy in the simulations
    using some flexible prescription. For the \eda~prescription in this work,
    we assign $A_V$ for each galaxy using the slab model, where $A_V$ is a
    function of galaxy inclination, $i$, and its optical depth,
    $\tau_V$~\citep[\eg][]{somerville1999, somerville2012}: 
}
\begin{equation} \label{eq:slab}
    A_V = -2.5 \log \left[ \frac{1 - e^{-\tau_V\,\sec i}}{\tau_V\,\sec i} \right].
\end{equation}
\chedit{
    To include correlation between $A_V$ and the galaxy's properties ($M_*$ and
    $\ssfr$), we parameterize $\tau_V$ using a linear $M_*$ and $\ssfr$ dependence:
}
\begin{equation} \label{eq:tauv}
    \tau_V(M_*, \sfr) = \mtaum \log \left(\frac{M_*}{10^{10} M_\odot}\right) +
    \mtaus \log \left(\frac{\ssfr}{10^{-10}yr^{-1}}\right) + c_\tau.
\end{equation}
$\mtaum$, $\mtaus$, and $c_\tau$ represent the $M_*$ dependence, the $\ssfr$
dependence, and amplitude of $\tau_V$. Since $\tau_V$ is optical depth, we
impose a $\tau_V \ge 0$ limit.
\chedit{
    For each galaxy, we uniformly sample $\cos i$ from 0 to 1. By sampling
    $\cos i$, our \eda~prescription includes significant variance in
    $A(\lambda)$. Galaxies with the same galaxy properties do not have the same
    dust attenuation.
}

\chedit{
    We use the slab model primarily as a flexible prescription for $A_V$ that
    depend on a randomly sampled $i$, with \emph{loose} physical motivations.
    The slab model is a naive approximation. $A_V$, in reality, depends on
    properties such as the detailed star-to-dust geometry or variations in the
    extinction curves. However, the purpose ~\eda~is to assign an accurate
    distribution of dust attenuation curves for the galaxy population --- 
    \emph{not} to accurately model dust attenuation for individual galaxies.
    In Appendix~\ref{sec:slab}, we present how our slab model based \eda~with
    randomly sampled $i$, produces a $A_V$ distribution, $p(A_V)$ that 
    matches $p(A_V)$ of the \cite{salim2018} GSWLC2 catalog. 
}

\chedit{
    All galaxies in the simulations are assigned $A_V$ from the slab model.
    For star-forming galaxies, which typically have disc-like morphologies, the
    slab model produces $A_V$ that is correlated with $i$ in a way consistent
    with the literature: edge-on galaxies have higher $A_V$ than face-on
    galaxies~\citep[\eg][]{conroy2010, wild2011, battisti2017, salim2020}. 
    Quiescent galaxies, however, typically have ellipitcal morphologies. In
    this case, the slab model is an \emph{empirical} prescription for statistically 
    sampling $A_V$. In Appendix~\ref{sec:slab}, we demonstrate that slab model 
    can match $p(A_V)$ of quiescent galaxies ($\ssfr < 10^{-11}yr^{-1}$) 
    in GSWLC2. We, therefore, conclude that the slab model is a sufficiently 
    flexible prescription for sampling $A_V$ for all galaxies. 
}


For the wavelength dependence of the attenuation curve, $k(\lambda)$, we
use the \cite{noll2009} parameterization: 
\begin{equation} \label{eq:noll}
    k(\lambda) = \left(k_{\rm Cal}(\lambda) + D(\lambda)\right) \left(
    \frac{\lambda}{\lambda_V} \right)^\delta.
\end{equation}
Here $k_{\rm Cal}(\lambda)$ is the \cite{calzetti2001} curve: 
\[
    k_{\rm Cal}(\lambda) = 
    \begin{cases} 
        2.659 (-1.857 + 1.040/\lambda) + R_V, & 6300 A \le \lambda \le
        22000 A \\ 
        2.659 (-2.156 + 1.509/\lambda - 0.198/\lambda^2 + 0.011/\lambda^3) +
        R_V & 1200 A \le \lambda \le 6300 A
    \end{cases}
\]
where $\lambda_V = 5500 A$ is the $V$ band wavelength and $\delta$ is the slope
offset of the attenuation curve from $k_{\rm Cal}$. Since $\delta$ correlates 
with galaxy properties~\citep[\eg][]{leja2017, salim2018},
we parameterize $\delta$ with a similar $M_*$ and $\ssfr$ dependence as
$\tau_V$:  
\begin{align} \label{eq:delta}
    \delta(M_*, \sfr) &= \mdeltam \log \left(\frac{M_*}{10^{10}
    M_\odot}\right) + \mdeltas \log \left(\frac{\ssfr}{10^{-10}yr^{-1}}\right)
    + c_\delta.
\end{align}
% Although a number of works have found correlation between the attenuation
% curve slope and inclination~\citep{wild2011, chevallard2013, battisti2017b},
% \cite{salim2020}, most recently, found that the driver of this trend is the
% relationship between $A_V$ and slope. We therefore do not include an
% inclination dependence in $\delta$. 
$D(\lambda)$ in Eq.~\ref{eq:noll} is the UV dust bump, which we parameter using
the standard Lorentzian-like Drude profile:
\begin{equation}
    D(\lambda) = \frac{E_b(\lambda~\Delta \lambda)^2}{(\lambda^2 -
    \lambda_0^2)^2 + (\lambda~\Delta \lambda)^2}
\end{equation}
where $\lambda_0$, $\Delta \lambda$, and $E_b$ are the central wavelength,
full width at half maximum, and strength of the bump, respectively. We assume
fixed $\lambda_0 = 2175
A$ and $\Delta \lambda = 350A$. \cite{kriek2013} and \cite{tress2018} find
that $E_b$ correlates with the $\delta$ for star-forming galaxies $z\sim2$.
\cite{narayanan2018} confirmed this dependence in simulations. 
\chedit{
    For our \eda~prescription, we include the UV dust bump since we use UV
    colors as one of our observables.
}
However, we assume a fixed relation between $E_B$ and $\delta$ from
\cite{kriek2013}: $E_b = -1.9~\delta + 0.85$. Allowing the slope and amplitude
of the $E_B$ and $\delta$ relation to vary does {\em not} impact our results;
however, we do not derive any meaningful constraints on them either. In
Table~\ref{tab:free_param}, we list and describe all of the free parameters in
the \eda. 

% motivation for the M* and SSFR dependence 
\todo{@tks: motivation for $M_*$ and $\ssfr$ dependence}
\tksedit{


\begin{figure}
\begin{center}
    \includegraphics[width=\textwidth]{figs/gswlc_Av_mstar_ssfr_dependence.pdf}
    \caption{\label{fig:dep}
    Dependence of the V-band attenuation on stellar mass (left, for $-11 <$ log(sSFR [yr$^{-1}$]) $< -10.5$ (purple),   
    $-10.5 <$ log(sSFR [yr$^{-1}$]) $< -10$ (red), and  $-10 <$ log(sSFR [yr$^{-1}$]) (orange)),
    and sSFR (right, for  $9.5 <$ log($M_* [M_{\odot}]) < 10.5$ (blue) and $10.5 <$ log($M_* [M_{\odot}]) < 11.5$ (green)) for the GALEX-SDSS-WISE Legacy Catalog (GSWLC)\footnote{\url{https://salims.pages.iu.edu/gswlc/}} \citep{salim2016}. 
    At lower stellar masses the $A_V$ and $M_*$ are more strongly correlated than $A_V$ and sSFR, while the opposite is true at higher stellar masses.
    The correlation between $A_V$ and sSFR is most noticeable at intermediate sSFR (${\sim}-11 <$ log(sSFR [yr$^{-1}$]) $< {\sim}-10$), 
    where there is no apparent correlation with stellar mass. At higher sSFR (e.g. for galaxies on the SFS) the correlation with stellar mass is stronger.
    \tks{not sure we want to include this figure because it is confusing that it uses different data. We could also move it to another appendix if we want?}
    }
\end{center}
\end{figure}
}
%In $\tau_V$ we include the correlation between $A_V$ and the galaxy's properties , found in both observations and simulations~\citep[\eg][]{narayanan2018, salim2020}. 


$\ssfr$ of galaxies are used to calculate $\tau_V$ and $\delta$ in
Eqs.~\ref{eq:tauv} and~\ref{eq:delta}. However, due to mass and temporal resolutions,
some galaxies in the simulations have $\sfr=0$ --- \ie~an unmeasurably low
SFR~\citep{hahn2019c}. They account for 17, 19, 9\% of galaxies
in SIMBA, TNG, and EAGLE, respectively. Since Eqs.~\ref{eq:tauv}
and~\ref{eq:delta} depend on $\log\ssfr$, they cannot be used in the equations
to derive $\tau_V$ and $\delta$ for these galaxies. To account for this issue,
we assign $\sfr_{\rm min}$, the minimum non-zero $\sfr$ in the simulations, to
$\sfr=0$ galaxies when calculating $\tau_V$ and $\delta$. For SIMBA, TNG, and
EAGLE, $\sfr_{\rm min}=0.000816$, $0.000268$, and $0.000707 M_\odot/yr$. Although 
this assumes that $\sfr=0$ galaxies have similar dust properties as the galaxies 
with $\sfr = \sfr_{\rm min}$, since the simulations have very low $\sfr_{\rm min}$ 
we expect galaxies with $\sfr = \sfr_{\rm min}$ to have little recent
star-formation and low gas mass, similar to $\sfr=0$ galaxies. 

%Since $\sfr=0$ galaxies do not account for a large fraction of our simulated galaxies, we directly sample their observables ($G, R, NUV$, and $FUV$) from the distribution of observables for SDSS quiescent galaxies. This way, we ensure that the attenuation of $\sfr=0$ galaxies does not impact the rest of the \eda~parameters. In Appendix~\ref{sec:res}, we discuss the resolution effects in more detail and demonstrate that our results are \emph{not} impacted by other prescriptions for attenuating $\sfr=0$ galaxies.

\chedit{
    In practice, to apply the \eda~to a simulated galaxy population, we first
    assign a randomly sampled inclincation, $i$, to each galaxy ($\cos i$
    uniformly sampled from 0 to 1).  $\tau_V$ and $\delta$ are calculated for
    the galaxy based on its $M_*$,
    $\ssfr$ and the \eda~parameters. We then determine $A_V$ from $i$ and
    $\tau_V$ and $k(\lambda)$ from $\delta$, which combined gives us
    $A(\lambda)$ for each galaxy.
} 
Afterwards, we attenuate the galaxy SEDs using Eq.~\ref{eq:full_atten} and use
the attenuated SEDs to calculate the observables: $g, r, NUV$, and $FUV$
absolute magnitudes. In Figure~\ref{fig:dem_av}, we present attenuation
curves,
$A(\lambda)$, generated by the \eda~for galaxies with different $\sfr$ and $M_*$: 
star-forming ($\sfr=10^{0.5}M_\odot/yr$) with low mass ($10^{10}M_\odot$;
blue), with high mass ($10^{11}M_\odot$; green) and quiescent
($\sfr=10^{-2}M_\odot/yr$) with low mass ($10^{10}M_\odot$; orange), with high
mass ($10^{11}M_\odot$; red). All galaxies are edge-on (\ie~$i=0$) and we use
\eda~parameters: $\{\mtaum, \mtaus, c_\tau, \mdeltam, \mdeltas, c_\delta\} =
\{2., -2., 2., -0.1, -0.1, -0.2\}$, arbitrarily chosen within the prior range
listed in Table~\ref{tab:free_param}. For comparison, we include the
\cite{calzetti2001} attenuation curve. Even for only edge-on galaxies, the
\eda~produces attenuation curves with a wide range of amplitude and
slope to galaxies based on their physical properties. 

\begin{figure}
\begin{center}
    \includegraphics[width=0.6\textwidth]{figs/dems.pdf}
    \caption{\label{fig:dem_av}
    \chedit{
        Attenuation curves, $A(\lambda)$, assigned by our Empirical Dust
        Attenuation (\eda) prescription to edge-on galaxies with different $\sfr$ and
        $M_*$ for an arbitrary set of \eda~parameters. We include the
        \eda~$A(\lambda)$ for star-forming galaxies ($\sfr=10^{0.5}M_\odot/yr$)
        with $M_* = 10^{10}M_\odot$ (blue) and $10^{11}M_\odot$ (green) and
        quiescent galaxies ($\sfr=10^{-2}M_\odot/yr$) with $M_* =
        10^{10}M_\odot$ (orange) and $10^{11}M_\odot$ (red). We set $i=0$ for
        all the galaxies in the figure for simplicity but in practice the
        \eda~uniformly samples $\cos i$ from 0 to 1 for each galaxy.
        For comparison, we include the \cite{calzetti2001} attenuation curve.
        {\em The \eda~provides a flexible prescription for assigning dust
        attenuation to galaxies based on their inclination, physical properties
        ($M_*$ and $\ssfr$), and the \eda~parameters.}
    }
    } 
\end{center}
\end{figure}


%%%%%%%%%%%%%%%%%%%%%%%%%%%%%%%%%%%%%%%%%%
% table of free parameters
%%%%%%%%%%%%%%%%%%%%%%%%%%%%%%%%%%%%%%%%%%
\begin{table}
    \caption{Free parameters of the Empirical Dust Attenuation Model}
    \begin{center}
        \begin{tabular}{ccc} \toprule
            Parameter & Definition & prior\\[3pt] \hline\hline
            %\multicolumn{3}{c}{DEM with slab model}\\ \hline
            $\mtaum$ & $M_*$ dependence of the optical depth, $\tau_V$ & flat $[-5., 5.]$\\
            $\mtaus$ & $\ssfr$ dependence of $\tau_V$  & flat $[-5., 5.]$\\
            $c_{\tau}$ & amplitude of $\tau_V$ & flat $[0., 6.]$\\
            %\hline
            %\multicolumn{3}{c}{DEM with $\mathcal{N}_T$ model}\\ \hline
            %$m_{\mu,1}$ & Slope of the $\log M_*$ dependence of optical depth,
            %$\tau_V$ & flat $[-5., 5.]$\\
            %$m_{\mu,2}$ & Slope of the $\log {\rm SFR}$ dependence of optical
            %depth, $\tau_V$ & flat $[-5., 5.]$\\
            %$c_{\mu}$ & amplitude of the optical depth, $\tau_V$ & flat $[0., 6.]$\\ 
            %$m_{\sigma,1}$ & Slope of the $\log M_*$ dependence of optical depth, $\tau_V$ & flat $[-5., 5.]$\\
            %$m_{\sigma,2}$ & Slope of the $\log {\rm SFR}$ dependence of optical depth, $\tau_V$ & flat $[-5., 5.]$\\
            %$c_{\sigma}$ & amplitude of the optical depth, $\tau_V$ & flat $[0.1, 3.]$\\ 
            %\hline
            $\mdeltam$ & $M_*$ dependence of $\delta$, the attenuation curve slope offset & flat $[-4., 4.]$\\
            $\mdeltas$ & $\ssfr$ dependence of $\delta$ & flat $[-4., 4.]$\\
            $c_{\delta}$ & amplitude of $\delta$ & flat $[-4., 4.]$\\
            %$f_{\rm neb}$ & nebular attenuation fraction & flat $[1., 4.]$\\
            \hline
        \end{tabular} \label{tab:free_param}
    \end{center}
\end{table}
%%%%%%%%%%%%%%%%%%%%%%%%%%%%%%%%%%%%%%%%%%



\section{Likelihood-Free Inference: Approximate Bayesian Computation} \label{sec:abc}
\chedit{
    With our forward model, which includes the \eda~prescription for dust
    attenuation, we can now generate synthetic observations for simulated
    galaxies and make an ``apples-to-apples'' comparison to SDSS. Next, we want
    to use this comparison to infer the posterior probability distribution of
    the \eda~parameters. Typically in astronomy, this inference is done
    assuming a Gaussian likelihood to compare the ``summary statistic''
    (\eg~SMF) of the model to observations and some sampling method (\eg~Markov
    Chain Monte Carlo) to estimate the posterior distribution. The functional form of the
    likelihood, however, depends on the summary statistic and assuming an
    incorrect form of the likelihood can significantly bias the inferred
    posteriors~\citep[\eg][]{hahn2019}. In this work, we use the optical and UV
    color-magnitude relations as our summary statistic. Since the statistic is
    three-dimensional histogram, its likelihood is {\em not} Gaussian but
    rather a Poisson distribution.
}

\chedit{
    Rather than \emph{incorrectly} assuming a Gaussian likelihood or attempting
    to estimate the true Poisson likelihood of the optical and UV
    color-magnitude relations, we use Approximate Bayesian Computation~\citep[hereafter
    ABC;][]{diggle1984, tavare1997, pritchard1999, beaumont2009, delmoral2012}
    for our inference. 
}
ABC is a likelihood-free (or ``simulation-based'') parameter inference
framework that approximates the posterior probability distribution, $p(\theta\given{\rm data})$, without
requiring evaluations of the likelihood.  Instead, ABC only requires a forward
model of the observed data, a prior that can be sampled, and a distance metric
that quantifies the ``closeness'' to the observed data. 
\chedit{
    Since ABC does not require evaluating the likelihood, it does not assume
    any functional form of the likelihood and so we avoid any biases from such
    assumptions. Furthermore, it also allows us to infer the posterior using
    summary statistics with likelihoods that are difficult or intractable to
    directly estimate~\citep{hahn2017a}.
}

In the simplest version of ABC, with a rejection sample
framework~\citep{pritchard1999}, a proposal set of parameter values are drawn
from the prior. The forward model is run with the proposal parameter values.
\chedit{
    The output of the forward model is then compared to the observed data using
    a distance metric that quantifies the ``closeness'' of the forward model
    output to the observed data. 
}
If the distance is within some small threshold, we keep the proposed
parameters; otherwise, we discard them.  Proposals are
drawn until enough pass the threshold to sample the posterior. A
rejection sampling framework requires a large number of evaluations of the
forward model, which
can be computationally costly. Many variations of ABC with more efficient
sampling strategies have now been applied to astronomy and
cosmology~\citep[\eg][]{cameron2012, weyant2013, ishida2015, lin2016, alsing2018}.
Among these methods, we use ABC with Population Monte Carlo (PMC) 
importance sampling~\citep{hahn2017a, hahn2017b, hahn2019a}.

ABC-PMC begins with an arbitrarily large threshold $\epsilon_1$ and $N$ proposals 
$\bar{\theta}_1$ sampled from the prior distribution. Each proposal is
assigned a weight $w^i_1 = 1/N$. Then for subsequent iterations ($n > 1$), the 
threshold, $\epsilon_n$, is set to the median distance of the previous iteration's
proposals. New proposals are drawn from the previous iteration's proposals perturbed 
by a kernel and kept if their distance is blelow $\epsilon_n$. This is repeated
until we assemble a new set of $N$ proposals $\bar{\theta}_n$. The entire
process is repeated for the next iteration until convergence is confirmed. 
We use the Python implementation of
\cite{akeret2015}\footnote{https://abcpmc.readthedocs.io/en/latest/index.html}.
For further details on the ABC-PMC implementation, we refer readers to \cite{hahn2017b}
and \cite{hahn2019a}.

\begin{figure}
\begin{center}
    \includegraphics[width=\textwidth]{figs/abc.pdf}
    \caption{\label{fig:abc}
    Posterior distributions of the \eda~parameters for the SIMBA (orange), TNG
    (blue), and EAGLE (green) hydrodynamical simulations. The contours mark the $68$
    and $95$ percentiles of the distributions. The posteriors are derived using the
    likelihood-free inference method: Approximate Bayesian Computation with
    Population Monte Carlo (Section~\ref{sec:abc}). We focus on the
    \eda~posteriors for TNG and EAGLE since the \eda~struggles to reproduce
    SDSS observations with SIMBA, which predicts an overabundance of starburst
    galaxies. Based on the posteriors, we find that \emph{galaxies with higher
    $M_*$ have overall higher dust attenuation and galaxies with higher $\ssfr$
    have steeper attenuation curves.}
    }
\end{center}
\end{figure}

\chedit{
    In this work, we use ABC-PMC with uninformative uniform priors on each of
    the \eda~parameters and choose ranges that encompass constraints in the
    literature.
}
The prior ranges of $\mtaum, \mtaus, c_\tau$ are chosen to
conservatively include the $A_V$ range and $M_*$ and $\sfr$ dependence of
\cite{narayanan2018} and \cite{salim2020}. Meanwhile, the prior ranges of 
$\mdeltam, \mdeltas, c_\delta$ are chosen to conservatively include the $\delta$
range and $M_*$ and $\sfr$ dependence of \cite{leja2017} and \cite{salim2018}. 
We list the range of the priors in Table~\ref{tab:free_param}. 
\chedit{
    For our forward model, we use the model described in
    Section~\ref{sec:fm}: we construct SEDs for every simulated galaxies from
    the hydrodynamic simulations, apply dust attenuation with our
    \eda, calculate the observables ($M_r$, $\gr$, and $\fnuv$),
    add uncertainties to them, and apply a $M_r < -20$ completeness 
    limit. 
    We use the optical and UV color-magnitude relation, $(\gr)- M_r$ and
    $(\fnuv)-M_r$ as our summary statistic to fully exploit the $(M_r, \gr,
    \fnuv)$ observational-space. We measure the color-magnitude relations by
    calculating the number density in bins of $(\gr, \fnuv, M_r)$ with widths
    $(0.0625, 0.25, 0.5)$ mags. For our distance metric, $\rho$, we use the L2
    norm between the summary statistics of 
    the SDSS observation, $X^{\rm SDSS}$ and of our forward model, $X^{\rm
    FM}(\theta_{\rm \eda})$: 
}
\begin{equation} \label{eq:distance}
    \rho(\theta_{\rm \eda}) = \left[X^{\rm SDSS} - X^{\rm FM}(\theta_{\rm
    \eda}) \right]^2.
\end{equation}
In Figure~\ref{fig:abc}, we present the posterior distributions of the \eda~parameters
derived using ABC-PMC for the SIMBA (orange), TNG (blue), and EAGLE (green) hydrodynamical 
simulations. The contours mark the $68$ and $95$ percentiles of the distributions. 

% --- results ---  
\begin{figure}
\begin{center}
    \includegraphics[width=0.9\textwidth]{figs/abc_observables.pdf}
    \caption{\label{fig:dem}
    \chedit{
        The optical ($(\gr) - M_r$; top) and UV ($(\fnuv) - M_r$; bottom
        panels) color-magnitude relations predicted by our \eda~prescription
        for the SIMBA (orange), TNG (blue), and EAGLE (green) hydrodynamical
        simulations. For the \eda~parameters of each simulation, we use the
        median of the posterior distributions inferred using ABC. For
        comparison, we include the color-magnitude relations of SDSS (black
        dashed). Comparing the color-magnitude relations above to those without
        dust attenuation in Figure~\ref{fig:obs}, we see that dust
        \emph{dramatically} impacts the color-magnitude relations. Therefore,
        dust attenuation must be accounted for when interpreting and comparing
        simulations. Furthermore, with our \eda~prescription, all three
        simulations reproduce the color-magnitude relations of SDSS
        observations.  \emph{Since the different simulations can reproduce
        observations just by varying dust, dust significantly limits our ability to
        constrain the underlying physical processes of galaxy formation
        models.}
    }
    }
\end{center}
\end{figure}

\section{Results} \label{sec:results}
% dust model can reproduce color magnitude relation
Without dust attenuation, all of the hydrodynamical simulations struggle to 
reproduce the $(\gr) - M_r$ and $(\fnuv) - M_r$ relations of SDSS (Figure~\ref{fig:obs}). 
In the optical, the simulations predict significantly bluer colors and find  
broader differences in color, ${\sim}0.5~mag$, between star-forming and
quiescent galaxies. In the UV, they predict quiescent galaxies with $\ssfr <
10^{-12}yr^{-1}$ that have redder $\fnuv$ colors beyond SDSS observations. 
Meanwhile, for the rest of the galaxies, the simulations predict significantly
bluer UV colors than SDSS. 

\emph{With our \eda~prescription, however, all three simulations reproduce the
color-magnitude relations of SDSS observations.} In Figure~\ref{fig:dem}, 
we present the optical and UV color-magnitude relations predicted by the 
\eda~for the SIMBA (orange), TNG (blue), and EAGLE (green) simulations. 
For the \eda~parameters, we use the median of the posterior distributions 
inferred using ABC (Figure~\ref{fig:abc}). We include the color-magnitude
relations of SDSS observations (black-dashed) comparison. The contours mark 
the $68$ and $95$ percentiles of the distributions. 

Dust dramatically impacts the observables of simulations. The \eda~affects the 
the optical and UV color-magnitude relations in three major ways to produce
good agreement with SDSS. 
First, the \eda~significantly reddens simulated galaxies in both the optical
and UV. Overall, $\gr$ colors are ${\sim}0.25~mag$ redder and $\fnuv$ colors
are ${\sim}0.5~mag$ redder. 
Second, the \eda~significantly attenuates (${\sim}0.5~mag$) intrinsically blue
galaxies (\ie~star-forming galaxies with $\log~{\rm SSFR} > -10.5$). As a
result, there are no luminous optically blue galaxies ($M_r < -21$ and $\gr <
0.5$) --- consistent with observations.
Lastly, in the UV, the \eda~attenuates low SSFR quiescent galaxies that are 
intrinsically red in the UV ($\fnuv{\sim}3~mag$) by ${\sim}1~mag$. Thus, unlike in
Figure~\ref{fig:obs}, the UV color-magnitude relations from the \eda~do not
have a significant number of galaxies with high $\fnuv$, in agreement with SDSS.

% SIMBA discrepancy
\chedit{
    For SIMBA, although the \eda~prescription predicts optical and UV
    color-magnitude relations consistent with observations, there are still
    some discrepancies in the color-magnitude relations with SDSS and the
    overall agreement is worse than TNG and EAGLE.
}
SIMBA+\eda~produces a $\gr$ color distribution that is narrower and a $\fnuv$
distribution that is broader than SDSS. The inferred \eda~parameters for SIMBA
also differ significantly from the parameters of TNG and EAGLE
(Figure~\ref{fig:abc}). 
\chedit{
    These discrepancies are primarily driven by the excess of high
    $\log{\rm SSFR} > -9.5$ ``starburst'' galaxies with $M_*<10^{10}M_\odot$
    that lie significantly above the SFS, which are {\em only} present in SIMBA
    (Figure~\ref{fig:smf_msfr}). This starburst population, which has also been
    identified in \cite{dave2019} (see their Figures~5 and 6) is caused by
    excess gas in low-mass galaxies at $z=0$. If we exclude $\log{\rm SSFR} >
    -9.5 {yr}^{-1}$ starburst galaxies and run the \eda~for SIMBA using median 
    values of the TNG and EAGLE posteriors for the \eda~parameters, we find 
    similar level of agreement with SDSS as TNG and EAGLE. 
}

Without dust attenuation, the excess starburst galaxies in SIMBA are blue
$\gr\sim0.1$ and have high luminosity, $M_r < -22$.
\chedit{
    Since such high luminosity blue galaxies do not exist in observations, they
    need to be both strongly attenuated and reddened. In our \eda~prescription,
    dust attenuation and reddening is a linear function of $\log\ssfr$ so it
    cannot strongly attenuate and redden the starburst population in
    particular. Although we can increase the flexibility of our
    \eda~prescription, the attenuation and reddening necessary for the
    starburst galaxies conflict with the attenuation-slope relation
    well-established in observations and
    simulations~\citep{inoue2005,chevallard2013,salim2018,salim2020,trayford2020}.
    The SIMBA starbursts require both high attenuation and steeper slopes but,
    based oo the attenuation-slope relations, galaxies with higher attenuation
    have shallower slopes. We therefore do not explore extending our
    \eda~prescription and focus on the TNG and EAGLE simulations for the rest
    of the paper. 
}

% comparison to literature 
Previous works in the literature have also compared simulations with different
dust prescriptions to observations in color-magnitude space. For EAGLE, 
\cite{trayford2015} calculate colors and luminosities with the {\sc Galaxev}
population synthesis models and a two-component screen model for dust. More
recently, \cite{trayford2017} calculated optical colors for EAGLE using {\sc
Skirt}, a Monte Carlo radiative transfer code~\citep{camps2015}, to model the
dust. At stellar masses and luminosities comparable to our SDSS sample, both 
\cite{trayford2015} and \cite{trayford2017} produce red sequences bluer than 
in GAMA observations. Also, \cite{trayford2015} predict an excess of luminous 
blue galaxies. Although a detailed comparison is difficult since both works 
compare to different observations, we note that with the \eda, EAGLE is able 
to successfully reproduce the position of the SDSS red sequence and does not 
predict a significant excess of luminous blue galaxies. Also using EAGLE and 
{\sc Skirt}, \cite{baes2019} find that they overesimtate the observed cosmic 
SED (CSED) in the UV regime and produce significantly higher $\fnuv$ color 
than GAMA. The \eda~predicts $\fnuv$ in good agreement with SDSS. 
%\cite{baes2019}: EAGLE+SKIRT SED compparison with GAMA Far UV is not attenuated enough. underrestimates optical and NIR
For TNG, \cite{nelson2018} calculate optical colors using a dust model that
includes attenuation due to dense gas birth clouds surrounding young stellar
populations and also due to simulated distribution of neutral gas and metals.
They find bluer red sequence peaks and a narrower blue cloud compared to SDSS.
We find neither of these discrepancies for the TNG with the~\eda. The
\eda~is a simpler prescription for applying dust attenuation than the dust
models in these works. Yet, with its flexibility, we are able to produce
optical and UV color-magnitude relations that are in good agreement with
observations. Furthermore, with its low computation costs we were able to fully
explore our dust parameters. 
%better agreement than observations than these works.

% without being fit, the EDA reproduces the attenuation-slope relation and SF
% attenuation of SDSS observations  
\begin{figure}
\begin{center}
    %\includegraphics[width=0.9\textwidth]{figs/abc_slope_AV_starforming.pdf}
    \caption{\label{fig:slope}
    The attenuation-slope relation of star-forming galaxies ($\ssfr >
    10^{-11}yr^{-1}$), using the attenuation curves predicted by our  
    \eda~prescription for the median posterior parameter values of SIMBA
    (left), TNG (center) and EAGLE (right). 
    For comparison, we include the observed attenuation-slope relation
    from GSWLC2~\citep{salim2020}. 
    We use $A_V$ and $S = A(1500\AA)/A_V$ as measurements of attenuation and
    slope, respectively. 
    \chedit{
        \emph{The \eda~does not predict $A_V < 0.3$ because star-forming galaxies in
        the simulations are too lumnious and require significant attenuation to
        reproduce observations.}
        Beyond $A_V > 0.3$, however, there is good agreement between the
        attenuation-slope relation predicted by the \eda~and observations. 
    }
    }
\end{center}
\end{figure}

\subsection{Comparison to Dust Observations} \label{sec:reproduce}
With our \eda~prescription, we are able to accurately reproduce the
observed optical and UV color-magnitude relations with our simulations. 
In addition to  reproducing observations, since the \eda~assigns dust
attenuation curves to each simulated galaxy, we can compare the
\eda~attenuation curves to dust attenuation measured from
observations. 
We begin with the well-established attenuation-slope relation: star-forming
galaxies with higher dust attenuation have shallower attenuation curves. 
This relation is a consequence of dust scattering dominating absorption at
low attenuation while dust absorption dominates at high
attenuation~\citep{gordon1994, witt2000, draine2003, chevallard2013}. 
In Figure~\ref{fig:slope}, we present the attenuation-slope relation of
star-forming galaxies with $\ssfr > 10^{-11}yr^{-1}$ based on the
dust attenuation curves predicted by the \eda~for the median posteriors of
SIMBA (left), TNG (center) and EAGLE (right).
For comparison, we include the observed attenuation-slope relations of
GSWLC2 galaxies~\citep[grey shaded;][]{salim2020}.
For attenuation we use $A_V$ and for slope we use the UV-optical slope, $S
= A(1500\AA)/A_V$, commonly found in the literature. 
The contours mark the 68 and 95 percentiles. 
Most noticably, we find that the \eda~does not predict $A_V < 0.3$. 
\chedit{
    This is not due to the selection function imposed by our forward model. 
    The attenuation-slope relation of star-forming galaxies in GSWLC2 does not
    change significantly if we impose similar selection cuts as our
    observational sample.
    Instead, the lack of star-forming galaxies with $A_V < 0.3$ is a
    consequence of SIMBA, TNG, and EAGLE predicting star-forming galaxies that
    are more luminous than observations.  
}
All of the simulations have star-forming galaxies with intrinsic $M_r <
-21$ and $\gr < 0.5$ (Figure~\ref{fig:obs}). 
This is further corroborated by the $\sfr-M*$ relations in
Figure~\ref{fig:smf_msfr}, where the simulations all have star-forming
galaxies with $M_* > 10^{11}M_\odot$, not found in SDSS. 
To reproduce the SDSS optical color-magnitude relation these galaxies would
need to be significantly reddened and attenuated. 
Hence, any dust prescription for the simulations would require high 
$A_V$ for star-forming galaxies.
Nevertheless, for $A_V > 0.3$, we find good agreement between the 
attenuation-slope relation predicted by the \eda~and observations. 

%We note that the difference in the $A_V$ ranges is due to the $M_r$
%completeness limit imposed by our forward model (Section~\ref{sec:fm}).
%The GSWLC2 sample in \cite{salim2020} extends down to $M_* \sim
%10^{8.5}M_\odot$; however, the TNG and EAGLE samples do not extend below
%$M_* \sim10^{10}M_\odot$.
%\emph{The \eda~predicts attenuation-slope relations for TNG and EAGLE that
%are in excellent agreement with observations.} 

\begin{figure}
\begin{center}
    \includegraphics[width=0.5\textwidth]{figs/abc_sf_attenuation.pdf}
    \caption{\label{fig:sfatten}
    The normalized attenuation curves of star-forming galaxies predicted by
    the \eda~for median posterior parameter values of SIMBA (orange), TNG
    (blue), and EAGLE (green).  
    Galaxies with $\log \ssfr > -11~yr^{-1}$ are classified as star-forming. 
    The attenuation curves are normalized at $3000\AA$ and we mark the
    68 percentile of the attenuation curves with the shaded region.
    For comparison, we include $A(\lambda)/A(3000\AA)$ measurements from
    the~\cite{narayanan2018} radiative transfer simulation (dashed) and
    \cite{salim2018} observations (dotted).
    %The \cite{calzetti2000} and \cite{battisti2017} attenuation curves are shallower than the \eda~attenuation curves; however, they probe lower $M_*$ galaxies than our forward modeled TNG and EAGLE samples.  For attenuation curve from \cite{salim2018}, which probe a similar $M_*$ range, we find goood agreement. 
    {\em The \eda~predict attenuation curves of star-forming galaxies that
    are in good agreement with the attenuation curves measured from
    the simulation and observations in the literature.}
    %We also find good agreement with median attenuation curve of star-forming galaxies in the radiative transfer simulations of \cite{narayanan2018}.
    }
\end{center}
\end{figure}

In addition to the attenuation-slope relation, we can also directly compare
the attenuation curves predicted by the \eda~to measurements from
observations for star-forming galaxies. 
In Figure~\ref{fig:sfatten}, we present the normalized attenuation curves
of star-forming galaxies predicted by the \eda~for the median posterior
parameter values of SIMBA(orange), TNG (blue), and EAGLE (green).
We again define galaxies with $\ssfr > 10^{-11}{yr}^{-1}$ as star-forming.
The attenuation curves are normalized at $3000\AA$ and we present the
variation in the attenuation curves in the shaded region, 68 percentile. 
For comparison, we include $A(\lambda)/A(3000\AA)$ from the
\cite{narayanan2018} radiative transfer simulation (dashed) and 
observations~\citep[][dotted]{salim2018}. 
The attenuation curve from \cite{salim2018} corresponds to star-forming
galaxies with $M_* > 10^{10.5}M_\odot$, a similar $M_*$ range as our
forward modeled samples. 
Since we do not vary the UV bump in our \eda~prescription, we ignore any
discrepancies in the amplitudes of the bump. 
\emph{Overall, we find good agreement between the \eda~attenuation curves for
star-forming galaxies and the attenuation curves from observations and
simulations in the literature.}

%Again, the fact that we reproduce the detailed dust attenuation curves of star-forming galaxies in observations and simulations with the \eda~without fitting for them, highlights the advantages of a forward modeling approach. 

%The \eda~attenuation curves are slightly steeper than the \cite{calzetti2000} and \cite{battisti2017} curves. 
%These attenuation curves, however, are derived from $M_* < 10^{9.9}M_\odot$
%star-forming galaxies, which lie below the $M_*$ limit of our forward
%modeled TNG and EAGLE samples. 
%Meanwhile, the TNG and EAGLE \eda~attenuation curves are in good agreement
%with the \cite{salim2018} attenuation curve for $M_* > 10^{10.5}M_\odot$ star-forming galaxies. 
%They are also consistent with the median curve of \cite{narayanan2018}. 

%The \eda~attenuation curves are noticeably steeper than the \cite{calzetti2000} and \cite{battisti2017} curves.  These attenuation curves, however, are derived from $M_* < 10^{9.9}M_\odot$ star-forming galaxies --- below our $M_*$ range. 
%Since we find $\mdeltam < 0$ for both the TNG and EAGLE posteriors, the \eda~attenuation curves are consistent with \cite{calzetti2000} and \cite{battisti2017}. 


%\chedit{ 
%    The \eda~predicts higher dust attenuation at lower wavelenghts for
%    star-forming galaxies.
%    Without dust attenuation, both TNG and EAGLE predict star-forming galaxies
%    that are bluer in the optical and UV than observations
%    (Figure~\ref{fig:obs}).
%    To reproduce the SDSS, the \eda~significantly reddens star-forming galaxies.
%}
%In Figure~\ref{fig:raw_atten}, we also find that more massive star-forming
%galaxies have higher attenuation. This is because the simulations overpredict 
%luminous blue star-forming galaxies, which must be attenuated to reproduce
%observations. 


%At low attenuation, dust scattering dominates absoprtion so the 
%attenuation curve steepens because red light scatters isotropically while blue light
%scatters forward~\citep{gordon1994, witt2000, draine2003}. %, which causes more optical-to-IR light to escape the galaxy than UV light
%At high attenuation dust absorption is dominant and the attenuation curve is
%shallower~\citep{chevallard2013}. For the $A_V$ range probed by the DEM, the
%$A_V$--slope relation is in good agreement with GSWLC2 galaxies~\citep[black shaded][]{salim2020}.
%They are also consistent with \cite{leja2017}. We also compare our results to
%theoretical predictions from radiative transfer models, \cite{inoue2005}
%(dotted), the radiative transfer models considered in \cite{chevallard2013}
%(dot dashed), and \cite{trayford2020} (light shaded), which all predict shallower 
%attenuation curves than observations. This is also the case for the
%\cite{narayanan2018} attenuation curves (not included). 
%\emph{The attenuation curve slopes from the DEM for are in excellent
%agreement with observations and better reproduces the observed
%attenuation--slope relation than radiative transfer models.}
 

\begin{figure}
\begin{center}
    \includegraphics[width=0.85\textwidth]{figs/abc_attenuation_unormalized.pdf}
    \caption{\label{fig:raw_atten}
    Same as Figure~\ref{fig:atten} but without normalizing the attenuation
    curves at 3000$\AA$. Based on \eda, quiescent galaxies have significant 
    dust attenuation. Compared to star-forming galaxies, they have 
    significantly lower attenuation in the UV and significantly shallower 
    attenuation curves overall.
    }
\end{center}
\end{figure}
\subsection{The Attenuation Curves of Quiescent Galaxies}  
We have demonstrated so far that the \eda~is able reproduce the observed UV and
optical color-magnitude relations and also predict dust attenuation curves
of star-forming galaxies consistent with observations and radiative transfer 
simulations. In addition, the \eda~also predicts dust attenuation curves of
quiescent galaxies. This is particularly valuable since there are many challenges 
to measuring attenuation curves for quiescent galaxies directly from observations. 
Methods that rely on IR luminosities can be contaminated by MIR emission from AGN
heating nearby dust~\cite{kirkpatrick2015}. Even SED fitting methods require 
accounting for AGN MIR emission~\citep{salim2016, leja2018, salim2018}. SED 
fitting methods also struggle to tightly constrain dust attenuation for quiescent 
galaxies since they are limited by the degeneracies with star formation history and 
metallicity.

With a forward modeling approach, we circumvent these challenges. We derive the 
attenuation curves necessary for quiescent galaxy population in simulations to
reproduce the observed optical and UV photometry.  In right panels of
Figure~\ref{fig:atten}, we present the attenuation curves of quiescent galaxies 
predicted by the \eda~model for the median posterior parameter values of TNG (blue) 
and EAGLE (green). In the top and bottom panels, we present galaxies with 
$\log M_* < 10^{10.5} M_\odot$ and $\log M_* > 10^{10.5} M_\odot$, respectively. 
We define galaxies with $\log {\rm SSFR} < -11$ as quiescent. The right panels 
of Figure~\ref{fig:raw_atten} are the same as in Figure~\ref{fig:atten}, except
the attenuation curves in Figure~\ref{fig:raw_atten} are not normalized. 

For both TNG and EAGLE posteriors, the \eda~predicts significant dust attenuation 
in quiescent galaxies. Compared to star-forming galaxies, however, they have 
lower attenuation in the UV and much shallower attenuation curves. The amplitude 
of the \eda~quiescent galaxy attenuation curves is driven by the fact that both
TNG and EAGLE --- without dust --- predict quiescent galaxies that are too 
luminous compared to observations. Hence, significant attenuation is necessary 
to lower their luminosity. Meanwhile, the shallow slope is driven by the
simulations predicting quiescent galaxies that are bluer in the optical but
redder in the UV than observations. The \eda~optically reddens the quiescent
galaxies but maintains a shallow enough slope to reproduce the UV
color-magnitude relation. This is also why TNG has a shallower slope than
EAGLE: TNG has an optically redder quiescent population and more quiescent
galaxies with high $\fnuv$ color. 

Given the challenges in observationally measuring attenuation curves of quiescent
galaxies, the predictions of the bestfit \eda~models for TNG and EAGLE
highlight the advantages of a forward modeling approach and provide valuable 
insights into dust attenuation in quiescent galaxies. \emph{Quiescent galaxies
have significant UV and optical attenuation with shallow attenuation curves.} 

 


\begin{figure}
\begin{center}
    \includegraphics[width=0.9\textwidth]{figs/abc_av_mssfr.pdf}
    \caption{\label{fig:avmsfr}
    \chedit{ 
        $M_*$ and $\ssfr$ dependence of dust attenuation at $1500 \AA$, $A_{1500}$,
        and at $5500\AA$, $A_{V}$.
    }
    }
\end{center}
\end{figure}

\subsection{The Galaxy -- Dust Connection}  
\chedit{
    With the \eda~framework, we can also shed light on the connection between
    the physical properties of galaxies and dust attenuations.
    In our \eda~prescription, we included a flexible $M_*$ and $\ssfr$ dependence in both the
    amplitude and slope of the attenuation curve~(Eqs~\ref{eq:tauv}
    and~\ref{eq:delta}. 
    Hence, we can reveal the $M_*$ and $\ssfr$ dependence of dust attenuation
    through the \eda~parameter constraints (Figure~\ref{fig:abc}) and the
    predicted attenuation curves. 
}


\chedit{
    Focusing first on the amplitude of dust attenuation, we find that TNG has
    little $M_*$ dependence in $\tau_V$: 
    $\mtaum = 0.14\substack{+0.64 \\ -0.58}$. 
    EAGLE has a more significant positive $M_*$ dependence:
    $\mtaum = 0.53\substack{+0.36 \\ -0.36}$.
    Though neither TNG nor EAGLE has a strong dependence, $V$-band dust
    attenuation is higher for more massive galaxies.  
    Meanwhile, we find significant $\ssfr$ dependences in both TNG 
    ($\mtaus = -0.42\substack{+0.2 \\ -0.18}$)
    and EAGLE
    ($\mtaus = -0.24\substack{+0.22 \\ -0.19}$): galaxies with higher $\ssfr$
    have lower $V$-band dust attenuation. 
    For the slope of the dust attenuation, we find significant $M_*$
    dependence in both TNG 
    ($\mdeltam = -0.36\substack{+0.23\\-0.19}$)
    and EAGLE
    ($\mdeltam = -0.2\substack{+0.16\\-0.16}$). 
    More massive galaxies have steeper attenuation curves. 
    We also find strong $\ssfr$ dependence in both TNG 
    ($\mdeltas = -0.55\substack{+0.08 \\ -0.08}$)
    and EAGLE
    ($\mdeltas = -0.43\substack{+0.08 \\ -0.08}$).
    Galaxies with higher $\ssfr$ have steeper attenuation curves.

}

\chedit{
    Next, we take a closer look at the $M_*$ and $\ssfr$ dependence of the
    attenuation curve in Figure~\ref{fig:avmsfr}. We present dust attenuation
    at $1500\AA$ ($A_{1500}$; top) and $5500\AA$ ($A_V$; bottom) as a function
    of $\log M_*$ and $\log {\rm SFR}$ predicted by the \eda~for TNG (left) and
    EAGLE (right). For each hexagonal bin, the colormap represents the median
    attenuation for all simulated galaxies in the bin. We only include
    $A_{1500}$ or $A_V$ values for galaxies that satisfy our $M_r < -20$
    completeness limit. The bottom panels of Figure~\ref{fig:avmsfr}
    corroborate our conclusions on $V$-band dust attenuation from the
    \eda~parameter posteriors (Figure~\ref{fig:abc}): $A_V$ has a slight $M_*$
    dependence but a significant $\ssfr$ dependence. 
    Furthermore, by comparing the top to the bottom panels, we can confirm the
    $M_*$ and $\ssfr$ dependence of the  attenuation curve slopes. 
}


The $M_*$ dependence in $\tau_V$ and $\delta$
are further illustrated in the attenuation curves in Figure~\ref{fig:raw_atten}.
For TNG, there is little difference in the $V$-band attenuation between the 
top and bottom panels. However, more massive galaxies have steeper slopes 
with similar $A_V$ so they have significantly higher UV attenuation. For
EAGLE, more massive galaxies have both higher $V$-band and UV attenuation.
Overall, we find that more massive galaxies have higher attenuation,
which is consistent with the literature. \cite{burgarella2005}, for instance,
found significant positive $M_*$ dependence in $FUV$ attenuation in
NUV-selected and FIR-selected samples. \cite{garn2010} and \cite{battisti2016}
also find higher attenuation in more massive SDSS star-forming galaxies. Most
recently, \cite{salim2018} find higher $V$ and $FUV$ attenuation for more
masssive star-forming galaxies in GSWLC2. 

From the attenuation curves in
Figure~\ref{fig:raw_atten}, we similarly find that star-forming galaxies have
attenuation curves with slightly lower $A_V$ but substantially higher UV 
attenuation. Quiescent galaxies, on the other hand, have significantly 
shallower attenuation curves. Although observations have examined the
SSFR dependence of dust attenuation, they cannot be compared to our findings since
they focus only on star-forming galaxies, due to the difficulty in
observationally constraining the attenuation curve in quiescent
galaxies~\citep[\eg][]{garn2010, reddy2015, battisti2016, battisti2017, salim2018}. 
In summary, from the bestfit \eda~models of TNG and EAGLE, we find that
\emph{galaxies with higher $M_*$ have overall higher dust attenuation and
galaxies with higher SSFR have steeper attenuation curves}.


\subsection{Discussion}  
We make a number of assumptions and choices in our \eda~prescription. 
\ch{
    First, we use the slab model (Eq.~\ref{eq:slab}) to assign $A_V$ as a
    function of the randomly sampled $i$ and $\tau_V$. 
    This choice is based on the fact that the slab model reproduces the 
    correlation between attenuations and inclination found in 
    observations~\citep{conroy2010b, wild2011, battisti2017, salim2020} as well
    as simulations~\citep[\eg][]{chevallard2013, narayanan2018, trayford2020}.
}
It can also reproduce the SDSS $A_V$ distribution (Figure~\ref{fig:av_dist}). If
we replace the slab model with a more flexible model for sampling $A_V$ using
truncated normal distributions, we find that our results are not significantly
impacted (see Appendix~\ref{sec:slab} for details). Therefore, we conclude
that our results do not sigificantly depend on our choice of the slab model. 
\ch{
    In our \eda, we also use a parameterization of $\tau_V$ and $\delta$ that
    depend linearly on $\log M_*$ and $\log {\rm SSFR}$. 
    While the $M_*$ and $\ssfr$ dependence of $A_V$ is well-motivated and is
    found in, for instance, the \cite{salim2018} GSWLC2 catalog (Appendix~\ref{sec:slab}), 
    the linear dependence was chosen primarily for its simplicity.
    The \eda~framework can be easily extended to more flexible
    parameterizations. 
}
In fact, a more flexible parameterization would likely reduce 
some of the discrepancies with the SDSS color-magnitude relations. The
\eda~produces broader distributions of optical colors than SDSS. Few galaxies
in SDSS have $\gr > 1.$ while some galaxies in the \eda~broadly extend beyond
this cut-off. In the UV, the \eda~struggles to accurately reproduce the redder
portions ($\fnuv > 1.5$) of the UV color-magnitude relation. The main
challenges for a more flexible parameterization would be model selection and
finding a well-motivated parameterization. 
\ch{
    Notwithstanding, for SDSS observations, our \eda~prescription using
    parameter values from the TNG and EAGLE posteriors find good agreement.
}

\ch{
    We demonstrate in this work that accounting for dust attenuation is
    essential when comparing simulations to observations. 
}
None of the simulations reproduce the UV and optical color-magnitude relation
without dust attenuation (Figure~\ref{fig:obs}). 
\ch{ 
    Furthermore, the fact that we can use the \eda~to reproduce SDSS observations 
    for different hydrodynamical simulations highlights how our current lack of 
    understanding of dust limits our ability to closely compare galaxy
    formation models. 
    Our \eda~prescription is built on our current understanding of dust
    attenuation in galaxies: \eg~the \citealt{noll2009} parameterization, the
    UV bump, the slab model, etc.
}
Yet with the \eda, two simulations that predict galaxy populations with
significantly different physical properties (Figure~\ref{fig:smf_msfr}) can
reproduce the same SDSS observations. 
This suggests that dust is highly degenerate with the differences between simulations. 
Put another way --- if we were to marginalize over dust in our comparison to observations, we would not
be able to differentiate between the different galaxy physics prescriptions in
the simulations. 
Hence, current limitations in our understanding of dust is 
a major bottleneck for investigating galaxy formation using simulations.
\ch{
    In the next paper of the series, Starkenburg et al. (in preparation), we
    will examine whether we can compare the prescriptions for star formation
    quenching in different galaxy formation models once we include the
    \eda~framework.
} 


\begin{figure}
\begin{center}
    \includegraphics[width=0.45\textwidth]{figs/abc_Lir.pdf}
    \caption{\label{fig:lir}
    IR dust emission luminosity predicted by the \eda~with median parameter
    values of the TNG (blue) and EAGLE (green) posteriors as a function of
    $M_r$. The dust emission is estimated assuming the \cite{dacunha2008}
    energy balance.  Despite reproducing the same SDSS UV and optical
    color-magnitude relations, \emph{the \eda~predicts significantly different
    IR dust emission for TNG and EAGLE}. Therefore, including IR
    observations will significantly improve the constraints on \eda~parameters
    and allow us to better differentiate galaxy formation models.
    }
\end{center}
\end{figure}

%There's hope! 
Fortunately, there are many avenues for improving our understanding of dust
with a forward modeling approach. In this work, we used a restrictive $M_r <
-20$ complete SDSS galaxy sample. 
\ch{
    Figure~\ref{fig:avmsfr} illustrates that our completeness cut limits the
    number of quiescent galaxies below $M_* < 10^{11}M_\odot$.  
}
Instead of imposing a completeness limit, 
we can include the actual SDSS selection function in the forward 
model~\citep[\eg~][]{dickey2020}. 
\ch{
    This would allow us to compare the simulations with \eda~to the entire SDSS
    sample, a substantially larger sample with a wider range of galaxies. 
}
Upcoming surveys, such as the Bright Galaxy Survey (BGS) of the Dark Energy
Spectroscopic Instrument~\citep[DESI;][]{desicollaboration2016, ruiz-macias2020} 
and galaxy evolution survey of the Prime Focus
Spectrograph~\citep[PFS;][]{takada2014,tamura2016}, will also vastly expand galaxy
observations. 
\ch{ 
    BGS, for instance, will measure $10\times$ the number of galaxy spectra as
    SDSS out to $z\sim0.4$  and with its $r\sim20$ magnitude limit will probe
    a significant number of low redshift dwarf galaxies. 
    Such an observational sample will allow us to place tigher constraints on
    the \eda~parameters, which may enable comparisons of the underlying galaxy
    formation models, 
    and shed light on dust in a broader range of galaxies. 
}

\ch{
    In this work, we also only used observables derived from UV and optical
    photometry, which means that we have only examined one side of the impact
    that dust has on galaxy spectra.
}
While dust attenuates light in the optical and UV, it emits light in
IR. In fact, even though the TNG and EAGLE simulations reproduce the same UV and
optical color-magnitude relations with the \eda, they predict significantly 
different dust emission in the IR. In Figure~\ref{fig:lir}, we present IR dust
emission luminosity, $L_{\rm IR}$, predicted by the \eda~with median parameter values of 
the TNG (blue) and EAGLE (green) posteriors as as a function of the $r$-band 
absolute magnitude, $M_r$. The dust emissions are estimated using the standard
energy balance assumption --- \ie~all starlight attenuated by dust is reemitted 
in the IR~\citep{dacunha2008}. 

Despite reproducing the same SDSS UV and optical color-magnitude relations, the
\eda~predicts significantly different IR dust emission for TNG and EAGLE. For
TNG, the \eda~predicts an overall ${\sim}0.3$ dex ($2\times$) higher dust
emissions than for EAGLE. Higher dust emission for TNG
is consistent with the higher $\ctau$ we infer for TNG (Figure~\ref{fig:abc}).
It is also consistent with the fact that TNG predicts bluer galaxies and more
luminous quiescent galaxies with red $\fnuv$ color than EAGLE
(Figure~\ref{fig:obs}). Since IR dust emission measures the total dust
attenuation, IR observations would specifically constrain the \eda~and
therefore break degeneracies between dust and the galaxy physics in simulations.
\ch{
    Upcoming will provide crucial observation on this front.  
    BGS, for instance, will have IR photometry from NEOWISE~\citep{meisner2018}. 
    \emph{James Webb Space Telecope (JWST)} will also provide valuable IR
    observations.
}

% Salim(2020): Chevallard et al. (2013), who aggregated and analyzed a diverse series of theoretical attenuation law studies by Pierini et al. (2004), Tuffs et al. (2004), Silva et al. (1998) and Jonsson et al. (2006), and showed that all the stud- ies predict, with some normalization differences, a relationship between the optical depth AV and attenuation law slope.

% Salmon+(2016): There is evidence that galaxy inclination correlates with the strength of Lyα emission, such that we observe less Lyα equivalent width for more edge-on galaxies (Charlot & Fall 1993; Laursen & Sommer-Larsen 2007; Yajima et al. 2012; Verhamme et al. 2012; U et al. 2015)
% Therefore, based on physi- cal models, one expects that galaxies with “greyer” dust laws and larger overall attenuation should have higher inclinations
% Salim+(2018): Chevallard et al. (2013) furthermore show that the depend- ence of the slope on AV is the same irrespective of whether the AV is driven by different levels of intrinsic (face-on) attenuation or is the result of inclined viewing geometry. 




%They find that
%star-forming galaxies with higher SFR have higher attenuation; however, this
%trend is driven by the $M_*$ dependence since star-forming galaxies lie on 
%the star-forming sequence~\citep{garn2010, battisti2017}. At fixed $M_*$,
%observations find no strong $\sfr$ dependence for the star-forming population. 
%Since previous works do not include quiescent galaxies, %\cite{tress2018} find positive dependence between E(B-V) (color excess) with M* and negative dependence with
%SSFR for 1753 star-forming galaxies within 1.5 < z < 3.  

%For instance, \cite{garn2010} find that SDSS star-forming galaxies with higher
%SFR have higher H$\alpha$ attenuation. \cite{battisti2016} find a consistent
%correlation for the Balmer optical depth of star-forming galaxies in GALEX and SDSS 
%using 10000 SF galaxies GALEX-SDSS. Although at higher redshifts,  
%\cite{reddy2015} also find this correlation among $z{\sim}2$ star-forming galaxies of
%MOSFIRE Deep Evolution Field.

%\cite{battisti2017} using 5000 SF galaxies from found little stellar mass dependence in the opposite direction (less attenuation at higher stellar masses). But they have big error bars and only probe up to 9.7
%\cite{reddy2015} SF galaxies from $z\sim2$ MOSFIRE Deep Evolution Field survey find strong correlation with SFR. %ionized gas is more reddened relative to the stellar continuum with increasing SFR 

%We can also examine the correlation between attenuation curve slopes and galaxy properties with the DEM. For TNG and EAGLE, we find that 
%\cite{leja2017} find composite and AGN galaxies generally have shallower  
%slopes, although their sample is limited to only 129 galaxies, In contrast,
%\cite{salim2018} find that quiescent galaxies in the GALEX-SDSS-WISE Legacy
%Catalog 2~\citep[GSWLC2;][]{salim2019} have significantly steeper curves. They,
%however, also find significantly steeper curves for the starburst population 
%(\ie~galaxies above the SFS). With no consensus in the literature and few
%observations that examine the correlation between $\delta$ and galaxy
%properties, the lack of $M_*$ and $\sfr$ dependence we find on $\delta$ is an
%interesting prediction for future observations. 


%\cite{calzetti2000} find  slopes of $2.3 < S < 2.9$ for low-redshift starburst galaxies, 
%\cite{burgarella2005} find $2.5 < S < 6.2$ for 50 UV and 100 IR selected galaxies,
%\cite{johnson2007} find $S\sim2.5$ for 1000 nearby galaxies, 
%\cite{conroy2010} find $S\sim4.5$ for 3400 $10^{9.5} < M_* < 10^{10} M_\odot$ disk galaxies,
%\cite{wild2011} find $2.5 < S < 4.5$ for 23,000 $z{\sim}0.07$ star-forming galaxies, 
%\cite{battisti2016, battisti2017} find slopes consistent with Calzetti (S=2.4) for SF galaxies 
%\cite{leja2017} 130 relatively massive galaxies 2 < S < 15
%\cite{salim2018} 230,000 SDSS galaxies 2 < S < 15 with median S = 5.4
% high z
%\cite{kriek2013}  using stacked SEDs of medium- and broadband photometry of
%galaxies at 0.5 < z < 2 find an average slope of delta=-0.2  but they restrict
%to galaxies with moderate to high optical attenuations (AV> 0.5), 
%\cite{salmon2016} is also spot on. 


% attenuation-slope relation 

%\cite{trcka2020}: EAGLE+SKIRT with CIGALE to get physical properties of
%galaxies. trcka2020 compares to dustpedia~\citep{davies2017} They lok at
%IRX-beta relation.

% variation of attenuation curve  
%\ch{what do we learn about quiescent galaxy attenuation?} 
%In \cite{leja2017}, they similar find composite and AGN galaxies
%to have shallow attenuation curves with higher $A_V$; however, the comparison
%is limited due to ther smaller sample size (129 galaxies). In contrast,  
%\cite{salim2018} find that quiescent galaxies in GSWLC2 have significantly
%steeper curves; they, however, focus their analysis mainly on star-forming 
%galaxies.

% Kartheik on why dust is hard to constrain for quiescent galaxies: 
%firstly because quiescent galaxies don't have much recent SFR and therefore not much dust (their most recently produced dust has likely happened more than a dust destruction timescale ago), 
% secondly because the continuum for quiescent galaxies is super hard to fit (and break degeneracies with SFH & metallicity), which leads to poor(er) constraints. You could basically think of this in terms of dust/total SNR, which drops sharply for this population.


% --- summary ---  
\section{Summary}
In this work, we present the DEM, an empirical framework for including dust
attenuation in simulated galaxy populations. It uses a parameterization of 
the attenuation curves motivated from observations~\citep[][]{noll2009} and a
flexible method for sampling the attenuation curve parameters that includes
correlations with galaxy properties ($M_*$ and $\sfr$). We apply the DEM to 
three state-of-the-art hydrodynamical simulations (SIMBA, TNG, and EAGLE) and
forward model the optical and UV color-magnitude relations. Afterwards, we
compare these forward modeled simulations to observed central galaxies in SDSS
using simulation-based inference. From the constraints on the DEM parameters,
we find the following results: 

\begin{itemize}
\item Dust attenuation is essential for simulations to reproduce observations.
With the DEM, we are able to reproduce SDSS observations for all three
hydrodynamical simulations. However, based on the inferred DEM parameter posteriors, SIMBA requires an
extreme dust attenuation that reverses the established relationship between
color and $\sfr$ because it overpredicts a large starburst population at
$<10^{10}M_\odot$.
\item Focusing on the DEM for TNG and EAGLE, we find significant $M_*$ and
$\sfr$ dependences in $A_V$. More massive galaxies have higher dust
attenuation; galaxies with lower $\sfr$ have  higher dust attenuation. 
\item The DEM attenuation curves are in good agreement with major observational 
constraints. They closely reproduce the observed attenuation--slope relation,
better than radiative transfer models. We also find significant variation in 
the attenuation curves as in observations. For star-forming galaxies, the DEM
atteunation curves are in good agreement with the literature.  
\item Lastly, the DEM is able to constrain the attenuation curves of quiescent
galaxies, which are poorly constrained by observations. We find that quiescent
galaxies have shallower attenuation curves with higher $A_V$ than star-forming
galaxies. 
\end{itemize}

Our results clearly demonstrate that the DEM can be used to provide insight
into dust attenuation. In this work, we are limited by the completeness limit
of the SDSS central galaxies. Upcoming surveys such as the Bright Galaxy Survey
of the Dark Energy Spectroscopic
Instrument~\citep[DESI;][\ch{Ruiz\etal2020}](desicollaboration2016), the Galaxy
Evolution Survey of the Prime Focus
Spectrograph~\citep[PFS;][]{takada2014,tamura2016}, and the Wide-Area VISTA
Extragalactic Survey~\citep[WAVES;][]{driver2016, driver2019} will provide
more statistically powerful observations at higher redshifts. With these
observations, the DEM will be able to tightly constrain and reveal new 
insights into dust attenuation. 

For those uninterested in dust, the DEM also provides a straightforward framework 
for marginalizing over dust. Although accounting for dust is necessy to reproduce 
observations, our current understanding of dust there broad enough that even 
simulations that predict galaxy populations with significantly different physical properties can
reproduce the same observable. Since varying dust alone can entirely reproduce 
observations, dust is highly degenerate with the variations in subgrid physics 
across simulations. After marginalizing over dust, observations do not have the
constraining power to differentiate between the various hydrodyanmical models.
Hence, detailed comparisons across simulations and to observations likely 
overinterpret the differences and similarities found in simulations. Therefore, 
we demonstrate that the current limitations in our understanding of dust is a 
major bottleneck for investigating galaxy formation using simulations.


\section*{Acknowledgements}
It's a pleasure to thank 
    Michael Blanton, 
    Nicholas T. Faucher, 
    Marla Geha, 
    Shy Genel,
    Jenny E. Green,
    Daniel Kelson, 
    Mariska Kriek, 
    Peter Melchior, 
    Desika Narayanan, 
    Samir Salim, 
    and Katherine Suess for
valuable discussions and comments.
This material is based upon work supported by the U.S. Department of Energy,
Office of Science, Office of High Energy Physics, under contract No.
DE-AC02-05CH11231. 
CH is supported by the AI Accelerator program of the Schmidt Futures Foundation.

We thank the Illustris collaboration and the Virgo Consortium for making their
simulation data publicly available, and the SIMBA collaboration for sharing
their data with us.
The EAGLE and SIMBA simulations were performed using the DiRAC-2 facility at
Durham, managed by the ICC, and the PRACE facility Curie based in France at
TGCC, CEA, Bruy\`{e}res-le-Ch\^{a}tel.

This research was supported in part through the computational resources and
staff contributions provided by the Quest high performance computing facility
at Northwestern University, which is jointly supported by the Office of the
Provost, the Office for Research, and Northwestern University Information
Technology. 

The data used in this work were, in part, hosted on facilities supported by the
Scientific Computing Core at the Flatiron Institute, a division of the Simons
Foundation, and the analysis was largely done using those facilities.
The IQ~(Isolated \& Quiescent)~Collaboratory thanks the Flatiron Institute for hosting the collaboratory and its meetings. 
The Flatiron Institute is supported by the Simons Foundation.
Funding for the Sloan Digital Sky Survey IV has been provided by the Alfred P.
Sloan Foundation, the U.S. Department of Energy Office of Science, and the
Participating Institutions. SDSS acknowledges support and resources from the
Center for High-Performance Computing at the University of Utah. The SDSS web
site is www.sdss.org.
The SDSS is managed by the Astrophysical Research Consortium for the
Participating Institutions.

\appendix
%\section{Modeling Observational Uncertainties} \label{sec:unc}

%\section{Resolution Effects} \label{sec:res}

Figure demonstrating imprint SFR=0 leave on the observable space and 

how we deal with them so we can ignore them...

\begin{figure}
    \begin{center}
        \includegraphics[width=0.66\textwidth]{figs/slab_tnorm.pdf} 
        \caption{Comparison of $A_V$ distribution of SDSS star-forming
        galaxies (blue) to predictions from the slab model (Eq.~\ref{eq:slab};
        black). {\color{red} detail on how SDSS SF galaxies are classified.} 
        The slab model assumes that there's a slab of dust in front of a galaxy.
        We use $\tau_V=2$ for the slab model above. Regardless of $\tau_V$,
        however, the slab model predicts a significantly more asymmetric and peaked $A_V$ distribution
        than observations. Given this disagreement, {\em we include in our
        analysis a DEM with an empirical prescription for $A_V$ based on a truncated normal 
        distribution, which better reproduce the observed $A_V$ distribution} (Section~\ref{sec:nonslab}). }
        \label{fig:av_dist}
    \end{center}
\end{figure}

 
\section{Beyond the Slab DEM}  \label{sec:nonslab} 
A major assumption of our fiducial DEM is that we sample the amplitude of
attenuation from the slab model. The slab model makes the simplifying assumption 
that dust in galaxies are in a slab-like geometry and illuminated by the
stellar radiation source~\citep{somerville1999}. Then, for a given $\tau_V$,
the attenuation depends solely on the orientation of the galaxy. This
simplification, ignores any complexities in the star-to-dust geometry that
impact the shape of the attenuation curve (\todo{Witt Gordon 1996, 2000, Seon
Drain 2016}). 

Besides its simplifications, the slab model predicts $A_V$ distribution with
significant differences than the $A_V$ distributions measured from
observations. In Figure~\ref{fig:av_dist}, we compare the $A_V$
distribution predicted by the slab model (black) to the $A_V$ 
distribution of star-forming galaxies in our SDSS sample (blue). The $A_V$
values are derived using SED fitting from the \cite{brinchmann2004} MPA-JHU
catalog and \todo{how are the SF galaxies classified}. The slab model $A_V$ values are derived using Eq.~\ref{eq:slab}
and~\ref{eq:tauv} with $M_*$s and SFRs from the same SDSS sample and the
inclinations, $i$, are uniformly sampled over the range $[0, \pi/2]$. 
With $\{m_{\tau,1}, m_{\tau,2}, c_\tau\}$ chosen to reproduce the observed
$A_V$ distribution, the slab model can reproduce the overall shape. However, it
predicts an extended high $A_V$ tail not found in observations.

Given these shortcomings of the slab model, we want to ensure that our results
do not hinge on the slab model. Modeling the star-to-dust geometries with
increased complexities, however, would involve expensive hydrodynamic 
simulations and dust radiative transfer
calculations~\citep[\emph{e.g.}][]{narayanan2018}\todo{jonsson2006, rocha2008,
natale2015,hayward smith2015,hou2017,trayford2020}. We instead
take an empirical approach and implement a flexible model for sampling $A_V$
based on a truncated normal distribution: 
\begin{equation} \label{eq:tnorm}
    A_V \sim \mathcal{N}_T(\mu_{A_V}, \sigma_{A_V}) =
    \frac{\mathcal{N}(\mu_{A_V}, \sigma_{A_V})}{1 -
    \Phi\left(-\frac{\mu_{A_V}}{\sigma_{A_V}}\right)}.
\end{equation}
Here, $\mathcal{N}$ is the standard normal distribution and 
$\Phi(x) = \frac{1}{2}\left(1+{\rm erf}(x/\sqrt{2})\right)$ is the cumulative
distribution function of $\mathcal{N}$. $\mu_{A_V}$ and $\sigma_{A_V}$
are the mean and variance of the truncated normal distribution. Similar to
Eq.~\ref{eq:tauv}, we allow $\mu_{A_V}$ and $\sigma_{A_V}$ to depend on the
physical properties of galaxies: 
\begin{align}\label{eq:tnorm_param} 
    \mu_{A_V}       &= m_{\mu,1} (\log~M_* - 10.) + m_{\mu,2} \log~{\rm SFR} + c_\mu \\
    \sigma_{A_V}    &= m_{\sigma,1} (\log~M_* - 10.) + m_{\sigma,2} \log~{\rm SFR} + c_\sigma. 
\end{align}

The $A_V$ distribution from our truncated normal (orange dashed) closely
reproduces the observed SDSS $A_V$ distribution (Figure~\ref{fig:dem}). $N_T$
is able to reproduce the overall skewness but unlike the slab model, it does
not have a long high $A_V$ tail. With more free parameters and a functional
form that closely resembles the observed $A_V$ distribution, the truncated
normal model provides a flexible altnerative to the slab model and we include
it in our analysis.  

\begin{figure}
    \begin{center}
        \includegraphics[width=0.66\textwidth]{figs/slab_tnorm.pdf} 
        \caption{Comparison of $A_V$ distribution of SDSS star-forming
        galaxies (blue) to predictions from the slab model (Eq.~\ref{eq:slab};
        black). {\color{red} detail on how SDSS SF galaxies are classified.} 
        The slab model assumes that there's a slab of dust in front of a galaxy.
        We use $\tau_V=2$ for the slab model above. Regardless of $\tau_V$,
        however, the slab model predicts a significantly more asymmetric and peaked $A_V$ distribution
        than observations. Given this disagreement, {\em we include in our
        analysis a DEM with an empirical prescription for $A_V$ based on a truncated normal 
        distribution, which better reproduce the observed $A_V$ distribution} (Section~\ref{sec:nonslab}). }
        \label{fig:av_dist}
    \end{center}
\end{figure}

 


\bibliographystyle{mnras}
\bibliography{galpopfm} 

\allauthors
\end{document}
