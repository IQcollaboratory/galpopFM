\section{Introduction} \label{sec:intro} 
% paragraph on why dust is important 
Dust in the interstellar medium of a galaxy can dramatically impact its
spectral energy distribution (SED). % over the full wavelength range. % of the electromagnetic spectrum. 
The combined effect of dust on a galaxy's SED is typically described using
an attenuation curve, $A(\lambda)$, which has now been broadly characterized 
by observations. 
In UV, attenuation curves steeply rise due to absorption by small grains.
At $2175\AA$, in the near-UV (NUV), there is an absorption bump referred to as
the ``UV dust bump''. 
At longer optical wavelengths, the curves take on a power-law shape. 
Finally, dust reemits the attenuated light in the optical and UV in the
infrared~\citep[for an overview see][]{calzetti2001, draine2003,
galliano2018}.
By impacting the SED, dust also affects the physical properties of a galaxy
that are inferred from the SED, such as its star formation rate ($\sfr$),
stellar mass ($M_*$), or star formation history~\citep[see reviews by][]{walcher2011,
conroy2013}. 
Assumptions on dust attenuation can dramatically vary these
properties~\citep{kriek2013, reddy2015, salim2016, salim2020}.
Since these properties are the building blocks to our understanding of
galaxies and how they evolve, a better understanding of dust not only provides
insights into dust, but also underpins all galaxy studies.  

% paragraph on why dust is hard with backwards modeling observations and RT
To better understand dust in galaxies, many observational works have
examined trends between dust attenuation and galaxy properties.
For example, UV and optical attenuation as well as the slope of the attenuation
curve are found to correlate with galaxy properties such as $M_*$, $\sfr$, and
metallicity in star-forming galaxies~\citep[][for a recent review see
\citealt{salim2020}]{garn2010, wild2011, battisti2016, battisti2017}. 
%The slope of the attenuation curves in star-forming galaxies also correlate with galaxy properties, such as $M_*$, specific $\sfr$ ($\ssfr$), metallicity, and axial ratio~\citep{wild2011, battisti2017}. 
%Recently, \cite{salim2018} argue that these correlations stem from the underlying ``attenuation-slope relation'', a trend between the amplitude of attenuation and slope. 
Despite the progress, there is still no clear consensus on the connection
between dust attenuation and galaxy properties.
Furthermore, studies so far have focused mainly on star-forming galaxies and
little is known about dust attenuation in quiescent galaxies. 
A major limitation of observational approaches is that dust attenation and
galaxy properties measured from galaxy SEDs are model-dependent and subject to
variations, inconsistencies, and biases of different methodologies.
Different methods can lead to vastly different derived values, even for the
same observations~\citep[\eg][see also Appendix~\ref{sec:slab}]{speagle2014, katsianis2020}.
SED modeling can also impose undesirable priors on derived galaxy
properties~\citep{carnall2018, leja2019} and suffer from parameter degeneracies
that are poorly understood. 

%Furthermore, galaxy properties and dust attenuation from
%galaxy SEDs are model-dependent and subject to variations, inconsistencies, and biases of
%different methodologies, which can 
%lead to vastly different derived results even for the same observed galaxies~\citep[\eg][see also Section~\ref{sec:slab}]{speagle2014, katsianis2020}.
%SED fitting can also impose undesirable priors on derived galaxy
%properties~\citep{carnall2018, leja2019} and suffer from parameter
%degeneracies that are poorly understood. 

Significant progress has also been made in theoretically modeling dust. 
Simulations can now model the radiative transfer of stellar light
through a dusty ISM for a wide range of configurations:
from simple slab-like dust geometries~\citep[\eg][]{witt1996, witt2000,
seon2016} to 3D hydrodynamical simulations of entire
galaxies~\citep[\eg][]{jonsson2006, rocha2008, hayward2015, natale2015,
hou2017}. 
Radiative transfer models have even been applied to cosmological
hydrodynamical simulations~\cite[\eg][]{camps2015, narayanan2018,
cochrane2019, rodriguez-gomez2019, trayford2020}. 
Dust has also been examined in a cosmological context using 
semi-analytic models ~\citep[SAMs; \eg][]{granato2000, fontanot2009, wilkins2012,
gonzalez-perez2013, popping2017}. 
Yet there are still major limitations in modeling dust. 
Dust models in cosmological simulations currently do not reproduce the
redshift evolution of dust properties~(\citealp{somerville2012, yung2019,
vogelsberger2020}, but see \citealp{li2019}) and radiative transfer models
produce attenuation-slope relations that arei significantly steeper than observations. 
Many models also require significant hand-tuning (\eg~propagating rays/photons into
particular cells) and make assumptions on the underlying dust grain models~\citep[see][for a review]{steinacker2013}. 
Lastly, radiative transfer models are computationally expensive.  
Applying a range of radiative transfer dust models to multiple simulations for
comparisons would require huge computational resources.
Using Markov Chain Monte Carlo to sample them for parameter exploration or
to marginalize over the impact of dust would be prohibitive.  

We take a different approach from the observational and theoretical works
above --- \emph{we investigate dust attenuation using a forward modeling
approach to compare simulations to observations}.
Our ``forward model'' starts with three major large-scale hydrodynamical
simulations: EAGLE \citep{schaye2015}, IllustrisTNG \citep{nelson2019},
and SIMBA \citep{dave2019}. 
We use their outputs (\eg~star formation history) to build SEDs for each
simulated galaxy.
We then apply dust attenuation to all the SEDs using the Empirical Dust
Attenuation (\eda) framework, which we describe shortly.
We construct photometry from the attenuated SEDs and afterwards apply a 
realistic noise model and sample selection function to construct synthetic
photometric observations. 
Finally, we compare the synthetic observations to actual observations and
constrain the free parameters of our EDA prescription. 
The EDA parameter constraints then provide insight into dust attenation and its
connection to galaxies.

The Empirical Dust Attenuation framework provides a flexible and
computationally inexpensive prescription for statistically assigning
attenatuion curves to simulated galaxy based on their physical properties. 
% applying dust attenuation.  The \eda~statistically assigns attenuation curves
% to each simulated galaxy based on its physical properties.  The \eda~first
% assigns attenuation curves to every simulated galaxy. 
In this work, we use attenuation curves %with the \cite{noll2009} parameterization 
that depend on $M_*$, $\ssfr$, galaxy inclination, and the \eda~parameters.  
We use the $M_*$ and $\ssfr$ of simulated galaxies, but include stochasticity
in our assignment by randomly sampling inclination.
The \eda~parameters set the strength of the $M_*$ and $\ssfr$ dependence in the
amplitude and slope of the attenuation curves. 
Unlike radiative transfer models, the \eda~does not seek to produce realistic
dust attenuation for individual galaxies. 
However, as we later demonstrate, it produces realistic distributions of dust
attenuation for galaxy populations. 
The \eda~provides an empirical framework for dust attenuation, analogous to the
halo occupation frameworks in galaxy formation~\citep[for a review see ][]{wechsler2018}. 
%In this work, we use the \cite{noll2009} attenuation curve parameterization with amplitude and slope of the curves determined from the simulated galaxy's $M_*$ and $\ssfr$ as well as a randomly sampled inclination ($i$). 
%We use a parameterization from observational studies so that the
%\eda~attenuation curves can easily be compared to observational constraints. 
%The $M_*$ and $\ssfr$ dependence is also motivated by
%observations~\citep[\eg~][]{garn2010, wild2011, battisti2016, leja2017,
%salim2018, salim2020} and allows us to easily explore the correlation
%between galaxy properties and dust attenuation.
%After the assignment, we simply apply the attenuation curves to the SEDs of the simulated galaxies.

With a forward modeling approach, the comparison between simulations and
observations are made entirely in observational space, so it is not impacted by
the inconsistencies of observational methods for measuring galaxy properties. 
%The comparisons are made in observational space, so they are not impacted by the inconsistencies of observational methods in measuring galaxy properties.
Forward models can also directly include the selection functions and
observational systematic effects so they can account for these effects to
exploit the full observational data set.
Furthermore, unlike radiative transfer models, with the \eda, we can apply a
wide range of realistic dust attenuation curves to simulated galaxies in a
matter of seconds. 
We can easily explore and sample the dust parameter space and infer the
relationship between dust attenuation and galaxy properties. 
That is the focus of this paper. 
Beyond investigating dust, the \eda~also provides a framework where we can
treat dust as {\em nuisance} parameters and tractably marginalize over dust
attenuation. 
In the subsequent paper of the IQ series, Starkenburg et al. (in
preparation), we will use the \eda~framework to compare star formation
quenching in cosmological galaxy formation models after marginalizing over
dust attenuation. 

%In principle, a radiative transfer model can be used instead of the \eda~in
%the forward modeling approach; however, radiative transfer models are computationally expensive.  
%Applying a range of radiative transfer dust models to multiple simulations
%for comparisons would require huge computational resources.
%Using them with Monte Carlo methods for parameter exploration would be
%prohibitive.  
%On the other hand, with the \eda, we can apply a wide range of realistic
%dust attenuation curves to simulated galaxies in a matter of seconds. 
%We can easily explore and sample the dust parameter space and infer the
%relationship between dust attenuation and galaxy properties. 
%That is the focus of this paper. 
%Beyond investigating dust, the \eda~also provides a framework where we can
%treat dust as {\em nuisance} parameters and tractably marginalize over dust
%attenuation. 
%In the subsequent paper of the IQ series, Starkenburg et al. (in
%preparation), we will use the \eda~framework to compare star formation
%quenching in cosmological galaxy formation models after marginalizing over
%dust attenuation. 

In Section~\ref{sec:sims}, we describe the three large-scale cosmological
hydrodynamical simulations (SIMBA, IllustrisTNG, and EAGLE) that we use in
our forward model.
We also describe the observed SDSS galaxy sample used for comparison. 
Next, we present the \eda~prescription used in this work
(Section~\ref{sec:dem}) and the likelihood-free inference method for
comparing the simulations to observations (Section~\ref{sec:abc}). 
Finally, in Section~\ref{sec:results}, we present the results of our
comparison and discuss their implications on dust attenuation and its
connection to galaxy properties. 

%motivation for an empirical dust attenuation model
%\ch{
%    We take a different approach from the observational and theoretical works above.  
%    In the standard observational approach, galaxy properties, including dust
%    attenuation, are measured from galaxy SEDs. 
%    The measurements, however, are subject to variations, inconsistencies, and
%    biases of different methodologies, which can be significant even for the
%    same observations~\citep[\eg][]{salim2007, kennicutt2012, speagle2014,
%    flores2020, katsianis2020}.
%    SED fitting can also impose undesirable priors on derived galaxy
%    properties~\citep{carnall2018, leja2019} and suffer from parameter
%    degeneracies that are poorly understood. 
%    Meanwhile, radiative transfer models are computationally expensive
%    Applying a range of radiative transfer dust models to multiple simulations
%    for comparisons would require huge computational resources.
%    Using Markov Chain Monte Carlo to sample them for parameter exploration or
%    to marginalize over the impact of dust would be prohibitive.  
%}


%\ch{
%    \emph{Instead, in this paper we investigate dust attenuation using a
%    forward modeling approach to compare simulations to observations}.
%    Our ``forward model'' starts with three major large-scale hydrodynamical
%    simulations: EAGLE, IllustrisTNG, and SIMBA. 
%    We use their outputs (\eg~star formation history) to build SEDs for each
%    simulated galaxy.
%    %We take three major large-scale hydrodynamical simulations (EAGLE, IllustrisTNG, and SIMBA) and use their outputs (\eg~star formation history) to build SEDs for the simulated galaxies. 
%    We then apply dust attenuation to all the SEDs using an Empirical Dust
%    Attenuation (\eda) framework.
%    Finally, we construct synthetic photometric observations from the
%    attenuated SEDs by applying a realistic noise model and survey selection
%    function. 
%    Afterwards, we compare the synthetic observations from the forward model to
%    actual observations. 
%
%}
    

%The \eda~model, with its flexibility and speed, provides a crucial step in a
%{\em forward modeling approach} to comparing simulations to
%observations~\citep[\eg][]{nelson2018, baes2019, trcka2020, dickey2020}.
%In the standard approach, the galaxy properties (\eg~SFR, $M_*$) predicted by
%simulations are compared to those derived from observations. 
%In the forward
%modeling approach, observable quantities (\eg~magnitude, color) are {\em forward
%modeled} for each galaxy in the simulations; then the simulations are compared 
%to observations in observational space. With a forward modeling approach, 
%comparisons are not limited by variations, inconsistencies, and biases of different
%observational methods for measuring galaxy properties (\eg~different tracers of
%SFR or $M_*$). Selection functions and systematic effects can also be accounted
%for in the forward model. 


%and present the Empirical Dust
%Attenuation (\eda) model, a flexible framework for statistically applying dust
%attenuation to galaxy populations. The \eda~model assigns, to every galaxy, a
%different dust attenuation curve. We present attenuation curves that are parameterized with
%a functional-form used in observational studies~\citep{noll2009} and the
%parameters of the curve for each galaxy (\eg~optical depth, slope) are sampled
%from distributions set by the \eda~model parameters and the galaxy's properties.  
%By sampling the attenuation curve parameters, it produces realistic variations 
%among the attenuation curves. The \eda~does not seek to produce realistic dust 
%attenuation for individual galaxies, like radiative transfer models. Instead, 
%it aims to produce realistic dust attenuations for a large ensemble of galaxies
%that enables direct comparison across galaxy populations. 

%There are a number of advantages to our \eda~approach. The \eda~uses the same
%functional-form for the attenuation curves as observational studies, so
%predictions of the model can easily be compared to observational constraints. 
%We also formulate the \eda~parameters so that they are easily interpretable and
%directly reveal correlations between dust attenuation and galaxy properties.
%Lastly, the \eda~model is computationally inexpensive. 
%
%The \eda~model, with its flexibility and speed, provides a crucial step in a
%{\em forward modeling approach} to comparing simulations to
%observations~\citep[\eg][]{nelson2018, baes2019, trcka2020, dickey2020}.
%In the standard approach, the galaxy properties (\eg~SFR, $M_*$) predicted by
%simulations are compared to those derived from observations. In the forward
%modeling approach, observable quantities (\eg~magnitude, color) are {\em forward
%modeled} for each galaxy in the simulations; then the simulations are compared 
%to observations in observational space. With a forward modeling approach, 
%comparisons are not limited by variations, inconsistencies, and biases of different
%observational methods for measuring galaxy properties (\eg~different tracers of
%SFR or $M_*$). Selection functions and systematic effects can also be accounted
%for in the forward model. 
%
%Of course, to produce realistic observables of simulated galaxies, the forward
%model must include some prescription for applying dust attenuation. In
%principle, a radiative transfer model can be used for this purpose; however,
%its computational cost would severely limit any exploration of the ``dust
%parameters''. With the \eda, however, we can apply a wide range of realistic
%dust attenuation to simulated galaxies in a matter of seconds and easily
%explore and sample the \eda~parameter space to infer the 
%relationship between dust attenuation and galaxy properties. {\em For readers 
%uninterested in dust}, the \eda~provides a way to treat dust parameters as
%{\em nuisance} parameters and tractably marginalize over dust attenuation. 

%In this work, we present a simple \eda~model that uses the \cite{noll2009}
%attenuation curve parameterization and includes correlations between dust
%attenuation and galaxy $M_*$ and $\sfr$ (Section~\ref{sec:dem}). We apply 
%the \eda~separately to three state-of-the-art cosmological large-scale hydrodynamical 
%simulations (SIMBA, IllustrisTNG, and EAGLE), which we describe in
%Section~\ref{sec:sims}, and compare them to a volume-limited galaxy sample from SDSS
%and GALEX observations (Section~\ref{sec:obs}). In Section~\ref{sec:dem}, we
%describe our \eda~model in detail. Finally, in Section~\ref{sec:results}, we
%present the results of our comparison and discuss their implications for our
%understanding of dust attenuation as well as of its connection to galaxy properties. 
