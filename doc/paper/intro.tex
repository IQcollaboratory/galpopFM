\section{Introduction} \label{sec:intro} 


dust is important because....


assumptions on the attenuation curve can dramatically impact the physical
properties inferred from SED fitting~\citep[\eg][]{kriek2013, shivaei2015,
reddy2015, salim2020}. 


motivation for an empirical dust attenuation model

attenuation vs extinction. While extinction curves have been derived from
observations and theoretically, it's not easy to map this to attenuation
curves. Attenuation curves are a product of complicated empirical processes
since it accounts for light that gets scattered and star light that is not
obscured 

This makes modeling them in a complete physically motivated method expensive.
People have done it \cite{narayanan2018, trayford2020}. 
%One approach adopted is to incorporate dust creation, growth, destruction and dynamics into simulations directly (McKinnon et al. 2016, 2017; Aoyama et al. 2018, 2019; Hou et al. 2019; Dav ́e et al. 2019). This, however is a very compu- tationally expensive method, involving many processes that remain poorly understood. A simpler method, that can be applied more easily to large-scale cosmological simulations, is to model dust based on information on gas and stars from the simulation (Camps et al. 2016; Trayford et al. 2017; Liang et al. 2018; Narayanan et al. 2018a; Cochrane et al. 2019; Liang et al. 2018; Ma et al. 2019; Rodriguez-Gomez et al. 2019), and then perform radiative transfer in post pro- cessing.
some detail about the
radiative transfer method and such. But besides being expensive they have to
make a number of assumptions anyway. e.g. \cite{narayanan2018} assumes a fixed
extinction curve. 

Moreover, because the radiative transfer method is expensive it's hard to
compare many different simulations. Not only that, observables generated from
simulations that take into radative transfer dust models complicates simulation
to simulation comparisons. Because you're simultaneously comparing the galaxy
formation prescription and all the dust prescription. 

emphasize somewhere here that using a prescription trained one sim doesn't
translate well to other sims. 

Instead, we present a framework using flexible dust empirical models that
paints attenuation curves onto galaxies. describe at a high level how we are
parameterizing DEMs 

emphasize comparisons in observable space (\cite{nelson2018} talks about
importance of color)

talk about the advantages: extremely flexible so it can encompass the wide variety of
attenuation curves found in radiative transfer, easy to correlate the
attenuation curve with galaxy properties. 

Also DEMs make it possible to statistically apply attenuation curves for large
galaxy population. Putting this ontop of simulations, we can use them to
generate observables and compare them to observations to constrain the DEM. 
This framework allows us to learn about attenuation curves given a model for 
galaxy formation. 

The other way around also works. If you don't care about dust at all, DEM
provides a framework to easily marginalize over dust attenuation and treat dust
as a nuisance parameter. 

In this paper, we do above for multiple simulations. 

Starkenburg et la. in prep will use this framework to marginalize over dust and compare galaxy populations predicted by multiple
simulations . 

\ch{why do we only do centrals?} 

