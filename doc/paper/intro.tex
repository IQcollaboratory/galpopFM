\section{Introduction} \label{sec:intro} 
Dust in the interstellar medium of a galaxy can dramatically impact the
observed light from the galaxy over the full range of the electromagnetic spectrum. 
In the infrared (IR), dust produces emissions and from near-infrared (NIR) to
ultraviolet (UV), dust modifies the galaxy's stellar radiation through
absorption and scattering. Dust can therefore dramatically impact the physical
properties of galaxies we infer from optical and UV light, such as star
formation rate ($\sfr$), stellar mass ($M_*$), or stellar ages~\citep[see
reviews by][]{walcher2011, conroy2013}. Since these properties are the
building blocks for our understanding of galaxies and how they evolve, dust
inevitably plays a crucial role for galaxy evolution. Beyond providing insights
into the physical processes related to dust, better understanding dust is
consequential for all galaxy studies.

The combined effect of dust on the spectral energy distribution (SED) of a
galaxy is typically described using an attenuation curve, $A(\lambda)$.
Observations have now established the major features in $A(\lambda)$. In the UV, the 
curve steeply rises due to absorptions by small grains. At $2175\AA$, in the 
near-UV (NUV), there is absorption bump referred to as the UV dust bump. Then 
at longer wavelengths, the curves on take a power-law shape. For an overview, we 
refer to reviews by \cite{calzetti2001, draine2003, galliano2018}. Attenuation
curves, however, are not universal. Large statistical samples of local galaxies
find a wide range of attenuation curves~\citep{wild2011, battisti2017,
salim2018, salim2020}. Observations also reveal a diversity of attenuation
curve at high redshifts~\citep[\eg][]{reddy2015, salmon2016}. For a given
amount of optical attenuation, the far-UV (FUV) attenuation can be $\sim
2-7\times$ greater. 

% summarize studies that looked at the connection between attenuation curves
% and galaxy properties (wild2011 battisti2016, battisti2017, and
% salim2018). List the hodge-podge correlations. Talk about how Salim claims
% that the attenuation-slope relations account for everything, but there's no
% clear concensus. Restate the importance of understanding the attenuaion
% curves properly.
To understand the origin of this variation in attenuation curves, previous
works have examined the correlation between them and galaxy properties. 
Using 23,000 star-forming galaxies in the Sloan Digital
Sky Survey (SDSS), \cite{wild2011} find that the slope of the attenuation curves varies 
strongly with galaxy axial ratio and weakly with specific SFR. Similarly, 
\cite{battisti2017}, from 5,500 star-forming galaxies, find only tentative
trends with stellar age, specific star formation rate, stellar mass, and
metallicity. Meanwhile, \cite{salim2018}, from 230,000 galaxies in the
GALEX-SDSS-WISE Legacy Catalog 2 (GSWLC2), find a significant $M_*$ 
dependence on the slope. They argue that this dependence is caused by
underlying trends between the amplitude of the $V$-band attenuation ($A_V$) and
slope. Based on this ``attenuation-slope relation'', galaxies with higher $A_V$
have shallower slopes. Nevertheless, there is still no clear consensus on the
connection between attenuation curves and galaxy properties. Furthermore, most
studies so far have focused only on star-forming galaxies and little is known
about dust attenuation in quiescent galaxies.

%Assumptions on the attenuation curve can dramatically impact the physical properties inferred from fitting the SED~\citep[\eg][]{kriek2013, shivaei2015, reddy2015, salim2020}. Moreover, the more we understand about dust the more constraining power we can keep for 
Alongside observations, theoretical efforts that model radiative transfer of
stellar light through a dusty ISM also provide insights into dust attenuation.
Radiative transfer models span a wide range of geometric configurations of
stars and dust. For instance, models focused on isolating the physical effects
of dust have considered simple slab or shell-like dust geometries illuminated
by stellar radiation~\citep[\eg][]{witt1996, witt2000, seon2016}. Other models,
focused on modeling dust attenuation in galaxies as a whole, have applied 3D
dust radiative transfer in hydrodynamic simulations of idealized
galaxies~\citep[\eg][]{jonsson2006, rocha2008, hayward2015, natale2015,
hou2017}. Dust attenuation has also been examined in a cosmological context
using semi-analytic models (SAMs) that do not track baryonic growth directly 
but make simple physically motivated assumptions about the resulting galaxy
properties from dark matter growth~\citep[\eg][]{granato2000,
fontanot2009, wilkins2012, gonzalez-perez2013}. Lastly, radiative transfer
models have recently been applied to cosmological hydrodynamical
simulations~\cite[\eg]{camps2015, narayanan2018, trayford2020}. As in
observations, these simulations find significant variation in attenuation
curves. They also reproduce the shape of the attenuation-slope relation. 

% Why can't we use RT for everything. 
Despite their progress, there are still major challenges for radiative transfer
models. For instance, computational constraints can result in under-resolved
dust and radiation field grids. Many models require significant
hand-tuning(\eg~propagating rays/photons into particular cells). Assumptions in
the underlying dust grain models also produce systematic uncertainties that are
difficult to quanitfy~\citep[see][for a review]{steinacker2013}. Furthermore,
radiative transfer models produce attenuation-slope relations that are
significantly steeper than observations. Finally, radiative transfer models 
are computationally expensive. So applying multiple radiative transfer dust
models to multiple simulations for comparisons is prohibitive. Marginalizing over
the impact of dust, for comparisons between simulations and observations aimed 
at examining subgrid galaxy physics, would be intractable. 

%motivation for an empirical dust attenuation model
In this paper, we take a different approach than these radiative transfer
models. We present the Dust Empirical Model (DEM), which provides a framework
for statistically assigning dust attenuation to galaxy populations. It uses a
parameterization of the attenuation curves motivated from observational
studies~\citep[\eg][]{noll2009} and a flexible method for sampling the
attenuation curve parameters (\eg~optical depth, slope) that include 
correlations with galaxy properties. There are a number of advantages to the
DEM. Given the DEM parameterization, constraints on the DEM can be directly 
compared to observed attenuation curves and correlations to galaxy properties
can be easily interpreted. The DEM can also produce the wide range of 
attenuation curves found in both observations and radiative transfer models. 
Lastly, the empirical approach of the DEM makes it computationally inexpensive. 

The DEM can be included on top of simulations and used to forward model
observables and statistically compared to
observations~\citep[\eg][]{nelson2018, baes2019, trcka2020, dickey2020}. From
such comparisons, we can exploit the statistical power of large galaxy surveys 
(\eg~SDSS) to constrain the DEM. These constraints provide insights into
attenuation curves and their connection to galaxy properties. Besides 
providing insights into dust, the DEM can also be used to tractably marginalize
over the impact of dust by treating the DEM parameters as nuisance parameters. 
The posterior probability distributions of the DEM parameters, derived from 
the comparison to observations, can be used to effectively marginalize over 
the DEM and disentangle the effect of dust from the subgrid galaxy formation 
prescriptions of the simulations. 
%In fact, given the parameterization of the DEM, informative priors for the DEM parameters can even be directly derived from observational constraints on attenuation curves. 

In this work, we present a simple DEM that uses the \cite{noll2009}
attenuation curve parameterization and includes correlations with galaxy $M_*$
and $\sfr$. We apply the DEM to three state-of-the-art cosmological large-scale hydrodynamical
simulations, SIMBA, IllustrisTNG, and EAGLE, which we describe in
Section~\ref{sec:sims} and compare them to observed SDSS galaxies,
which we describe in Section~\ref{sec:obs}. In Section~\ref{sec:dem}, we
describe our DEM in detail. Finally, in Section~\ref{sec:results}, we present
the results of our comparison and discuss insights the DEM provides for our
understanding of dust attenuation as well as of its interplay with galaxy evolution in cosmological simulations. 

%Unlike other empirical approaches~\citep{trayford2015, trayford2020}, constraints on the DEM can be directly compared to observed attenuation curves
%\cite{trayford2015} uses an empirical dust model but doesn't implement an attenuation curve but rather multiplicative factors for the broadband photometry.
