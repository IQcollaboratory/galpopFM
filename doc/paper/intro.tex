\section{Introduction} \label{sec:intro} 
% paragraph on why dust is important 
\ch{ 
    Dust in the interstellar medium of a galaxy can dramatically impact its
    spectral energy distribution (SED). % over the full wavelength range. % of the electromagnetic spectrum. 
    The combined effect of dust on a galaxy's SED is typically described using
    an attenuation curve, $A(\lambda)$, which has now been broadly characterized 
    by observations. 
    In the UV, attenuation curves steeply rise due to absorption by small grains.
    At $2175\AA$, in the near-UV (NUV), there is an absorption bump referred to as
    the ``UV dust bump''. 
    At longer optical wavelengths, the curves take on a power-law shape. 
    Finally, dust reemits the attenuated light in the optical and UV in the
    infrared~\citep[for an overview see][]{calzetti2001, draine2003,
    galliano2018}.
    By impacting the SED, dust also affects the physical properties that are
    inferred from the SED, such as star formation rate ($\sfr$), stellar
    mass ($M_*$), or stellar ages~\citep[see reviews by][]{walcher2011, conroy2013}. 
    Assumptions on dust attenuation can dramatically vary these
    properties~\citep{kriek2013, reddy2015, salim2016, salim2020}.
}
Since these properties are the building blocks to our understanding of
galaxies and how they evolve, a better understanding of dust not only provides
insights into dust, but also underpins all galaxy studies.  

% paragraph on why dust is hard with backwards modeling observations and RT
\ch{
    To better understand dust in galaxies, many observational works have
    examined trends between dust attenuation and galaxy properties.
    For example, UV and optical attenuation are found to correlate with 
    $M_*$, $\sfr$, and metallicity in star-forming
    galaxies~\citep{garn2010, battisti2016}. 
    The slope of the attenuation curves in star-forming galaxies also correlate
    with galaxy properties, such as $M_*$, specific $\sfr$ ($\ssfr$),
    metallicity, and axial ratio~\citep{wild2011, battisti2017}. 
    Recently, \cite{salim2018} argue that these correlations stem from the
    underlying ``attenuation-slope relation'', a trend between the amplitude of
    attenuation and slope. 
    Despite the progress, there is still no clear consensus on the connection
    between attenuation curves and galaxy properties.
}
Also, studies so far have focused solely on star-forming galaxies and little is known
about dust attenuation in quiescent galaxies. 
\ch{
    Furthermore, galaxy properties and dust attenuation measured from
    galaxy SEDs are subject to variations, inconsistencies, and biases of
    different methodologies, which can be significant even for the same
    observations~\citep[\eg][]{speagle2014, katsianis2020}.
    SED fitting can also impose undesirable priors on derived galaxy
    properties~\citep{carnall2018, leja2019} and suffer from parameter
    degeneracies that are poorly understood. 
}

\ch{
    Significant progress has also been made in theoretically modeling dust. 
    Simulations can now model the radiative transfer of stellar light
    through a dusty ISM for a wide range of configurations:
    from simple slab-like dust geometries~\citep[\eg][]{witt1996, witt2000,
    seon2016} to 
    3D hydrodynamic simulations of entire galaxies~\citep[\eg][]{jonsson2006,
    rocha2008, hayward2015, natale2015, hou2017}. 
    Radiative transfer models have even been applied to cosmological
    hydrodynamical simulations~\cite[\eg][]{camps2015, narayanan2018,
    trayford2020}. 
    Dust has also been examined in a cosmological context using 
    semi-analytic models ~\citep[SAMs; \eg][]{granato2000, fontanot2009, wilkins2012,
    gonzalez-perez2013, popping2017}. 
    Yet there are still major limitations in modeling dust. 
    Dust models in cosmological simulations currently do not reproduce the
    redshift evolution of dust properties~\citep{somerville2012, yung2019,
    vogelsberger2020}. 
    Also, radiative transfer models produce attenuation-slope relations that are
    significantly steeper than observations. 
}
Many models also require significant hand-tuning (\eg~propagating rays/photons into
particular cells) and make assumptions on the underlying dust grain models~\citep[see][for a review]{steinacker2013}. 
%Lastly, radiative transfer models are computationally expensive.  
%Applying a range of radiative transfer dust models to multiple simulations for
%comparisons would require huge computational resources.
%Using Markov Chain Monte Carlo to sample them for parameter exploration or
%to marginalize over the impact of dust would be prohibitive.  

\ch{
    We take a different approach from the observational and theoretical works
    above --- \emph{we investigate dust attenuation using a forward modeling
    approach to compare simulations to observations}.
    Our ``forward model'' starts with three major large-scale hydrodynamical
    simulations: EAGLE, IllustrisTNG, and SIMBA. 
    We use their outputs (\eg~star formation history) to build SEDs for each
    simulated galaxy.
    %We take three major large-scale hydrodynamical simulations (EAGLE, IllustrisTNG, and SIMBA) and use their outputs (\eg~star formation history) to build SEDs for the simulated galaxies. 
    We then apply dust attenuation to all the SEDs using an Empirical Dust
    Attenuation framework, which we describe shortly.
    Finally, we apply a realistic noise model and survey selection function on
    the attenudated SEDs and construct synthetic photometric observations. 
    Afterwards, we compare the synthetic observations from the forward model to
    actual observations. 
    The comparisons are made in observational space, so they are not impacted
    by the inconsistencies of observational methods in measuring galaxy
    properties.
    Furthermore, since the forward models can directly include the selection
    functions and observational systematic effects, forward modeling makes it
    easier to account for these effects and to exploit the full observational
    data set.
}

\ch{ 
    An essential step in our forward model is applying the Empirical Dust
    Attenuation (\eda) framework, which provides a flexible and computationally
    inexpensive prescription for applying dust attenuation.  
    The \eda~first assigns attenuation curves to every simulated galaxy. 
    In this work, we use the \cite{noll2009} parameterization for the
    attenuation curves and determine the amplitude and slope of the curves from
    the simulated galaxy's $M_*$ and specific star formation
    rate ($\ssfr$) as well as a randomly sampled inclination ($i$). 
    We use the same parameterization as observational studies so that the
    \eda~attenuation curves can easily be compared to observational constraints. 
    Also, the $M_*$ and $\ssfr$ dependence is motivated by
    observations~\citep[\eg~][]{garn2010, wild2011, battisti2016, leja2017,
    salim2018, salim2020} and allows us to easily explore the correlation
    between galaxy properties and dust attenuation.
    After the assignment, we simply apply the attenuation curves to the SEDs of
    the simulated galaxies.
    The \eda, unlike radiative transfer models, does not produce realistic dust
    attenuation for individual galaxies. 
    However, as we later demonstrate, it produces realistic distributions of
    dust attenuation for galaxy populations. 
    %Unlike radiative transfer models, however, the \eda~does not produce realistic dust attenuation for individual galaxies.  
    The \eda~provides an empirical mapping framework for dust attenuation,
    analogous to the halo occupation or abundance matching frameworks in 
    galaxy formation~\citep[for a review see ][]{wechsler2018}. 
}

\ch{
    In principle, a radiative transfer model can be used instead of the \eda~in
    the forward modeling approach; however, radiative transfer models are computationally expensive.  
    Applying a range of radiative transfer dust models to multiple simulations
    for comparisons would require huge computational resources.
    Using them with Monte Carlo methods for parameter exploration would be
    prohibitive.  
    On the other hand, with the \eda, we can apply a wide range of realistic
    dust attenuation to simulated galaxies in a matter of seconds. 
    We can easily explore and sample the dust parameter space and infer the
    relationship between dust attenuation and galaxy properties. 
    That is the focus of this paper. 
    Beyond investigating dust, the \eda~also provides a framework where we can
    treat dust as {\em nuisance} parameters and tractably marginalize over dust
    attenuation. 
    In the subsequent paper of the IQ series, Starkenburg et al. (in
    preparation), we will use the \eda~framework to compare star formation
    quenching in cosmological galaxy formation models after marginalizing over
    dust attenuation. 
}

\ch{
    In Section~\ref{sec:sims}, we describe the three large-scale cosmological
    hydrodynamical simulations (SIMBA, IllustrisTNG, and EAGLE) that we use in
    our forward model.
    We also describe the observed SDSS galaxy sample used for the comparison. 
    Next, we present the specific \eda~prescription used in this work in
    Section~\ref{sec:dem} and the likelihood-free inference method for
    comparing the simulations to observations in Section~\ref{sec:abc}. 
    Finally, in Section~\ref{sec:results}, we present the results of our
    comparison and discuss their implications on dust attenuation and its
    connection to galaxy properties. 
}

%motivation for an empirical dust attenuation model
%\ch{
%    We take a different approach from the observational and theoretical works above.  
%    In the standard observational approach, galaxy properties, including dust
%    attenuation, are measured from galaxy SEDs. 
%    The measurements, however, are subject to variations, inconsistencies, and
%    biases of different methodologies, which can be significant even for the
%    same observations~\citep[\eg][]{salim2007, kennicutt2012, speagle2014,
%    flores2020, katsianis2020}.
%    SED fitting can also impose undesirable priors on derived galaxy
%    properties~\citep{carnall2018, leja2019} and suffer from parameter
%    degeneracies that are poorly understood. 
%    Meanwhile, radiative transfer models are computationally expensive
%    Applying a range of radiative transfer dust models to multiple simulations
%    for comparisons would require huge computational resources.
%    Using Markov Chain Monte Carlo to sample them for parameter exploration or
%    to marginalize over the impact of dust would be prohibitive.  
%}


%\ch{
%    \emph{Instead, in this paper we investigate dust attenuation using a
%    forward modeling approach to compare simulations to observations}.
%    Our ``forward model'' starts with three major large-scale hydrodynamical
%    simulations: EAGLE, IllustrisTNG, and SIMBA. 
%    We use their outputs (\eg~star formation history) to build SEDs for each
%    simulated galaxy.
%    %We take three major large-scale hydrodynamical simulations (EAGLE, IllustrisTNG, and SIMBA) and use their outputs (\eg~star formation history) to build SEDs for the simulated galaxies. 
%    We then apply dust attenuation to all the SEDs using an Empirical Dust
%    Attenuation (\eda) framework.
%    Finally, we construct synthetic photometric observations from the
%    attenuated SEDs by applying a realistic noise model and survey selection
%    function. 
%    Afterwards, we compare the synthetic observations from the forward model to
%    actual observations. 
%
%}
    

%The \eda~model, with its flexibility and speed, provides a crucial step in a
%{\em forward modeling approach} to comparing simulations to
%observations~\citep[\eg][]{nelson2018, baes2019, trcka2020, dickey2020}.
%In the standard approach, the galaxy properties (\eg~SFR, $M_*$) predicted by
%simulations are compared to those derived from observations. 
%In the forward
%modeling approach, observable quantities (\eg~magnitude, color) are {\em forward
%modeled} for each galaxy in the simulations; then the simulations are compared 
%to observations in observational space. With a forward modeling approach, 
%comparisons are not limited by variations, inconsistencies, and biases of different
%observational methods for measuring galaxy properties (\eg~different tracers of
%SFR or $M_*$). Selection functions and systematic effects can also be accounted
%for in the forward model. 


%and present the Empirical Dust
%Attenuation (\eda) model, a flexible framework for statistically applying dust
%attenuation to galaxy populations. The \eda~model assigns, to every galaxy, a
%different dust attenuation curve. We present attenuation curves that are parameterized with
%a functional-form used in observational studies~\citep{noll2009} and the
%parameters of the curve for each galaxy (\eg~optical depth, slope) are sampled
%from distributions set by the \eda~model parameters and the galaxy's properties.  
%By sampling the attenuation curve parameters, it produces realistic variations 
%among the attenuation curves. The \eda~does not seek to produce realistic dust 
%attenuation for individual galaxies, like radiative transfer models. Instead, 
%it aims to produce realistic dust attenuations for a large ensemble of galaxies
%that enables direct comparison across galaxy populations. 

%There are a number of advantages to our \eda~approach. The \eda~uses the same
%functional-form for the attenuation curves as observational studies, so
%predictions of the model can easily be compared to observational constraints. 
%We also formulate the \eda~parameters so that they are easily interpretable and
%directly reveal correlations between dust attenuation and galaxy properties.
%Lastly, the \eda~model is computationally inexpensive. 
%
%The \eda~model, with its flexibility and speed, provides a crucial step in a
%{\em forward modeling approach} to comparing simulations to
%observations~\citep[\eg][]{nelson2018, baes2019, trcka2020, dickey2020}.
%In the standard approach, the galaxy properties (\eg~SFR, $M_*$) predicted by
%simulations are compared to those derived from observations. In the forward
%modeling approach, observable quantities (\eg~magnitude, color) are {\em forward
%modeled} for each galaxy in the simulations; then the simulations are compared 
%to observations in observational space. With a forward modeling approach, 
%comparisons are not limited by variations, inconsistencies, and biases of different
%observational methods for measuring galaxy properties (\eg~different tracers of
%SFR or $M_*$). Selection functions and systematic effects can also be accounted
%for in the forward model. 
%
%Of course, to produce realistic observables of simulated galaxies, the forward
%model must include some prescription for applying dust attenuation. In
%principle, a radiative transfer model can be used for this purpose; however,
%its computational cost would severely limit any exploration of the ``dust
%parameters''. With the \eda, however, we can apply a wide range of realistic
%dust attenuation to simulated galaxies in a matter of seconds and easily
%explore and sample the \eda~parameter space to infer the 
%relationship between dust attenuation and galaxy properties. {\em For readers 
%uninterested in dust}, the \eda~provides a way to treat dust parameters as
%{\em nuisance} parameters and tractably marginalize over dust attenuation. 

%In this work, we present a simple \eda~model that uses the \cite{noll2009}
%attenuation curve parameterization and includes correlations between dust
%attenuation and galaxy $M_*$ and $\sfr$ (Section~\ref{sec:dem}). We apply 
%the \eda~separately to three state-of-the-art cosmological large-scale hydrodynamical 
%simulations (SIMBA, IllustrisTNG, and EAGLE), which we describe in
%Section~\ref{sec:sims}, and compare them to a volume-limited galaxy sample from SDSS
%and GALEX observations (Section~\ref{sec:obs}). In Section~\ref{sec:dem}, we
%describe our \eda~model in detail. Finally, in Section~\ref{sec:results}, we
%present the results of our comparison and discuss their implications for our
%understanding of dust attenuation as well as of its connection to galaxy properties. 
