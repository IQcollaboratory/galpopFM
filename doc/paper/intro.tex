\section{Introduction} \label{sec:intro} 
Dust in the interstellar medium of a galaxy can dramatically impact the
observed light from the galaxy over the full range of the electromagnetic spectrum. 
Dust emits in the infrared (IR) and modifies the galaxy's stellar radiation
through absorption and scatter from the near-infrared (NIR) to ultraviolet
(UV). It can, therefore, dramatically impact the physical properties of
galaxies we infer from optical and UV light, such as star
formation rate ($\sfr$), stellar mass ($M_*$), or stellar ages~\citep[see
reviews by][]{walcher2011, conroy2013}. Since these properties are the
building blocks for our understanding of galaxies and how they evolve,
a better understanding of dust no only provides insights into the physical
processes related to dust, but it is also crucial to all galaxy studies.  

% dust inevitably plays a crucial role for galaxy evolution. Beyond providing insights into the physical processes related to dust, better understanding dust is consequential for all galaxy studies.

The combined effect of dust on the spectral energy distribution (SED) of a
galaxy is typically described using an attenuation curve, $A(\lambda)$.
Observations have now established the major features in $A(\lambda)$. In the UV, the 
curve steeply rises due to absorptions by small grains. At $2175\AA$, in the 
near-UV (NUV), there is absorption bump referred to as the UV dust bump. Then 
at longer wavelengths, the curves on take a power-law shape. For an overview, we 
refer to reviews by \cite{calzetti2001, draine2003, galliano2018}. Attenuation
curves, however, are not universal. Observations find a wide range of
attenuation curves in the local Universe~\citep{wild2011, battisti2017,
salim2018, salim2020} as well as at high redshifts~\citep[\eg][]{reddy2015,
salmon2016}. %For a given amount of optical attenuation, the far-UV (FUV) attenuation can be $\sim 2-7\times$ greater.

% summarize studies that looked at the connection between attenuation curves
% and galaxy properties (wild2011 battisti2016, battisti2017, and
% salim2018). List the hodge-podge correlations. Talk about how Salim claims
% that the attenuation-slope relations account for everything, but there's no
% clear concensus. Restate the importance of understanding the attenuaion
% curves properly.
To understand the origin of this variation in attenuation curves, a number of
observational  works have examined the correlation between dust attenuation and
galaxy properties. Using 23,000 star-forming galaxies in the Sloan Digital
Sky Survey (SDSS), \cite{wild2011} found that the slope of the attenuation
curves varies strongly with galaxy axial ratio and weakly with specific SFR.
Similarly, \cite{battisti2017}, from 5,500 star-forming galaxies, found tentative
trends with stellar age, specific star formation rate, stellar mass, and
metallicity. Meanwhile, \cite{salim2018}, from 230,000 galaxies in the
GALEX-SDSS-WISE Legacy Catalog 2 (GSWLC2), find a significant $M_*$ 
dependence on the slope. They argue that this is caused by the underlying 
``attenuation-slope relation'', a trend between the amplitude of the 
$V$-band attenuation ($A_V$) and slope where galaxies with higher $A_V$ have
shallower slopes. Nevertheless, there is still no clear consensus on the
connection between attenuation curves and galaxy properties. Also, studies 
so far have focused only on star-forming galaxies and little is known
about dust attenuation in quiescent galaxies.

%Assumptions on the attenuation curve can dramatically impact the physical properties inferred from fitting the SED~\citep[\eg][]{kriek2013, shivaei2015, reddy2015, salim2020}. Moreover, the more we understand about dust the more constraining power we can keep for 
Alongside observations, theoretical efforts that model radiative transfer of
stellar light through a dusty ISM also provide insights into dust attenuation.
Radiative transfer models span a wide range of geometric configurations of
stars and dust. For instance, models focused on isolating the physical effects
of dust have considered simple slab or shell-like dust geometries illuminated
by stellar radiation~\citep[\eg][]{witt1996, witt2000, seon2016}. Other models,
focused on modeling dust attenuation in galaxies as a whole, have applied 3D
dust radiative transfer in hydrodynamic simulations of idealized
galaxies~\citep[\eg][]{jonsson2006, rocha2008, hayward2015, natale2015,
hou2017}. Dust attenuation has also been examined in a cosmological context
using semi-analytic models (SAMs) that do not track baryonic growth directly 
but make simple physically motivated assumptions about the resulting galaxy
properties from dark matter growth~\citep[\eg][]{granato2000,
fontanot2009, wilkins2012, gonzalez-perez2013}. Lastly, radiative transfer
models have recently been applied to cosmological hydrodynamical
simulations~\cite[\eg][]{camps2015, narayanan2018, trayford2020}. As in
observations, simulations find significant variation in attenuation
curves. %They also reproduce the shape of the attenuation-slope relation. 

% Why can't we use RT for everything. 
Despite their progress, there are still major challenges for radiative
transfer models. For instance, they produce attenuation-slope relations that
are significantly steeper than observations. Also, %computational constraints can result in under-resolved dust and radiation field grids. 
many models require significant hand-tuning (\eg~propagating rays/photons into
particular cells). Assumptions in the underlying dust grain models also produce
systematic uncertainties that are difficult to quanitfy~\citep[see][for a
review]{steinacker2013}. Lastly, radiative transfer models 
are computationally expensive. So applying multiple radiative transfer dust
models to multiple simulations for comparisons is prohibitive. Using Markov
Chain Monte Carlo sampling with them for parameter exploration or to marginalize
over the impact of dust would be {\em intractable}. 

%motivation for an empirical dust attenuation model
In this paper, we take a different approach and present the Empirical Dust
Attenuation (\eda) model, a flexible framework for statistically applying dust
attenuation to galaxy populations. The \eda~model assigns, to every galaxy, a
different dust attenuation curve. The attenuation curves are parameterized with
a function-form used in observational studies~\citep{noll2009} and the
parameters of the curve for each galaxy (\eg~optical depth, slope) is sampled
from distributions set by the \eda~model parameters and the galaxy's properties.  
By sampling the attenuation curve parameters, it produces realistic variations 
among the attenuation curves. We emphasize that the \eda~does not seek to produce 
realistic dust attenuation for individual galaxies, like radiative transfer
models. Instead, it aims to produce realistic dust attenuations for the entire
galaxy sample and enable galaxy population-to-population comparisons.

There are a number of advantages to our \eda~approach. The \eda~uses the same
functional-form for the attenuation curves as observational studies, so
predictions of the model can easily be compared to observational constraints. 
We also formulate the \eda~parameters so that they are easily interpretable and
directly reveal correlations between dust attenuation and galaxy properties.
Lastly, the \eda~model is computationally inexpensive. 

The \eda~model, with its flexibility and speed, provides a key crucial in a
{\em forward modeling approach} to comparing simulations to
observations~\citep[\eg][]{nelson2018, baes2019, trcka2020, dickey2020}..
In the standadrd approach, the galaxy properties (\eg~SFR, $M_*$) predicted by
simulations are compared to those derived from observations. Instead, in the
forward modeling approach, observables (\eg~magnitude, color) are {\em forward
modeled} for each galaxy in the simulations. Then the simulations are compared 
to observations directly in observational space. With a forward modeling approach, 
comparisons are not limited by variations, inconsistencies, and biases of different
observational methods for measuring galaxy properties (\eg~different tracers of
SFR or $M_*$). Selection functions and systematic effects can also be included 
in the forward model. 

Of course to produce realistic observables of simulated galaxies, the forward
model must include some prescription for applying dust attenuation. In
principle, a radiative transfer model can be used for this purpose; however,
its computational cost would severely limit any exploration of the ``dust
parameters''. With the \eda, however, we can apply a wide range of realistic
dust attenuation to simulated galaxies in a matter of seconds. We can,
therefore, easily explore and sample the \eda~parameter space to infer the 
relationship between dust attenuation and galaxy properties. {\em For readers 
uninterested in dust}, the \eda~provides a way to treat dust parameters as
{\em nuisance} parameters and tractably marginalize over them. In other words, 
the \eda~can be used to effectively marginalize over dust attenuation. 

In this work, we present a simple \eda~model that uses the \cite{noll2009}
attenuation curve parameterization and includes correlations between dust
attenuation and galaxy $M_*$ and $\sfr$ (Section~\ref{sec:dem}). We apply 
the \eda~to three state-of-the-art cosmological large-scale hydrodynamical 
simulations, SIMBA, IllustrisTNG, and EAGLE (Section~\ref{sec:sims}), and 
compare them to observed SDSS galaxies (Section~\ref{sec:obs}). In 
Section~\ref{sec:dem}, we describe our \eda~model in detail. Finally, in
Section~\ref{sec:results}, we present the results of our comparison and discuss 
their implications for our understanding of dust attenuation as well as of its
connection to galaxy properties. 
