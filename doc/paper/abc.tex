\section{Simulation-Based Inference: Approximate Bayesian Computation} \label{sec:abc}
With a forward modeling approach and the \eda~model, we can generate
observables for simulated galaxies and make an ``apples-to-apples'' compraison
to SDSS observations (Section~\ref{sec:sims}). For the comparison, which we can
use to infer \eda~parameters, we use Approximate Bayesian
Computation~\citep[hereafter ABC;][]{diggle1984, tavare1997, pritchard1999,
beaumont2009, delmoral2012}.  ABC is a simulation-based (or
``likelihood-free'') parameter inference framework that approximates the
posterior probability distribution, $p(\theta\given{\rm data})$, without
requiring evaluations of the likelihood.  Instead, ABC only requires a forward
model of the observed data, a prior that can be sampled, and a distance metric
that quantifies the ``closeness'' to the observed data. Since ABC does not
require evaluating the likelihood, it does not assume any functional form of
the likelihood, which can significantly bias the inferred
posterior~\citep[\eg][]{hahn2019}. It also enables us to estimate the posterior using
observables with difficult or intractable likelihoods~\citep{hahn2017a}

In the simplest version of ABC, with a rejection sample
framework~\citep{pritchard1999}, a proposal set of parameter values are drawn
from the prior. The forward model is run with the proposal parameter values.
The output of the forward model is then compared to the observed data
using the distance metric. If the distance is within some small distance
threshold, we keep the proposed parameters; otherwise, we discard them. 
Proposals are drawn until enough of them pass the threshold to sample the posterior. A rejection sampling
framework requires a large number of evaluations of the forward model, which
can be computationally costly. Many variations of ABC with more efficient
sampling strategies have now been applied to astronomy and
cosmology~\citep[\eg][]{cameron2012, weyant2013, ishida2015, lin2016, alsing2018}.
Among these methods, we use ABC in conjuction with Population Monte Carlo (PMC) 
importance sampling~\citep{hahn2017a, hahn2017b, hahn2019a}.

The forward model in our scenario starts with the galaxies from our
hydrodynamic simulations. We construct SEDs using their SFH and ZH,
apply dust attenuation with the \eda~model, and calculate the observables from
the attenuated SED. For each simulation and a set of \eda~parameters, our forward model
produces $G$, $R$, $NUV$, and $FUV$ absolute magnitudes. We use uninformative uniform priors on each of
the \eda~parameters and choose ranges to encompass constraints in the
literature. The prior ranges of $\mtaum, \mtaus, c_\tau$ are chosen to
conservatively include the $A_V$ range and $M_*$ and $\sfr$ dependence of
\cite{narayanan2018} and \cite{salim2020}. Meanwhile, the prior ranges of 
$\mdeltam, \mdeltas, c_\delta$ are chosen to conservatively include the $\delta$
range and $M_*$ and $\sfr$ dependence of \cite{leja2017} and \cite{salim2018}. 
We list the range of the priors in Table~\ref{tab:free_param}. We note that
uniform priors on the \eda~parameters do not necessarily produce uniform priors 
on $\tau_V$ or $\delta$~\citep[\eg][]{handley2019}. However, we are interested in
marginalizing over dust attenuation and understanding the dependence of dust
attenuation on galaxy properties, so we use uniformative priors on the
\eda~parameters and not on $\tau_V$ or $\delta$. 

\begin{figure}
\begin{center}
    \includegraphics[width=\textwidth]{figs/abc.pdf}
    \caption{\label{fig:abc}
    Posterior distributions of the \eda~parameters for the SIMBA (orange), TNG
    (blue), and EAGLE (green) hydrodynamical simulations. The contours mark the $68\%$
    and $95\%$ confidence intervals. The posteriors are derived using the
    simulation-based inference method: Approximate Bayesian Computation with Population Monte Carlo
    (Section~\ref{sec:abc}). We focus on the \eda~posteriors for TNG and EAGLE
    since the \eda~struggles to reproduce SDSS observations with SIMBA, which
    predicts an overabundance of starburst galaxies. Based on the posteriors, we find that \emph{galaxies with higher
    $M_*$ have overall higher dust attenuation and galaxies with higher $\ssfr$
    have steeper attenuation curves.}
    }
\end{center}
\end{figure}


ABC also requires a distance metric that quantifies the ``closeness'' of the
forward model output to the observed data. For our distance metric, we use the
L2 norm between the summary statistics, $X$, of the SDSS observation and our 
forward model: 
\begin{equation} \label{eq:distance}
    \rho(\theta_{\rm \eda}) = \left[X^{\rm SDSS} - X^{\rm FM}(\theta_{\rm
    \eda}) \right]^2.
\end{equation}
$\theta_{\rm \eda}$ are the \eda~parameters. 
The summary statistics are based on the optical and UV color-magnitude
relations, $(\gr)-R$ and $(\fnuv)-R$, of galaxies brighter than $M_r < -20$,
the completeness limit of our SDSS sample (Figure~\ref{fig:obs}).
More specifically, for $X$, we calculate the number density in 3D bins of $\gr$,
$\fnuv$, and $M_r$ with widths 0.0625, 0.25, and 0.5 mags. We choose this summary 
statistic to fully exploit the observable-space predicted by the forward model. 
Later in Section~\ref{sec:results} we discuss other potential observables. 

ABC-PMC begins with an arbitrarily large threshold $\epsilon_1$ and $N$ proposals 
$\bar{\theta}_1$ sampled from the prior distribution. Each proposal is
assigned a weight $w^i_1 = 1/N$. Then for subsequent iterations ($i > 1$), the 
threshold, $\epsilon_i$, is set to the median distance of the previous iteration's
proposals. New proposals are drawn from the previous iteration's proposals perturbed 
by a kernel and kept if their distance is blelow $\epsilon_i$. This is repeated
until we assemble a new set of $N$ proposals $\bar{\theta}_i$. The entire
process is repeated for the next iteration until convergence is confirmed. 
We use the Python implementation of
\cite{akeret2015}\footnote{https://abcpmc.readthedocs.io/en/latest/index.html}.
For further details on the ABC-PMC implementation, we refer readers to \cite{hahn2017b}
and \cite{hahn2019a}.
In Figure~\ref{fig:abc}, we present the posterior distributions of the \eda~parameters
derived from ABC-PMC for the SIMBA (orange), TNG (blue), and EAGLE (green) hydrodynamical 
simulations. The contours mark the $68\%$ and $95\%$ confidence intervals. 

