\section{Likelihood-Free Inference: Approximate Bayesian Computation} \label{sec:abc}
With our forward model, which includes the \eda~prescription for dust
attenuation, we can now generate synthetic observations for simulated
galaxies and make an ``apples-to-apples'' comparison to SDSS. Next, we want
to use this comparison to infer the posterior probability distribution of
the \eda~parameters. Typically in astronomy, this inference is done
assuming a Gaussian likelihood to compare the ``summary statistic''
(\eg~SMF) of the model to observations and some sampling method (\eg~Markov
Chain Monte Carlo) to estimate the posterior distribution. The functional form of the
likelihood, however, depends on the summary statistic and assuming an
incorrect form of the likelihood can significantly bias the inferred
posteriors~\citep[\eg][]{hahn2019}. In this work, we use the optical and UV
color-magnitude relations as our summary statistic. Since this statistic is
a three-dimensional histogram, the likelihood is {\em not} Gaussian.
Furthermore, since the bins are not independent, the true likelihood is
difficult to analytically write down.

Rather than \emph{incorrectly} assuming a Gaussian likelihood or attempting
to estimate the true likelihood of the optical and UV color-magnitude
relations, we use Approximate Bayesian Computation~\citep[hereafter
ABC;][]{diggle1984, tavare1997, pritchard1999, beaumont2009, delmoral2012}
for our inference. 
ABC is a likelihood-free (or ``simulation-based'') parameter inference
framework that approximates the posterior probability distribution, $p(\theta\given{\rm data})$, without
requiring evaluations of the likelihood.  Instead, ABC only requires a forward
model of the observed data, a prior that can be sampled, and a distance metric
that quantifies the ``closeness'' to the observed data. 
Since ABC does not require evaluating the likelihood, it does not assume
any functional form of the likelihood so we avoid any biases from such
assumptions. 
It also expands the summary statistics we can use to infer the posteriors and,
therefore, provides a general inference framework for a forward modeling 
approach. 

In the simplest version of ABC, with rejection sampling~\citep{pritchard1999}, a proposal set of parameter values are drawn
from the prior. The forward model is run with the proposal parameter values.
The output of the forward model is then compared to the observed data using
a distance metric. 
If the distance is within some small threshold, we keep the proposed
parameters; otherwise, we discard them.  Proposals are
drawn until enough pass the threshold to sample the posterior. A
rejection sampling framework requires a large number of evaluations of the
forward model, which
can be computationally costly. Many variations of ABC with more efficient
sampling strategies have now been applied to astronomy and
cosmology~\citep[\eg][]{cameron2012, weyant2013, ishida2015, lin2016, alsing2018}.
Among these methods, we use ABC with Population Monte Carlo (PMC) 
importance sampling~\citep{hahn2017a, hahn2017b, hahn2019a}.

ABC-PMC begins with an arbitrarily large threshold $\epsilon_1$ and $N$ proposals 
$\bar{\theta}_1$ sampled from the prior distribution. Each proposal is
assigned a weight $w^i_1 = 1/N$. Then for subsequent iterations ($n > 1$), the 
threshold, $\epsilon_n$, is set to the median distance of the previous iteration's
proposals. New proposals are drawn from the previous iteration's proposals perturbed 
by a kernel and kept if their distance is below $\epsilon_n$. This is repeated
until we assemble a new set of $N$ proposals $\bar{\theta}_n$. The entire
process is repeated for the next iteration until convergence is confirmed. 
We use the Python implementation of
\cite{akeret2015}\footnote{https://abcpmc.readthedocs.io/en/latest/index.html}.
For further details on the ABC-PMC implementation, we refer readers to \cite{hahn2017b}
and \cite{hahn2019a}.

\begin{figure}
\begin{center}
    \includegraphics[width=0.9\textwidth]{figs/abc.pdf}
    \caption{\label{fig:abc}
    Posterior distributions of the \eda~parameters for the SIMBA (orange), TNG
    (blue), and EAGLE (green) hydrodynamical simulations derived from comparing
    the simulations to SDSS with a forward modeling approach. 
    The \eda~parameters determine the $M_*$ dependence, $\ssfr$ dependence, and
    amplitude of $\tau_V$ and $\delta$ (Table~\ref{tab:free_param}). 
    The contours mark the $68$ and $95$ percentiles of the distributions. 
    The posteriors are derived from likelihood-free inference using Approximate
    Bayesian Computation with Population Monte Carlo (Section~\ref{sec:abc}). 
    }
\end{center}
\end{figure}

In this work, we use ABC-PMC with uninformative uniform priors on each of
the \eda~parameters and choose ranges that encompass constraints in the
literature.
The prior ranges of $\mtaum, \mtaus, c_\tau$ include the $A_V$ range and $M_*$
and $\sfr$ dependence of \cite{narayanan2018} and \cite{salim2020}. 
Meanwhile, the prior ranges of $\mdeltam, \mdeltas, c_\delta$ include the
$\delta$ range and $M_*$ and $\sfr$ dependence of \cite{leja2017} and
\cite{salim2018}. 
We list the range of the priors in Table~\ref{tab:free_param}. 
We use the forward model described in Section~\ref{sec:fm}, where we construct
SEDs for every simulated galaxy from SIMBA, TNG, and EAGLE, apply dust
attenuation with our \eda, calculate the observables ($M_r$, $\gr$, and
$\fnuv$), add realistic noise, and apply a $M_r < -20$ completeness limit. 
We use the optical and UV color-magnitude relation, $(\gr)- M_r$ and
$(\fnuv)-M_r$ as our summary statistic to fully exploit the $(M_r, \gr,
\fnuv)$ observational-space. We measure the color-magnitude relations by
calculating the number density in bins of $(\gr, \fnuv, M_r)$ with widths
$(0.0625, 0.25, 0.5)~mags$. For our distance metric, $\rho$, we use the L2
norm between the number density of the SDSS observation, $n^{\rm SDSS}$ and
of our forward model, $n^{\rm FM}(\theta_{\rm \eda})$: 
\begin{equation} \label{eq:distance}
    \rho(\theta_{\rm \eda}) = \sum\limits_{i,j} \left[n_{ij}^{\rm SDSS} -
    n_{ij}^{\rm FM}(\theta_{\rm \eda}) \right]^2.
\end{equation}
In Figure~\ref{fig:abc}, we present the posterior distributions of the \eda~parameters
derived using ABC-PMC for the SIMBA (orange), TNG (blue), and EAGLE (green).
The contours mark the $68$ and $95$ percentiles of the distributions. 
